% !TEX root = apxthy.tex


\section{Algebraic Polynomials}
%
\label{sec:poly}
% 
Our second major topic concerns approximation of functions defined on an
interval $f : [-1, 1] \to \R$, without loss of generality. But contrary to
\S~\ref{sec:trig} we no longer assume periodicity. Instead we will approximate
$f$ by algebraic polynomials,
\[
      f(x) \approx p_N \in \Poly_N
\]
where $\Poly_N$ denotes the space of degree $N$ polynomials,
\[
   \Poly_N := \bg\{ \sum_{n = 0}^N c_n x^n \bsep c_n \in \R \bg\}.
\]
Note in particular that in the terms of "simplicity" these are indeed the 
simplest functions to evaluate numerically in that they only require addition 
and multiplication operations. 

In terms of a basic convergence result we have the following initial 
proposition, which we will not prove now, but it will follow from our 
later work.

\begin{proposition}[Weierstrass Approximation Theorem] \label{th:poly:Weierstrass}
   $\bigcup_{N \in \N} \Poly_N$ is dense in $C([-1,1])$ and by extension also 
   in $L^p(-1,1)$ for all $p \in [1, \infty)$.
\end{proposition}

\bigskip 

Indeed, as we have argued before, convergence in itself of {\em some} sequence
of approximations  is rarely useful, but we require (i) rates and (ii) explicit
constructions. Much of this chapter is therefore devoted to interpolation.

It is a standard fact (and easy to prove) that for any $N+1$ distinct points
$x_0, \dots, x_N \in \R$ and values $f_0, \dots, f_N$ there exists exactly one
polynomial $p_N \in \Poly_N$ interpolating those values, i.e., 
\[
   p_N(x_j) = f_j, \qquad j = 0, \dots, N.
\]
(Indeed, the same is even true for $x_j \in \C$.) These equations form 
a linear system for the coefficients $c_n$, which can be solved to obtain 
the interpolation polynomial, which in turn can be easily readily 
numerically. 

A key question is how to choose the interpolation points $x_j$? It may seem
intuitive to take equispaced nodes, $x_j = -1 + 2j/N$.  We start this section by
exploring precisely this approach to approximate some smooth functions on
$[-1,1]$; see \nbpoly for some motivating examples. In this Julia notebook we
clearly observe that this yields a divergent sequence of polynomials, but by
exploring also other kinds of fits we also see that this does not preclude the
possibility of computing a (very) good approximation. We therefore focus
initially by deriving a ``good'' set of interpolation nodes. The same idea will
also naturally lead to the Chebyshev polynomials.


\subsection{Chebyshev Points, Chebyshev Polynomials and Chebyshev Series}
%
We can motivate the idea of the Chebyshev points by mapping the polynomial
approximation problem to the trigonometric approximation problem:

Let $f\in C([-1,1])$, then let $g \in C(\TT)$ be defined by
\[
   g(\theta) = f(\cos\theta).
\]
Note that $g$ ``traverses'' $f$ twice!

We will later see that $g$ inherits the regularity of $f$ even across domain
boundaries; for now let us understand the consequence of this observation. We
know from \S~\ref{sec:trig} that equispaced interpolation of $g$ yields an
excellent trigonometric interpolant, i.e., we choose $\theta_j = -\pi + 2\pi
j/N$ and we choose coefficients $\hat{g}_k$ such that
\[
   t_N(\theta_j) = \sum_{-N}^N \hat{g}_k e^{ik \theta_j} = g(\theta_j)
\]
%
We may ask to interpolate $f$ at the analogous points, $x_j = \cos(\theta_j)$
but since $g$ contains ``two copies'' we only take half of the nodes.
This gives the Chebyshev nodes 
%
\begin{equation} \label{eq:poly:chebnodes}
   x_j := \cos\b( \pi j / N \b) \qquad j = 0, \dots, N.
\end{equation}

We can readily test our hypothesis that these yield much better approximations;
see again \nbpoly. Thus, for future reference we define the Chebyshev
interpolant $I_N f$ to be the unique function $I_ f \in \Poly_N$ such that
\[
   I_N f(x_j) = f(x_j) \qquad \for j = 0, \dots, N,
\]
where $x_j$ are the Chebyshev nodes \eqref{eq:poly:chebnodes}.


Next, we ask what the analogue of the Fourier series is. We write
\[
   g(\theta) = \sum_{k \in \Z} \hat{f}_k e^{ik\theta},
\]
then using that $g$ is real and $g(-\theta)=g(\theta)$,
\[
   g(\theta) = \hat{g}_0 + 2 \sum_{k = 1}^N \hat{g}_k \cos(k\theta)
\]
It is therefore natural to define the {\em Chebyshev polynomials}
%
\begin{equation} \label{eq:poly:defn_Tk}
   T_k(\cos\theta) = \cos(k\theta), \qquad k \in \N := \{0,1,2,\dots\}.
\end{equation}
%
A wide-ranging consequence of this definition is that
\[
      |T_k(x)| \leq 1 \qquad \forall k.
\]

\begin{lemma} \label{th:poly:chebpolys}
   The functions $T_k : [-1,1] \to \R$ are indeed polynomials and
   satisfy the recursion
   \begin{equation} \label{eq:poly:chebrecursion}
      T_{k+1}(x) = 2 x T_k(x) - T_{k-1}(x),
   \end{equation}
   with initial conditions $T_0(x) = 1, T_1(x) = x$.
\end{lemma}
\begin{proof}
   The identities $T_0(x) = 1, T_1(x) = x$ follow immediately from
   \eqref{eq:poly:defn_Tk}. If we can prove the recursion, then
   the fact that $T_k$ are polynomials follows as well.

   To that end, we introduce another representation,
   \[
      T_k\B( \smfrac{z + z^{-1}}{2} \B)
      = T_k(\Re z)
      = \Re z^k = \frac{z^k + z^{-k}}{2},
   \]
   where $|z| = 1$. Then,
   \begin{align*}
      & \hspace{-1cm} T_{k+1}(\Re z) - 2 \Re z T_k(\Re z) + T_{k-1}(\Re z) \\
      &= \smfrac12 \B(
         z^{k+1} + z^{-k-1}  - (z+z^{-1}) (z^k+z^{-k})
         + z^{k-1} + z^{-k+1} \B) \\
      &= \smfrac12 \B( z^{k+1} + z^{-k-1}
               - z^{k+1} - z^{k-1} - z^{1-k} - z^{-1-k}
               + z^{k-1} + z^{-k+1} \B) \\
      &=0. \qedhere
   \end{align*}
\end{proof}

For future reference we define the Joukowsky map
\[
   \phi(z) = \frac{z+z^{-1}}{2}.
\]
and note that it is analytic in $\C \setminus \{0\}$.

We now know that $T_k(x)$ are indeed polynomials of degree $k$ and in light of
the foregoing motivating discussion, we have the following result.

\begin{lemma}
   Let $f \in C([-1,1])$ is uniformly continuous, then there exists {\em
   Chebyshev coefficients} $\tilde{f}_k \in \R$ such that the {\em Chebyshev
   series}
   \begin{equation} \label{eq:poly:chebseries}
      f(x) = \sum_{k = 0}^\infty \tilde{f}_k T_k(x)
   \end{equation}
   is absolutely and uniformly convergent.

   The Chebyshev coefficients are given by the following equivalent formulas,
   \begin{align*}
      \tilde{f}_k
      &=  \frac{2}{\pi} \int_{-1}^1 \frac{f(x) T_k(x)}{\sqrt{1-x^2}} \,dx \\
      &=  \frac{1}{2\pi i} \oint_{\SS} \,\,\b(z^{-1+k} + z^{-1-k}\b) f(\phi(z))
                  \, dz \\
      &= \frac{1}{\pi i} \oint_{\SS} \,\, z^{-1+k} f(\phi(z)) \, dz \\
      &= \frac{1}{\pi i} \oint_{\SS} \,\, z^{-1-k} f(\phi(z)) \, dz.
   \end{align*}
   For $k = 0$ a factor $1/2$ must be applied.
\end{lemma}
\begin{proof}
   If $f \in C([-1,1])$ with modulus of continuous $\omega$, then $g \in C(\TT)$
   also has a modulus of continuity and hence the Fourier series converges
   uniformly and equivalently, the Chebyshev series does as well.

   The expressions for $\tilde{f}_k$ are simply transplanting the fourier
   coefficients $\hat{g}_k$ to Chebyshev coefficients $\tilde{f}_k$.
\end{proof}

In analogy with the truncation of the Fourier series $\Pi_N g$ (which
is the $L^2(\TT)$-projection or best-approximation we define
the Chebyshev projection
\[
   \PCheb_N f(x) := \sum_{k = 0}^N \tilde{f}_k T_k(x).
\]


\subsection{Convergence rates}
%
\label{sec:poly:rates}
%
As we have learned in \S~\ref{sec:trig}, the real power of polynomials is in
the approximation of analytic functions, hence we begin again with this
setting.

Intuitively, the idea is that analyticity of $f$ on $[-1,1]$ translates into
analyticity of the corresponding periodic function $g(\theta) = f(\cos\theta)$.
Exponential decay of the Fourier coefficients $\hat{g}_k$ then translates into
exponential decay of the Chebyshev coefficients  $\tilde{f}_k$. But we can prove
this exponential decay directly with a relatively straightforward variation of
the argument we used in \S~\ref{sec:trip:pw}, which is interesting to see the
analogies.


We begin by defining
\[
   F(z) := f(\Re z) = f\b( \smfrac12(z+z^{-1}) \b)
         = f(\phi(z)) \qquad \for z \in \SS := \{|z|=1\}.
\]
%
where $\phi(z) = \smfrac12 (z+z^{-1})$ is also called Joukowsky map. $\phi$ is
clearly analytic in $\C \setminus \{0\}$. Thus, if $f$ is analytic on $[-1,1]$
then $F$ must be analytic on $\SS$. Next, we note that analyticity of $g(\theta)$
on the strip $\Omega_\alpha$ is equivalent to analyticity of $F$ on the annulus
%
\[
   \SS_\rho := \{ z \in \C \sep \rho^{-1} \leq |z| \leq \rho \},
\]
%
with $\rho = 1+\alpha$. Let the corresponding {\em Bernstein ellipse} be the
pre-image of $\SS_\rho$ under the Joukowsky map,
%
\[
   E_\rho := \phi(\SS_\rho),
\]
%
then analyticity of $f$ in $E_\rho$ implies analyticity of $F$ in $\SS_\rho$.

Finally, we recall from the derivation of the Chebyshev polynomials $T_k(x)$
that they can also be written as
\[
   \smfrac12 (z^k + z^{-k}) = T_k(\phi(z)).
\]

After these preparations, we can prove the following result.


\begin{theorem}[Decay of Chebyshev coefficients]
   Let $\rho > 1$ and $f \in A(E_\rho)$ with  $M := \|f\|_{L^\infty(E_\rho)} <
   \infty$, then the Chebyshev coefficients of $f$ satisfy
   \[
      |\tilde{f}_k| \leq 2 M \rho^{-k}, \qquad k \geq 1.
   \]
\end{theorem}
\begin{proof}
   We start with
   \[
      \tilde{f}_k = \frac{1}{\pi i} \oint_{\SS} z^{-1-k} F(z) \, dz
   \]
   Since $F$ is analytic on $\SS_\rho$ (and hence in the neighbourhood of $\SS_\rho$)
   we can expand the contour to {\it (Exercise: explain why this can be done using
   Cauchy's integral formula and a suitable sketch!)}
   \[
      \tilde{f}_k = \frac{1}{\pi i} \oint_{|z|=\rho} z^{-1-k} F(z) \, dz
   \]
   and hence we immediately obtain
   \[
      |\tilde{f}_k| \leq \frac{2\pi \rho \rho^{-1-k} M}{\pi} = 2 M \rho^{-k}.
      \qedhere
   \]
\end{proof}

Decay of Chebyshev coefficients gives the following approximation error
estimates.

\begin{theorem}[Chebyshev Projection and Interpolation Error]
   %
   \label{th:poly:err_analytic}
   %
   Let $\rho > 1$ and $f \in A(E_\rho)$ with $M := \|f\|_{L^\infty(E_\rho)} <
   \infty$, then
   \begin{align}
      \label{eq:poly:projerror}
      \| f - \PCheb_N f \|_{L^\infty(-1,1)} &\leq \frac{2M \rho^{-N}}{\rho-1}, \\
      \label{eq:poly:interperror}
      \| f - I_N f \|_{L^\infty(-1,1)} &\leq C M \log N \rho^{-N},
   \end{align}
   where $C$ is a generic constant.
\end{theorem}
\begin{proof}
   For the proof of \eqref{eq:poly:projerror} we use the fact that
   $\|T_k\|_\infty \leq 1$ to estimate
   \begin{align*}
      \| f - \PCheb_N f \|_\infty
      &\leq
      \sum_{k = N+1}^\infty |\tilde{f}_k|  \\
      &\leq
      2M \sum_{k = N+1}^\infty \rho^{-k} \\
      &=
      \frac{2M \rho^{-N}}{\rho-1}.
   \end{align*}
   The estimate \eqref{eq:poly:interperror} follows from the bound on the
   Lebesgue constant
   \[
      \| I_N \|_{L(L^\infty)} \leq C \log N,
   \]
   which follows from the analogous bound for trigonometric interpolation
   given in Theorem~\ref{th:trig:lebesgue}.

   (For a sharp bound, it is in fact known that $\Lambda_N \leq \frac{2}{\pi}
   \log(N+1) + 1$.)
\end{proof}

\begin{remark}
   One can in fact prove that
   \[
      \| f - I_N f \|_{L^\infty(-1,1)} \leq \frac{4M \rho^{-N}}{\rho-1},
   \]
   using an aliasing argument; see \cite[Thm. 8.2]{Trefethen2013-rg},
   somewhat similar to the argument we used for our convergence estimate of
   the trapezoidal rule in Exercise~\ref{exr:trig:trapezoidal rule}.
\end{remark}

\begin{example}[Fermi-Dirac Function]
   %
   \label{exa:poly:fermi-dirac}
   %
   Consider the Fermi-Dirac function
   \begin{equation}
     f_\beta(x) = \frac{1}{1 + e^{\beta x}},
   \end{equation}
   where $\beta > 0$.
 
   {\it REMARK: The Fermi--Dirac function describes the distribution of particles
   over energy states in systems consisting of many identical particles that obey
   the Pauli exclusion principle, e.g., electrons. A broad range of important
   algorithms in computational physics are fundamentally about approximating the
   Fermi--Dirac function. The parameters $\beta$ is inverse proportional to
   temperature (that is, Fermi-temperature).}
 
   Extending $f_\beta$ to the complex plane simply involves replacing $x$ with
   $z$, i.e.,
   \[
     f_\beta(z) = \frac{1}{1 + e^{\beta z}},
   \]
   which is well-defined {\em except at the poles}
   \[
       z_j = \pm i  \frac{\pi}{\beta}.
   \]

   In Exercise~\ref{exr:poly:ellipse} we show that the semi-minor axis of the 
   Bernstein ellipse $E_\rho$ is $\frac12 (\rho-\rho^{-1})$, hence the largest 
   $\rho$ for which ${\rm int}E_\rho$ does not intersect any singularity is 
   given by 
   \[ \frac12 (\rho-\rho^{-1}) = \frac{\pi}{\beta}x. \]
   Solving this quadratic equation for $\rho$ yields one positive root 
   \[ \rho = \smfrac{\pi}{\beta}+\sqrt{1 + \smfrac{\pi^2}{\beta^2}}
   \]
   Of particular interest is the low temperature regime $\beta \to \infty$ 
   (recall that $\beta \propto$ inverse temperature), for which we obtain 
   \[
      \rho \sim 1 + \smfrac{\pi}{\beta}.
   \]
   
   In this regime we therefore expect an approximation rate close to 
   \[
      \| f_\beta - I_N f_\beta \|_{\infty} 
      \lesssim \beta \b(1 + \smfrac{\pi}{\beta}\b)^{-N}
      \sim \beta \exp\b( - \pi \beta^{-1} N).
   \] 
   (Why is this not a rigorous and in fact likely false bound? You can get 
   a rigorous reformulation from the foregoing theorems.)
 \end{example}


For convergence rates for $C^{j,\sigma}([-1,1])$ and similar functions, we
want to adapt the Jackson theorems. We could again "transplant" the argument
from the Fourier to the Chebyshev setting, but it will be more convenient
this time to simply apply the Fourier results directly. The details
are carried out in Exercise~\ref{exr:poly:convergence}. We obtain
the following result.

\begin{theorem}[Jackson's Theorem(s)]
   \label{th:poly:jackson}
   %
   Let $f \in C^{(j)}([-1,1])$, $j \geq 0$, where $f^{(j)}$ has modulus of
   continuity $\omega$, then
   \begin{equation}
      \label{eq:poly:jackson1}
      \inf_{p_N \in \Poly_N} \| f - p_N \|_{L^\infty} \leq
      C N^{-j} \omega\b(N^{-1}\b).
   \end{equation}
\end{theorem}
\begin{proof}
   See Exercise~\ref{exr:poly:convergence}.
\end{proof}

We cannot yet test these predictions numerically, since we don't yet have 
a numerically stable way to evaluate the Chebyshev interpolants (or projections). 
We will remedy this in the next two sections. 
 

\subsection{Chebyshev transform}
%
We have seen in \nbpoly that naive evaluation of the Chebyshev interpolant leads
to highly unstable numerical results. The emphasis here is on the term
``naive''. Indeed, there exist at least two natural and numerically stable way
to evaluate the Chebyshev interpolant.

The first approach we consider is the Discrete Chebyshev transform (DCT), an
immediate analogy of the Discrete Fourier transform (DFT). As in the Fourier case, 
once we have transformed the polynomial to the Chebyshev basis, we can 
evaluate it in $O(N)$ operations. But in the Chebyshev case, this is even more 
efficient due to the recursion formula \eqref{eq:poly:chebrecursion}. Moreover, 
the polynomial derivatives are straightforward to compute in this case as well.


Let $F = (F_j) \in \R^{N+1}$ (we immagine that $F_j = f(x_j)$ are nodal values
of some $f \in C([-1,1])$ at the Chebyshev nodes), then there exists a unique
polynomial $p_N \in \Poly_N$ such that $p_N(x_j) = F_j$. We write $p_N(x) =
\sum_{k = 0}^N \tilde{F}_k T_k(x)$, then
\begin{equation}
   \label{eq:poly:chebtransform}
   \tilde{F} := {\rm DCT}[F] := \b( \tilde{F}_k \b)_{k = 0}^N.
\end{equation}
Since polynomial interpolation is linear and unique the operator is
an invertible linear mapping, with inverse (obviously) given by
\begin{equation}
   \b({\rm IDCT}[\tilde{F}]\b)_j = \sum_{k = 0}^N \tilde{F}_k T_k(x_j).
\end{equation}

\begin{lemma} \label{th:poly:dct_explicit}
   Let $\tilde{F} = {\rm DCT}[F]$, then
   \[
      \tilde{F}_k = \frac{p_k}{N}\bg\{
            \smfrac12 \b( (-1)^k F_0 + F_N \b)
            + \sum_{k = 1}^{N-1} F_k T_k(x_j)
         \bg\}.
   \]
\end{lemma}

We won't prove \Cref{th:poly:dct_explicit} since we won't need this expression. 
It is only stated here for the sake of completeness. The interested reader 
will be able to check it by a direct computation; it is also implicitly 
contained in the following discussion. 


{\it A priori} the cost of evaluating the DCT and IDCT is $O(N^2)$, but the 
connection between the Fourier and Chebyshev settings gives us an $O(N\log N)$
algorithm which we now derive. Let $F = {\rm IDCT}[\tilde{F}]$, then writing 
\[
   T_k(x_j) = T_k(\cos(j\pi/N)) = \cos(kj\pi/N) 
\]
we obtain 
\begin{align}
   \label{eq:poly:costtransform}
   F_j 
   &= 
   \sum_{k = 0}^N \tilde{F}_k \cos(kj\pi/N) 
   \\ \notag &= 
   \sum_{k = 0}^N \tilde{F}_k \smfrac12 \b( e^{i2\pi kj/ (2N)} + e^{-i2\pi kj/(2N)}),
\end{align}
which looks {\em almost} like a IDFT on the grid $\{-N, \dots, N\}$. We 
can rewrite this a little more, 
\begin{align*}
   F_j
   &= 
   \tilde{F}_0 + \sum_{k = 1}^{N-1} \b[\smfrac12 \tilde{F}_k\b] e^{i2\pi kj/ (2N)}
   + \tilde{F}_N \smfrac12 \b( e^{i2\pi N j/ (2N)} + e^{-i2\pi N j/ (2N)} \b) 
   \\  & \qquad 
   + \sum_{k = -N+1}^{-1} \b[\smfrac12 \tilde{F}_{-k}\b] e^{i2\pi kj/ (2N)}
   \\ &= 
   \tilde{F}_0 + \sum_{k = 1}^{N-1} \b[\smfrac12 \tilde{F}_k\b] e^{i2\pi kj/ (2N)}
   + \tilde{F}_N e^{i2\pi N j/ (2N)}
   + \sum_{k = N+1}^{2N-1} \b[\smfrac12 \tilde{F}_{2N-k}\b] e^{i2\pi kj/ (2N)}
   \\ &=: 
   \sum_{k = 0}^{2N-1} \hat{G}_k e^{i2\pi kj/ (2N)},
\end{align*}
where we have defined 
\[
   \hat{G}_k := \cases{
      \tilde{F}_k, & k = 0, \\ 
      \smfrac12 \tilde{F}_k, & k = 1, \dots, N-1, \\ 
      \tilde{F}_k, & k = N, \\ 
      \smfrac12 \tilde{F}_{2N-k}, & k = N+1, \dots, 2N-1.
   }
\] 
Let $\hat{G}[\tilde{F}]$ be defined by this expression, then we have shown 
that 
\[
   F_j = \b({\rm IDCT}[\tilde{F}]\b)_j = 
   \b( {\rm IDFT}[\hat{G}[\tilde{F}]] \b)_j, \qquad j = 0, \dots, N.
\]
After determining $F_j$ for $j = N+1, \dots, 2N-1$ we can then evaluate the 
DCT via the DFT.  From the expression \eqref{eq:poly:costtransform} we 
immediately see that 
\begin{align*}
   F_{j}
   &= 
   \sum_{k = 0}^N \tilde{F}_k \cos(kj\pi/N - 2\pi k) 
   \\ &= 
   \sum_{k = 0}^N \tilde{F}_k \cos(k2\pi(j-2N)/2N) 
   \\ &= 
   \sum_{k = 0}^N \tilde{F}_k \cos(k2\pi(2N-j)/2N)
   \\ &= 
   F_{2N-j}
\end{align*}
That is, if we define 
\[
   G_j := \cases{
      F_j, & j = 0, \dots, N, \\ 
      F_{2N-j}, & j = N+1, \dots, 2N-1
   }
\]
then we obtain 
\[
   {\rm DFT}[G] = \hat{G},   
\]
from which we can readily obtain $\tilde{F}$. 

In {\tt Julia} code an $O(N\log N)$ scaling Chebyshev transform might 
look as follows: 

\begin{verbatim}
   "fast Chebyshev transform"
   function fct(F)
      N = length(F)-1
      G = [ F; F[N:-1:2] ]
      Ghat = real.(fft(F))
      return [Ghat[1]; 2 * Ghat[2:N]; Ghat[N+1]]
   end 

   "fast inverse Chebyshev transform"
   function ifct(Ftil)
      N = length(Ftil)-1
      Ghat = [Ftil[1]; 0.5 * Ftil[2:N]; Ftil[N+1]; 0.5*Ftil[N:-1:2]]
      G = real.(ifft(Ghat))
      return G[1:N+1]
   end
\end{verbatim}


\begin{remark}
   The expression \eqref{eq:poly:costtransform} is in fact another kind of 
   well-known transform, the {\em Discrete Cosine Transform} (one of several 
   variants). A practical implementation of the fast Chebyshev transform 
   should therefore use an efficient implementation of the fast cosine transform 
   rather than the FFT.
   For the sake  of simplicity (to avoid studying yet another transformation) 
   we did not study this transform in detail, but there is plenty of literature 
   and software available on this topic. 
\end{remark}



\subsection{Barycentric interpolation formula}
%
\label{sec:poly:bary}
%
The second method we discuss is the {\em barycentric interpolation formula}.
After precomputing some ``weights'' it gives another $O(N)$ method to evaluate
the Chebyshev interpolant (or indeed {\em any} polynomial interpolant) in a
numerically stable manner. This method entirely avoids the transformation to the
Chebyshev basis. (This section is taken almost verbatim from
\cite{Trefethen2013-rg}; see also \cite[Ch. 5]{Trefethen2013-rg} for a more
detailed, incl historical, discussion).

We begin with the usual Lagrange formula for the nodal interpolant. 
Let $p(x_j) = f_j, j = 0, \dots, N$ where $p \in \Poly_N$, then  
\[
   p(x) = \sum_{j = 0}^N f_j \ell_j(x), 
   \qquad \text{where} \quad 
   \ell_j(x) = \frac{ \prod_{n \neq j} (x - x_n)}{\prod_{n \neq j} (x_j-x_n)}.
\]
This formula has the downside that it costs $O(N^2)$ to evaluate $p$ at a 
single point $x$. 

But we observe that $\ell_j(x)$ have a lot of terms in common. This can be 
exploited by defining the {\em node polynomial}
\[
   \ell(x) := \prod_{n = 0}^N (x-x_n),
\]
then we obtain 
\begin{equation} \label{eq:poly:bary_weights}
   \ell_j(x) = \ell(x) \frac{\lambda_j}{x - x_j} 
   \qquad \text{where}  \qquad 
   \lambda_j = \frac{1}{\prod_{n \neq j} (x_j - x_n)}.
\end{equation}
The ``weights'' $\lambda_j$ still cost $O(N^2)$, but they are independent of $x$
and can therefore be precomputed (Moreover, for various important sets of nodes
there exist fast algorithms. For Chebyshev nodes there is an explicit
experession; see below.). Since the common factor $\ell(x)$ does not depend on
$j$ we can now evaluate all $\ell_j(x), j = 0, \dots, N$ at $O(N)$ cost and thus
obtain the {\em first form of the barycentric interpolation formula}, 
\begin{equation} \label{eq:poly:bary1}
   p(x) = \ell(x) \sum_{j = 0}^n \frac{\lambda_j}{x - x_j} f_j.
\end{equation}
Once the weights $\lambda_j$ have been precomputed, the cost of evaluating 
$p(x)$ becomes $O(N)$. However, \eqref{eq:poly:bary1} has a different 
shortcoming: in floating point arithmetic it is prone to overflow or underflow.
Specifically, suppose that $x = -1$ and we compute $\ell(x)$ with $x_j$ 
ordered decreasingly as defined in \eqref{eq:poly:chebnodes}, then after 
approximately the first $M \approx N/4$ terms we have evaluated 
\[
   \bg|\prod_{n = 0}^M (x - x_j) \bg|
   \geq \b( \smfrac34 \b)^{M+1} 
\]
which quickly becomes very large. The issue is also reflected in the
coefficients $\lambda_j$, which for Chebyshev points are $O(2^N)$ (cf.
Exercise~\ref{exr:poly:bary}). In practise, one typically gets overflow 
beyond 100 or so grid points. 

This can be avoided with the second form of the barycentric formula: observing
that $\sum_{j = 0}^N \ell_j \equiv 1$ we obtain 
\[
   1 = \ell(x) \sum_{j = 0}^N \frac{\lambda_j}{x-x_j}, 
\]
and hence arrive at the second form of the barycentric interpolation formula:

\begin{theorem}[Barycentric interpolation formula]
   \label{th:poly:bary}
   Let $p \in \Poly_N$, with $p(x_j) = f_j$ at $N+1$ distincts points 
   $\{x_j\}$ then 
   \[
      p(x) = \frac{ 
         \sum_{j = 0}^N \frac{\lambda_j f_j}{x-x_j} 
      }{
         \sum_{j = 0}^N \frac{\lambda_j}{x-x_j}
      }, 
      \qquad \text{where} \qquad 
      \lambda_j = \frac{1}{ \prod_{n \neq j} (x_j-x_n)},
   \] 
   with the special case $p(x_j) = f_j$.
\end{theorem}

\begin{theorem}[Barycentric interpolation formula in Chebyshev Points]
   \label{th:poly:barycheb}
   Let $\{x_j\}$ be the Chebyshev points \eqref{eq:poly:chebnodes}, then 
   the barycentric weights $\lambda_j$ from \Cref{th:poly:bary} 
   may be chosen as 
   \[
       \lambda_j = \cases{ 
          (-1)^j, & j = 1, \dots, N-1, \\ 
          \frac12 (-1)^j, & j = 0, N. }
   \]
\end{theorem}
\begin{proof}
   See Exercise~\ref{exr:poly:bary}. 
\end{proof}

\subsubsection{Numerical stability of barycentric interpolation}
%
\label{sec:poly:barystab}
%
While the DFT is matrix multiplication with an othogonal matrix, and the FFT 
an algorithm that even reduced the number of operations it is natural to 
expect that these algorithms are numerically stable. By contrast, this is 
not at all obvious {\it a priori} for the barycentric formula. We will therefore 
spend a little time discussing this. 
%
To simplify this discussion we will only analyse the numerical stability 
of the {\em first} barycentric formula \eqref{eq:poly:bary1}. Understanding 
stability of the second barycentric formula is slightly more involved; 
see \cite{Higham2004-fn} for the details. 

We have to begin by explaining the standard model of floating point arithmetic. 
Let $\otimes \in \{ +, -, *,  / \}$ be one of the standard four floating point 
operations, then applying the operation $a \otimes b$  to two floating point
numbers will give an error, which we express as 
\[
   {\rm fl}\b( a \otimes b \b) = (a\otimes b)(1+\delta),
\]
where $|\delta| \leq \eps$ and $\eps$ denotes machine precision (typically
$10^{-6}$). That is, floating point arithmetic controls the {\em relative
error}. For more on this topic, in particular additional subtleties that we are
sweeping under the carpet here, see \cite{Higham2002-nk}.


For example, consider the evaluation of an inner product of two vectors 
${\bf a}, {\bf b} \in \R^2$, 
\begin{align*}
   \fl({\bf a} \cdot {\bf b})
   &= \fl\b( \fl(a_1 b_1) + \fl(a_2b_2)\b) \\ 
   &= \fl\b( a_1 b_1 (1+\delta_1) + a_2b_2 (1+\delta_2)) \\ 
   &= \b( a_1 b_1 (1+\delta_1) + a_2b_2 (1+\delta_2))(1+\delta_3) \\ 
   &= a_1 b_1 (1+\delta_1)(1+\delta_3)
      + a_2b_2 (1+\delta_2)(1+\delta_3).
\end{align*}
Upon setting 
\[
   \tilde{a_1} = a_1 (1+\delta_1), \quad 
   \tilde{b_1} = b_1 (1+\delta_3), \quad 
   \tilde{a_2} = a_2 (1+\delta_2), \quad 
   \tilde{b_2} = b_2 (1+\delta_3),
\]
we obtain 
\[
   \fl({\bf a} \cdot {\bf b}) = \tilde{\bf a} \cdot \tilde{\bf b},
\]
where $\|{\bf a} - \tilde{\bf a}\| = O(\eps)$ and $\|{\bf b} - \tilde{\bf b}\| =
O(\eps)$. This is called {\em backward stability}: the numerically evaluated
quantity is the exact quantity for an exact computation with perturbed data. 

% As a second example we can consider 
% \begin{align*}
%    \fl\bg( \frac{f(x+h) - f(x)}{h} \bg)
%    &= \frac{(f(x+h)(1+\delta_1) - f(x))(1+\delta_2)}{h} (1+\delta_3)  \\ 
%    &= \frac{f(x+h) - f(x)}{h}(1+\delta_3) + 
% \end{align*}

We can now turn to the first barycentric formula. First we consider the 
evalutation of a weight $\ell(x)$, 
\begin{align}
   \fl\B( \prod_{n=0}^N (x-x_n) \B) 
   &= 
   \fl\bg( \fl\bg( \prod_{n=0}^{N-1} (x-x_n)\bg) * \fl(x-x_N) \bg)  \\ 
   &= 
   \fl\bg( \prod_{n=0}^{N-1} (x-x_n)\bg) * (x-x_N) (1+\delta_{1}) (1+\delta_{2}),
\end{align}
and by induction 
\begin{align}
   \fl\bg( \prod_{n=0}^N (x-x_n) \bg) = 
   \ell(x) \, \prod_{m = 1}^{2N+1} (1+\delta_m). 
\end{align}
The argument for $\lambda_j$ is of course analogous, hence we obtain
with a little extra work:

\begin{proposition} \label{th:poly:barystab}
   Let 
   \[
      \tilde{p}_N(x) := \fl\bg( \ell(x) \sum_{j = 0}^N \frac{\lambda_j}{x - x_j} \bg)
   \]
   be the numerically evaluated polynomial in the standard model of 
   floating point arithmetic, then 
   \[
      \tilde{p}_N(x) = 
      \ell(x) \sum_{j = 0}^N \frac{\lambda_j f_j}{x - x_j}
      \prod_{m = 1}^{5N+5} (1+\delta_{jm}).
   \]
\end{proposition}
\begin{proof}
   This is a straightforward continuation of the calculations above. 
\end{proof}

The key point of \Cref{th:poly:barystab} is that this is a {\em backward
stability} result, i.e., let $\tilde{f}_j = f_j\prod_{m = 1}^{5N+5}
(1+\delta_{jm})$, then $\hat{p}_N$ is interpolates the values $\tilde{f}_j$. 
In particular, the error in the floating point polynomial $\hat{p}_N(x)$ 
is no larger than if we had small errors in the nodal values $f_j$, which 
we will normally have anyhow. 

Finally, for the second barycentric formula, the numerical stability result is
weaker, but one can still show that for interpolation nodes with moderate
Lebesgue constant, and reasonable functions $f$ that we are interpolating, the
numerical stability is of no concern; see \cite{Higham2004-fn} for more details.



\subsection{Applications}

The following applications of the theory in this chapter will be 
covered in \nbpoly.

\begin{itemize}
   \item Evaluating special functions
   \item Approximating a Matrix function
   \item Spectral methods for BVPs; see also \cite[Sec. 21]{Trefethen2013-rg}
\end{itemize}

\noindent Further applications that could be explored in self-directed reading: 
\begin{itemize}
   \item Chebyshev filtering 
   \item Conjugate gradients and other Krylov methods 
   \item Quadrature: \cite{Trefethen2013-rg}, 
   \item Richardson extrapolation: \cite{Trefethen2013-rg}, p. 258
   \item \dots 
\end{itemize}

\subsection{Exercises}
%
% IDEAS FOR COMPUTATIONAL EXERCISES: 
% - test instability of Vandermonde interpolation 
% - super-exponential convergence for entire functions ... 
%   e^x should be easy to analyse too! Also e^{-x^2}!
%   http://www.chebfun.org/examples/approx/EntireBound.html


% \begin{exercise}
%    \label{exr:poly:vandermonde}
%    The seemingly ``canonical'' approach to constructing a polynomial interpolant
%    in the monomial basis is via the linear system 
%    \[
%       c_0 + c_1 x_j + c_2 x_j^2 + \cdots + c_N x_j^N = f_j.
%    \]
%    The system matrix $V = (x_j^n)_{j,n=0}^N$ is called a {\em Vandermonde matrix}. 
   
%    1. Suppose we take $x_j$ to be equispaced points on the complex unit circle, 
%    i.e., $x_j = e^{i2\pi j/N}$, $j = 0, \dots, N-1$ (i.e., $N \to N-1$!, then 
%    show that $V$ is the discrete Fourier transform operator, in particular it 
%    is unitary (up to rescaling) and thus has condition number $\kappa(V) = 1$.

%    2. 

%    Show that 

   
%    \alert{condition number of the Vandermonde matrix}.
% \end{exercise}

\begin{exercise}[Interpolation: Existence and Uniqueness] 
   \label{exr:poly:interpunique}
   Prove that for any collection of nodes $z_0, \dots, z_N \subset \C$ with $x_i
   \neq z_j$  for $i \neq j$, and nodal values $f_j$, there exists a unique
   interpolant $p \in \Poly_N$ such that $p(z_j) = f_j$. 
\end{exercise}

\begin{exercise}[Runge Phenomenon]
   \label{exr:poly:Runge Phenomenon}
   %
   For a partial explanation of the Runge phenomenon (cf \nbpoly) consider 
   the following steps: 
   \begin{enumerate} \ilist 
      \item Suppose $f \in C^{N+1}([-1,1])$. Prove that there exists 
      $\xi \in (-1,1)$ such that 
      \[
         f(x) - I_N f(x) =  \frac{f^{(N+1)}(\xi)}{(N+1)!} 
            \ell_N(x),
      \]
      where $\ell_N(x)$ is the node polynomial for the interpolation points. 

      {\it Hint: Let $e(t) = f(t)-I_N f(t)$ and show that $y(t) = e(t) - e(x)
      \ell(t) / \ell(x)$ has $N+2$ roots. What does this imply about the roots
      of $y^{(N+1)}$?}

      \item Prove that for equispaces nodes, $\|\ell_N\|_\infty \geq \frac14
      (N/2)^{-N-1} (N-1)!$.

      \item For $f(x) = 1 / (1+25 x^2)$ (The Witch of Agnesi), prove that $\|
       f^{(N+1)} \|_\infty \| \ell_N \|_\infty  / (N+1)! \to \infty$ as $N \to
       \infty$. {\it [HINT: $(1+c^2x^2)^{-1} = (1+cix)^{-1}+(1-cix)^{-1}$]}

       {\it (Note this does not prove divergence but proved a strong 
       hint why divergence occurs.)}
   \end{enumerate}
\end{exercise}


\begin{exercise}[Clenshaw's Algorithm]
   %
   \label{exr:poly:clenshaw}
   %
   Let $p \in \Poly_N$, $N \geq 1$, be given in the Chebyshev basis, with
   coefficients $(\tilde{f}_k)_{k = 0}^N$ and let $x \in [-1,1]$. Show that
   $p(x)$ can be evaluated by Clenshaw's algorithm:
   \begin{align*}
      & u_{N+1} = 0, \qquad u_N = \tilde{f}_N; \\ 
      & u_n = 2 x u_{n+1} - u_{n+2} + \tilde{f}_n, n = N-1, N-2, \dots, 0; \\ 
      & p(x) = \smfrac12 \b( \tilde{f}_0 + u_0 - u_2).   
   \end{align*}
   What is the purpose of the Clenshaw algorithm, i.e., the potential 
   advantage over simply summing over the Chebyshev basis?
\end{exercise}

\begin{exercise}[Orthogonality of $T_k$]
   Consider the weighted space 
   \begin{align*}
      L^2_C &:= \b\{ f : (-1,1) \to \R, \text{ measurable, } 
            \|f\|_{L^2_C} < \infty \b\}, \qquad 
            \text{where} \\  
      \|f\|_{L^2_C}^2 &:= \int_{-1}^1 \frac{|f|^2}{\sqrt{1 - x^2}} \,dx.
   \end{align*}
   Prove that $L^2_C$ is a Hilbert space and show that the Chebyshev 
   polynomials are (up to scaling) and orthonormal basis of this space. 

   Thus, conclude that the Chebyshev projection $\tilde\Pi_N$ is in fact that 
   best-approximation with respect to the $\|\cdot\|_{L^2_C}$-norm. 
\end{exercise}

\begin{exercise}[Bernstein Ellipse] 
   \label{exr:poly:ellipse}
   %
   Prove that the Bernstein Ellipse $E_\rho$, $\rho > 1$ is indeed an ellipse
   with centre $z = 0$, foci $\pm 1$, semi-major axis $\frac12 (\rho+\rho^{-1})$
   and semi-minor axis $\frac12 (\rho-\rho^{-1})$.
\end{exercise}



\begin{exercise}[Convergence Bounds]
   \label{exr:poly:convergence}
   \begin{enumerate} \ilist
   \item Complete the proof of \eqref{eq:poly:interperror} by proving
      \[
         \| I_N \|_{L(L^\infty)} \leq C \log N,
      \]
      where $I_N$ is the Chebyshev nodal interpolation operator.

   \item In preparation for the proofs of the best approximation error estimates
      for differentiable (non-analytic) functions, prove that, if $f \in
      C([-1,1])$ with modulus of continuity $\omega$, then $g \in C(\TT)$ and it
      has the same modulus of continuity.

   \item Prove Theorem~\ref{th:poly:jackson}, case $j = 0$.

   \item Let $E_N(f) := \inf_{p \in \Poly_N} \|f - p\|_\infty$. Prove that 
      \[
         E_N(f)  \leq C N^{-1} E_{N-1}(f'),
      \]
      where $C$ is independent of $N$ and try to quantify $C$.

   \item Complete the proof of Theorem~\ref{th:poly:jackson} (general $j$).
   Indeed, you should obtain a more precise formula.
   \end{enumerate}
\end{exercise}

\begin{exercise} 
   \label{exr:poly:examplefunctions}
   %
   \begin{enumerate} \ilist 
   \item For the following functions give bounds on the rate of polynomial best
   approximation in the max-norm, as sharp as you can manage: 
   \begin{enumerate} \ilist
      \item $f(x) = |\sin(5 x)|$ 
      \item $f(x) = \sqrt{|x|}$
      \item $f(x) = x (1 + 1000 (x - 1/2)^2)^{-1/2}$
      \item $f(x) = e^{- \cos(3x)}$
      \item $f(x) = x^{100}$
      \item $f(x) = e^{-x^2}$ 
      \item $f(x) = {\rm sign}(x)$
   \end{enumerate}
   \item and for the following two functions also in the $L^2$-norm:
   \begin{itemize}
      \item $f(x) = {\rm sign}(x)$
      \item $f(x) = \sqrt{|x|}$ \qedhere 
   \end{itemize}
   \end{enumerate}
\end{exercise}

\begin{exercise}[Barycentric Chebyshev Interpolation]
   %
   \label{exr:poly:bary}
   %
   Let $x_j$ be the Chebyshev points on $[-1,1]$.
   \begin{enumerate} \ilist
   \item In general (not only for Chebyshev points), demonstrate that the
   barycentric weights satisfy $\lambda_j = 1 / \ell'(x_j)$.
   \item Prove that the node polynomial satisfies 
   \[
      \ell(x) = 2^{-N} \b(T_{N+1}(x) - T_{N-1}(x)\b)
   \]
   \item Show that 
   \[
      T_{N+1}'(x_j) - T_{N-1}'(x_j) = 
      \cases{ 
         2N(-1)^j, & 1 \leq j \leq N-1, \\
         4N(-1)^j, & j = 0, N.
      }
   \]
   \item Deduce that, if $\lambda_j$ is given by \eqref{eq:poly:bary_weights}, then 
   \[
      \lambda_j = \frac{2^{N-1}}{N} (-1)^j, \qquad j = 1, \dots, N-1,
   \]
   and suitably adjusted for $j = 0, N$. Explain why we can rescale the weights
   $\lambda_j$ without changing the validity of the barycentric formula, and
   hence complete the proof of Theorem~\ref{th:poly:barycheb}.
   \end{enumerate}

   {\it WARNING: it turns out, this exercise needs more material than I
   realised, namely aliasing of Chebyshev coefficients. It is still very
   interesting so I will leave it here for now. An interested reader should
   follow to \cite[Sec. 5]{Trefethen2013-rg} to derive this formula.}
\end{exercise}


\begin{exercise}[Coordinate Tranformations]
   \label{exr:poly:coordinates}
   %
   The purpose of this exercise is to investigate how the choice of 
   coordinate systems can expand the range of approximable functions, as 
   well as have an affect on the rate of convergence.

   The basic idea is to considier functions $F : [a, b] \to \R$ and via a
   coordinate transformation $f(x) = F(\xi(x))$ transform them to functions $f :
   [-1,1] \to \R$. This can can multiple consequences, including: (1) we can
   represent functions on an arbitrary interval (inclusinf $\R$); (2) we can
   transform functions in such a way to increase the region of analyticity and
   thus accelerate convergence.
   %
   \begin{enumerate} \ilist
      \item Consider the Morse potential $F(y) = e^{-2\alpha y} - 2 e^{-\alpha
      y}$, then $F(y) = f(e^{-\alpha y})$ where $f(x) = x^2 - 2x$ is a quadratic
      polynomial. Suppose though that this ``optimal'' coordinate transform $x =
      e^{-y}$ is not known. 

      Instead, consider the Morse coordinate transformation $\xi^{-1}(y) = 2
      e^{-y} - 1$ and the transformed function $f(x) = F(\xi(x))$.
      
      coordinate transformation $x = 2/(1+y) - 1 =
      \xi^{-1}(y)$, that is, $\xi^{-1}(0) = 1, \xi^{-1}(\infty) = -1$, and 
      let $f(x) = F(\xi(x))$. 
      \begin{enumerate} \alist 
         \item Establish an upper bound (as sharp as you can manage) for
          approximation by Chebyshev projection and interpolation of $f$ on
          $[-1,1]$ in the max-norm. 
         \item Convert this bound to an approximation result for $F(y)$ on $[0,
          \infty)$. 
         \item Can you give a simpler / more direct characterisation of the
         effective approximation space for functions on $[0, \infty)$ that you
         used here?
      \end{enumerate}
      
      \item Now consider the function $F(y) = (\eps^2 + y^2)^{-1/2}$ on 
      $[-1, 1]$. Recall the rate of convergence of Chebyshev projection and 
      Chebyshev interpolation. 

      Now consider a coordinate transformation 
      \[
         \xi^{-1}(y) = \frac{\arctan(x/\eta)}{\arctan(1/\eta)},
      \]
      and explicitly compute its inverse. Show that $\xi, \xi^{-1} : [-1,1] \to
      [-1,1]$ are bijective. 

      \begin{enumerate} \alist 
         \item For any $\eta > 0$ establish an upper bound (as sharp as you can
         manage) for approximation of $f(x) = F(\xi(x))$ by Chebyshev projection
         and Chebyshev interpolation.
         
         \item Discuss which choices of $\eta$ appear to be particularly good.
         Visualise the effect of $\xi$ on the function $f$ as well as on 
         the singularities in the complex plane.
      \end{enumerate}
   \end{enumerate}
\end{exercise}

