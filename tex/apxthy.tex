% - review of background, in particular function spaces, regularity of functions
% Part 1: univariate approximation
% - spline approximation of smooth functions in 1D
% - polynomial and trigonometric approximation of analytic functions in 1D
% - linear best approximation
% - best n-term approximation (to be decided whether feasible)
% - multi-variate approximation by tensor products (low dimension)
% Part 2: Multi-variate approximation: cover a selection of the following:
% - mixed regularity, splines and sparse grids, Smolyak algorithm
% - radial basis functions and Gaussian processes
% - ridge functions and neural networks
% - compressed sensing and best n-term approximation
%
% Throughout the lecture each topic will cover (1) approximation rates, (2) algorithms, and (3) examples.

% Approximation Theory and Methods, M. J. D. Powell
% Approximation Theory and Approximation Practice, N. Trefethen
% A course in approximation theory, E.W.Cheney and W.A.Light
% Nonlinear approximation, R. DeVore  (Acta Numerica)



\documentclass[a4paper, 11pt, reqno]{article}
\usepackage{graphicx, epsfig, psfrag}
\usepackage{amsfonts,amsmath,amssymb} % ,wasysym,amsbsy
\usepackage[left=0.5cm,right=5.5cm,top=3cm,bottom=3cm]{geometry}
% \usepackage[left=3cm,right=3cm,top=3cm,bottom=3cm]{geometry}
\usepackage{float}
\usepackage{hyperref}
\usepackage{theorem}
\usepackage{color}
\usepackage{subfigure}
\usepackage{mathrsfs}
\usepackage{bm}
\usepackage{bbm}
\usepackage{cleveref}
\usepackage{ams_proof}
% \usepackage{showkeys}
\usepackage{showlabels}

% \usepackage{minted}


\sloppy
\frenchspacing

\numberwithin{equation}{section}

\newcommand{\examnote}[1]{{\color{red} \it [\![ #1 ]\!]}}
\newcommand{\examrem}[1]{ \examnote{ removed non-examinable section }
}
% \newcommand{\examnote}[1]{}
\def\notexam{{\color{red} \bf (NE) }}



\newcommand{\alert}[1]{{\color{red} \it $\{$ #1 $\}$}}

\def\nbintro{{\tt [Notebook 01]}}
\def\nbtrig{{\tt [Notebook 02]}}
\def\nbpoly{{\tt [Notebook 03]}}

%%%%%%%%%%%%%%%%%%%%%%%%%%%%%%
% THEOREM-STYLE-ENVIRONMENTS %
%%%%%%%%%%%%%%%%%%%%%%%%%%%%%%
\newcounter{theorem}
\numberwithin{theorem}{section}
\def\openthm#1#2{\refstepcounter{theorem}\bigskip
{\noindent\bf#1~\thetheorem\if#2!{. }\else{ (#2).}\fi}
\it}
\def\thmskip{}
\newenvironment{theorem}[1][!]{\openthm{Theorem}{#1}}{\thmskip}
\newenvironment{lemma}[1][!]{\openthm{Lemma}{#1}}{\thmskip}
\newenvironment{corollary}[1][!]{\openthm{Corollary}{#1}}{\thmskip}
\newenvironment{proposition}[1][!]{\openthm{Proposition}{#1}}{\thmskip}
\newenvironment{definition}[1][!]{\openthm{Definition}{#1}}{\thmskip}
\newenvironment{axiom}[1][!]{\openthm{Axiom}{#1}}{\thmskip}


%%%%%%%%%%%%%%%%%%%%%%%%%%%%%
% REMARK-STYLE-ENVIRONMENTS %
%%%%%%%%%%%%%%%%%%%%%%%%%%%%%
\def\openrem#1#2{\refstepcounter{theorem}\bigskip
{\noindent\it\bfseries#1~\thetheorem\if#2!{. }\else{ (#2). }\fi} }
\newenvironment{remark}[1][!]{\pushQED{\qed}
  \openrem{Remark}{#1}}{\popQED \medskip}
\newenvironment{example}[1][!]{\pushQED{\qed}
  \openrem{Example}{#1}}{\popQED \medskip}


\newcounter{exercise} \numberwithin{exercise}{section}
\def\openex#1#2{\refstepcounter{exercise}\medskip \noindent
  {\it\bfseries#1~\theexercise\if#2!{. }\else{ (#2). }\fi} }
\newenvironment{exercise}[1][!]{\pushQED{\qed}\openex{Exercise}{#1}}{\popQED
\medskip}


%%%%%%%%%%%%%%%%%%%%%%%%%
%    LISTS              %
%%%%%%%%%%%%%%%%%%%%%%%%%
\def\alist{\renewcommand{\labelenumi}{(\alph{enumi})}}
\def\ilist{\renewcommand{\labelenumi}{(\roman{enumi})}}
\def\deflist{\renewcommand{\labelenumi}{\arabic{enumi}.}}

\newenvironment{longilist} %
{\begin{list}{\hspace{1.4cm}{\it (\roman{enumi})}}{ %
      \usecounter{enumi} %
      \setlength{\leftmargin}{0mm} %
      \setlength{\rightmargin}{0mm} %
      \setlength{\listparindent}{\parindent}}
  } %
  {\end{list}} %



\input notation.tex



\begin{document}

\begin{center}
  \hfill \\[4cm]
  {\Large \bf MA3J8} \\[1mm]
  {\Large \bf Approximation Theory and Applications} \\[4mm]
  {\large Christoph Ortner} \\
  {\small \tt c.ortner@warwick.ac.uk} \\
  {\small Zeeman Building, University of Warwick, Coventry CV4 7AL, UK} \\[4mm]
  {\today}
\end{center}

\clearpage
\quad
\clearpage

\tableofcontents

\clearpage
% !TeX root = ./apxthy.tex

\section{Introduction}
%
\label{sec:intro}
%
\begin{quote}
In mathematics, approximation theory is concerned with how functions can best be
approximated with simpler functions, and with quantitatively characterizing the
errors introduced thereby. Note that what is meant by best and simpler will
depend on the application. (Wikipedia)
\end{quote}

Approximation theory underpins much of numerical computation and arises also  in
several other branches of mathematics. It is one of the most mature disciplines
of computational mathematics, to the extent that is often treated as a
sub-discipline of pure mathematics. This module takes a more computational
perspective. While it still focuses primarily on mathematics and theory, the
choice of material is with an eye to applications in numerical simulation and
data science rather than purely for its own sake. The theory will be
supplemented with numerical examples and it will allow us to explain what we
observe numerically. We will often sacrifice optimality of the results for
simplicity and to obtain good intuitions.

The first question is to address what we mean by ``simple functions''. Briefly,
we mean functions that are efficient and accurate (numerical stability!) to
evaluate in (typically) floating point arithmetic on a modern processor. This
simple  observation already shows that approximation theory cannot be detached
from numerical analysis and computer simulation. In Part I we will focus on
trigonometric polynomials ($\cos nx$, $\sin nx$), algebraic polynomials ($x^n$)
and splines (piecewise algebraic polynomials). In some PDEs but in particular in
data science the approximation problems are often high-dimensional; we will
explore some examples in Part II of this module.


Motivation / Applications:
\begin{itemize}
  \item Solving differential and integral equations
  \item machine learning, data-driven modelling, data assembly: The recent
  explosion in machine learning has given the field a new boost; indeed, many
  machine learning problems can be interpreted as approximation problems.
\end{itemize}


Themes:
\begin{enumerate}
\item Approximation spaces: what are ``good'' functions that we can combine
    to approximate general functions well.
  \begin{itemize}
    \item Global approximation: trigonometric and algebraic  polynomials
    \item Piecewise approximation: splines
    \item Ridge functions
    \item Radial basis functions
    \item sparse grids
  \end{itemize}

\item Algorithms, constructive approximation:
  \begin{itemize}
    \item best approximation, projection
    \item interpolation
    \item kernel methods
    \item least squares
    \item adaptive grids
  \end{itemize}

\item Miscellaneous
  \begin{itemize}
    \item Regularity
    \item Numerical stability
    \item Curse of dimensionality
  \end{itemize}
\end{enumerate}


\subsection{Literature \& Acknowledgements}
%
\label{sec:acknowledgements}
%
Section~\ref{sec:trig} is largely based on random online available lecture notes
but partly motivated by \cite{Trefethen2000-fr,Trefethen2013-rg}.

Section~\ref{sec:poly} largely follows \cite{Trefethen2013-rg}, adding only the
Chebyshev transform and Jackson's theorem which are natural consequences of the
material on trigonometric approximation. The book \cite{Trefethen2013-rg} is
available for free online at 
\begin{quote}
  {\tt http://www.chebfun.org/ATAP/}
\end{quote}

The section on splines is fairly standard material, but is based to some extend
on the classical text \cite{Powell1981-bg}.

Exercises are partly based on gaps in the lecture material, partly adapted from
these references.

All of these texts are good references for further reading.

\section{Preliminaries}
%
\label{sec:prelims}
%

\subsection{Abstract Approximation Problems}
%
We are concerned with approximating specific functions given to us, or classes
of functions with specific properties, such as some given regularity,
periodicity, symmetries, etc. To study generic approximation schemes it is
therefore useful to begin by specifying a class $Y \subset X$ of functions of
interest. Typically $X$ will be an infinite-dimensional linear space,  and $Y$
an infinite-dimensional non-trivial subset of $X$. $X$ will be endowed with
a notion of distance $d$. We will later always assume this is given by a norm,
but this is not important for now.

In linear approximation (which is what most of this module is about)
we are given a set $B_N \subset X$, consisting of $N$ linearly independent
{\em basis functions}. Given some $f \in Y$ we then wish to find
an approximation to $f$ from ${\rm span} B_N =: Y_N$.

Fundamental questions/problem arising in this are, e.g.,
\begin{itemize}
\item Convergence: $\inf_{p \in Y_N} d(p, f) \to 0$ as $N \to \infty$
\item Best approximation: Find $p_N \in Y_N$ such that $d(p_N, f)$ is minimal.
\item Approximation to within some tolerance: given $\tau > 0$ find $N$ (minimal?)
  and $p_N \in Y_N$ such that $d(p_N, f) < \tau$.
\item Rates of approximation: $\inf_{p \in Y_N} d(p, f) \leq \epsilon_N$
  and characterise the rate, possibly uniformly for all $f \in Y$
\item Construction of approximations: Given $f$ give an algorithm to
  construct an approximation $p_N$ e.g., the best approximant.
\item Evaluation: efficient and numerically stable construction and
  evaluation of $p_N$.
\end{itemize}

In the exercises of Section~\ref{sec:prelims} we will collect a few basic
examples and generic facts.

\subsection{Basics}
%
In this section we briefly review some fact from analysis and linear algebra,
and most importantly, complex analysis.

\subsubsection{$\R^N$}
%
The majority of the analysis in this module is for general
$N$-dimensional systems of ODEs. We will use the structure of $\R^N$
as a vector space, supplied with the Euclidean norm and inner product
\begin{displaymath}
  x \cdot y := x^T y = \sum_{i = 1}^N x_i y_i, \quad \text{and} \quad
  |x| := \sqrt{x \cdot x}
\end{displaymath}
Key inequalities that we will use on a regular basis are the {\em
  triangle inequality}
\begin{equation}
  \label{eq:triangle_ineq}
  |x + y| \leq |x| + |y| \qquad \text{ for } x, y \in \R^N,
\end{equation}
the {\em Cauchy--Schwarz inequality}
\begin{equation}
  \label{eq:cauchyschwarz_ineq}
  |x \cdot y| \leq |x| |y| \qquad \text{ for } x, y \in \R^N,
\end{equation}
and  {\em Cauchy's inequalities},
\begin{align}
  \label{eq:cauchy_ineq}
  a b &\leq \smfrac12 a^2 + \smfrac12 b^2 \qquad \text{ for } a, b \in
  \R, \\
  \label{eq:cauchy_eps_ineq}
  a b &\leq \smfrac\eps{2} a^2 + \smfrac1{2\eps} b^2 \qquad \text{ for
  } a, b \in \R, \eps > 0.
\end{align}


\subsubsection{Smooth functions}
%
Recall from the introductory analysis modules the definitions of
continuous functions and of uniform convergence. Here, we define the
spaces, for an interval $D \subset \R$,
\begin{align*}
  C(D) &:= \b\{ f : D \to \R \bsep f \text{ is continuous on } D \b\}
\end{align*}
If $D$ is compact ($D = [a, b]$ for $a, b \in \R$), then $C(D)$
is {\em complete} when equipped with the sup-norm
\begin{align*}
  \| f \|_{\infty} := \|f\|_{L^\infty} := \|f\|_{L^\infty(D)} := \sup_{x \in D} |f(x)|.
\end{align*}
We will more typically write $\|f\|_\infty$ if it is clear over which set the
supremum is taken. Note also that $D$ need not be compact in the
definition of $\|\cdot\|_{\infty, D}$.

Moreover, we define the spaces of $j$ times continuously differentiable functions
\[
  C^j(D) := \b\{ f : D \to \R \bsep f \text{ is $j$ times continuously
                differentiable on } D \b\},
\]
and the associated norms
\[
   \|f \|_{C^j} :=  \|f \|_{C^j(D)} :=
    \max_{n = 0, \dots j} \| f^{(n)} \|_{\infty, D},
\]
where $f^{(n)}$ denotes the $n$th derivative.

We also define $C^\infty(D) := \bigcup_{j > 0} C^j(D)$.

We say $f : D \to \R$ is H\"{o}lder continuous if there exists $\sigma \in (0, 1]$
such that
\[
    |f(x) - f(x')| \leq C |x - x'|^\sigma \qquad \forall x, x' \in D.
\]
The associated space is denoted by $C^{0,\sigma}$. If $\sigma = 1$ then
we call $f$ {\em Lipschitz continuous}. Further, we define the space
$C^{j,\sigma}(D) := \{ u \in C^j(D) \sep u^{(j)} \in C^{0,\sigma}(D)\}$.

The right-hand side in the definition of H\"{o}lder  continuity is 
a special case of a {\em modulus of continuity}. We say that $f \in C([a,b])$ has a 
{\em modulus of continuity} $\omega : [0, \infty) \to \R$ if 
$\omega$ is monotonically increasing, $\omega(r) \to 0$ as $r \to 0$ and 
\[
  |f(x) - f(x')| \leq \omega(|x-x'|) \qquad \forall x, x' \in [a,b].
\]


\subsubsection{Integrable functions}
%
Sometimes it will be convenient to consider measurable functions, and
for the sake of precision we briefly review the relevant definitions.
For $D = (a, b)$ an interval and $f : D \to \R$ measurable (i.e.,
$f^{-1}(B)$ is a Lebesgue set whenever $B$ is a Lebesgue set), we define
\[
    \| f \|_{L^p} := \|f \|_{L^p(D)} :=
      \left(\int_D |f|^p \,dx\right)^{1/p}, \qquad 1 \leq p < \infty,
\]
and
\[
    \|f\|_{L^\infty} := \|f\|_{L^\infty(D)} :=
    {\rm ess.}\sup_{x \in D} |f(x)|.
\]
Finally, we define the spaces
\[
  L^p(D) := \big\{ f : D \to \R \bsep \text{$f$ is measureable
                  and $\|f\|_{L^p(D)} < \infty$} \b\}.
\]

% \alert{Introduce $\AC$??? Not if we can avoid it.}

\subsubsection{Normed Spaces and Hilbert spaces}
%
A tuple $(X, \|\cdot\|)$ is called a normed space or normed vector space if
it is a linear space over the field $\mathbb{F}$ and
$\|\cdot\| : X \to \R$ defines a norm, i.e., for all $f, g \in X, \lambda \in \mathbb{F}$
\begin{itemize}
  \item $\|f + \lambda g\| \leq \|f\| + |\lambda| \|g\|$
  \item $\|f\| \geq 0$ and $\|f\| = 0$ iff $f = 0$.
\end{itemize}
$X$ is called a Banach space if it is complete (i.e. all Cauchy sequences
in $X$ have a limit in $X$).

If $D$ is compact then the spaces $(C^j, \|\cdot\|_{C^j})$ and
$(L^p, \|\cdot\|_{L^p})$ are Banach spaces. $C^{j,\sigma}$ may also be made
into Banach spaces, though we won't need this.

A tuple $(X, \<\cdot, \cdot\>)$ is called a Hilbert space over $\mathbb{F} \in
\{\R, \C\}$ if the following conditions are satisfied:
\begin{itemize}
  \item $X$ is a linear vector space
  \item $\<\cdot, \cdot\> : X \times X \to \mathbb{F}$ is an inner product, i.e.,
  for all $f, g, h \in X, \lambda \in \mathbb{F}$ we have
  \begin{itemize}
      \item $\<f, g\> = \overline{\< g, f\>}$
      \item $\< f + \lambda g, h \> = \<f, h \> + \lambda \< g, h \>$
      \item $\< f, f \> \geq 0$
      \item $\< f,f \> = 0$ iff. $f = 0$.
  \end{itemize}
  \item $X$ is complete under the norm $\|f\| := \<f,f\>^{1/2}$.
\end{itemize}

The most common example we will encounter are $L^2$-type spaces. In particular,
if $D$ is an interval (or in fact any measurable set), then $L^2(D)$ equipped
with the inner product
\[
  \< f, g \>_{L^2} := \int_D f \overline{g} \,dx
\]
is a Hilbert space.

% avoid $H^k$, $H^s$??? Or make this an exercise???


\subsection{Analytic functions}
%
A proper study of analytic functions requires far more time than we have
available. But some basics will suffice for the most important ideas.
To save time (and unfortunately skip some beautiful structures of
complex numbers) we will work exclusively with the definitions via power
series.

Recall therefore that each power series
\[
    \sum_{n = 0}^\infty c_n (z - z_0)^n
\]
has a radius of convergence
\[
    r = \frac{1}{\limsup_{n \to \infty} \sqrt[n]{|c_n|}}
\]

\begin{definition}
  Let $D \subset \C$ be open and $f : D \to \C$. We say that $f$ is
  analytic at a point $z_0 \in D$ if there exists a power series
  $\sum_{n = 0}^\infty c_n (z - z_0)^n$ with  positive radius
  of convergence $r > 0$ such that
  \[
    f(z) = \sum_{n = 0}^\infty c_n (z - z_0)^n \qquad \forall
    z \in D, |z - z_0| < r.
  \]
  We say $f$ is analytic in $D$ if it is analytic in each point
  $z_0 \in D$.
\end{definition}

We will need two simple concepts around analytic functions: (1) continuations;
and (2) path integrals. We will formulate simplified versions that are
sufficient for our purposes and only give rough ideas of the proofs
in the lectures (these are not contained in these lecture notes).

\begin{proposition}[Analytic Continuation]
  \begin{enumerate} \ilist
    %
    \item Let $D \subset \C$ be open, $f : D \to \C$ and let $D' \subset D$ be the
    set of points in which $f$ is analytic. Then $D'$ is open.
    %
    \item Let $D' \subset D \subset \C$, with $D$ open and connected and $D'$
    contains a line segment $\{(1-t) z_0 + t z_1 | t \in [0,1] \}$ with $z_0
    \neq z_1$. Let $f : D' \to \C$ be analytic and let $f_1, f_2 : D \to \C$ be
    two analytic continuations of $f$ to $D$ i.e., $f_j$ are analytic on $D$ and
    $f_j = f$ on $D'$. Then, $f_1 = f_2$.

    (Note: this result can be significantly generalised.)
    %
    \item Let $f \in A([a, b])$ then there exists $D \supset [a,b]$ open in $\C$
    such that $f$ can be uniquely extended to a function $f \in A(D)$.
  \end{enumerate}
\end{proposition}

\def\calC{\mathcal{C}}

Concerning path integrals, let $\calC$ be a continuous and piecewise smooth
oriented  curve in $\C$, i.e., we identify $\calC$ with a parametrisation $(\zeta(t))_{t \in [0, 1]}$
\[
    \int_{\calC} f(z) \, dz := \int_{t = 0}^1 f(\zeta(t)) \zeta'(t) \, dt.
\]
Note that this definition makes sense even if $\zeta$ is not $C^1$, but
only piecewise $C^1$ (with finitely many pieces!).

If $\calC$ is a Jordan curve (simple and closed), then we assume that the orientation is counter-clockwise and we will write
\[
    \oint_{\calC} f(z) \,dz := \int_{\calC} f(z) \, dz
    = \int_{t = 0}^1 f(\zeta(t)) \zeta'(t) \, dt
\]
for this {\em contour integral}.

\begin{proposition}[Cauchy's Integral Theorem]
  Let $D \subset \C$ be simply connected, $f$ analytic in $D$ and
    $\calC \subset D$ a Jordan curve, then
    \[
        \oint_{\calC} f(z) \, dz = 0.
    \]
    %
\end{proposition}



\subsection{Exercises}

\begin{exercise}[Best Approximations]
  \label{exr:prelims:bestapprox}
  \begin{enumerate} \ilist
  \item Let $X$ be a vector space endowed with a norm $\|\cdot\|$,
  $X_N \subset X$ with ${\rm dim} X_N = N < \infty$ and let
  $Y_N \subset X_N$ be closed. (E.g. $Y_N=X_N$ is admissible.)
  Prove that for all $f \in X$ there exists a best approximation
  $p_N \in Y_N$, i.e.,
  \[
    \| p_N - f \|  = \inf_{y_N \in Y_N} \|y_N - f\|.
  \]

  \item Suppose $\|\cdot\|$ is strictly convex, i.e., for $f_0, f_1 \in X, \lambda \in (0, 1)$,
  \[
    \| (1-\lambda) f_0 + \lambda f_1 \| \leq (1-\lambda) \|f_0\| + \lambda \|f_1 \|
  \]
  with equality if and only if $f_0 \propto f_1$. Suppose also that $Y_N$ is
  convex. Under these two conditions prove that the best approximation from (i)
  is unique.

  \item Suppose that the {\em best approximation operator}
    $\Pi_N : f \mapsto p_N$ where $p_N$ is the unique best approximation to $f$
    is well-defined (e.g. in the setting of (ii)). Prove that $\Pi_N : X \to
    Y_N$ is continuous.
  \end{enumerate}
\end{exercise}

\begin{exercise}[Best Approx. in max-norms]
  \label{exr:prelims:bestapprox_maxnorms}
  \begin{enumerate}\ilist
  \item Consider $X = \R^2$ equipped with the $\ell^\infty$-norm. Show that
  this norm is {\em not} strictly convex. Consider the best approximation
  from $Y_N := \{ x \in \R^2 | |x|_\infty \leq 1 \}$. Show that
  \begin{itemize}
    \item $f = (2, 0)$, then the best-approximation is non-unique.
    \item $f = (2,2)$, then the best-approximation is unique.
  \end{itemize}

  \item Now consider $X = C([-1,1])$ and
  \[
    X_0 = Y_0 = \{ x \mapsto a | a \in \R \}
  \]
  i.e., approximation by constant functions. Prove that $\|\cdot\|_C =
  \|\cdot\|_{L^\infty}$ is {\em not} strictly convex, but nevertheless the best
  approximation problem for $Y_0, Y_1$ has a unique solution.

  {\it Hint: An easy way to see this is that $X_0$ is one-dimensional hence,
  any norm is strictly convex on $X_0$. But an alternative way to prove this is
  to simply construct the best approximation operator explicitly, which also
  helps with (iii).}

  \item {Bonus: } Now carry out (ii) for
  \[
      X_1 = Y_1 = \{ x \mapsto a + bx | a,b \in \R \},
  \]
  i.e., best approximation by affine functions.

  {\it Hint: One can still ``guess'' from geometric intuition an explicit characterisation of the
  best approximation operator by first choosing $b$ and then $a$. Then prove
  that $\|f - (a+bx)\|_\infty$ is attained at three points $x_1 < x_2 < x_3$.
  This can be used to prove uniqueness of the best approximation.

  This proof is not entirely trivial (at least I don't see a simple way to
  prove it) and we will revisit this in \S~\ref{sec:poly}.
  } \qedhere
  \end{enumerate}
\end{exercise}

\begin{exercise}[Best Approximation in a Hilbert Space]
  \label{exr:prelims:bestapprox_hilbert}
  Let $X$ be a Hilbert space with inner product $\<\cdot,\cdot\>$ and
  $Y_N = X_N \subset X$ an $N$-dimensional subspace.
  \begin{enumerate} \ilist
  \item Show that the best approximation $p_N$ of $f \in X$ in $X_N$ is characterised
    by the variational equation
    \[
         \< p_N, u \> = \< f, u \> \qquad \forall u \in X_N.
    \]
    Show that this has a unique solution.

    \item Let $\Pi_N f = p_N$ denote the best approximation operator. Show
    that it is an orthogonal projection.

    \item Deduce that
    \[
        \| f - \Pi_N f \|^2 = \|f\|^2 - \| \Pi_N f \|^2.
    \]

    \item {\bf Linear Approximation: }  Let $\{ e_j \}_{j \in \N}$ be an
    orthonormal basis of $X$, i.e., $\<e_j, e_n \> = \delta_{jn}$ and ${\rm
    clos}{\rm span}\{e_j\}_j = X$. Let $X_N := {\rm space}\{e_1, \dots, e_N\}$,
    \[
        \Pi_N f = \sum_{j = 1}^N \< f, e_j \> e_j. \qedhere
    \]
  \end{enumerate}
\end{exercise}


\begin{exercise} \label{exr:prelims:inequalities}
  \begin{enumerate} \ilist
  \item Prove \eqref{eq:cauchy_ineq} and
  \eqref{eq:cauchy_eps_ineq}.
  \item Use \eqref{eq:cauchy_ineq} to prove
  \eqref{eq:cauchyschwarz_ineq}.
  \item Use \eqref{eq:cauchyschwarz_ineq} to prove \eqref{eq:triangle_ineq}.
  \qedhere
  \end{enumerate}
\end{exercise}


\begin{exercise}   \label{exr:prelims:functions}
  For the following functions $f$, specify to which of the following
  spaces they belong: $C^{j,\sigma}([-1,1])$ (specify $j$ and $\sigma$),
  $C^\infty([-1,1])$, $A([-1,1])$, $L^p(-1,1)$. No rigorous proofs are
  required.
  %
  \begin{enumerate} \ilist
    \item $f(x) = x^n$, $n \in \N$
    \item $f(x) = |x|$
    \item $f(x) = |x|^{3}$
    \item $f(x) = |x|^{3/2}$
    \item $f(x) = (1+x^2)^{-1}$
    \item $f(x) = \exp( - 1 / (1/2-x) ) \chi_{[-1,1/2)}(x)$
    \item $f(x) = e^{-x^2}$
    \item $f(x) = \cos(1.23 x)$ \qedhere
  \end{enumerate}
\end{exercise}

\begin{exercise} \label{exr:prelims:extensions}
  Construct the analytic extensions of the following functions
  to a maximal set $D$ in  the complex plane, which you should specify:
  \begin{enumerate} \ilist
    \item $f(x) = e^{-x^2}$ on $\R$
    \item $f(x) = (1+x^2)^{-1}$ on $\R$
    \item $f(x) = \sum_{j = 0}^\infty x^j$ for $x \in (-1,1)$
    \item $f(x) = \int_0^\infty e^{-t (1-x)} \,dt$ for $x < 1$
  \end{enumerate}
\end{exercise}




%%%%%%%%%%%%%%%%%%%%%%%%%%%%%%%%%%%%%%%%%%%%%%%%%%%%%%%%%%%%%%%%%%%%
%%%%%%%%%%%%%%%%% PART I %%%%%%%%%%%%%%%%%%%%%%%%%%%%%%%%%%%%%%%%%%%
%%%%%%%%%%%%%%%%%%%%%%%%%%%%%%%%%%%%%%%%%%%%%%%%%%%%%%%%%%%%%%%%%%%%
\clearpage

\vspace{5cm}

{\huge Part I: Univariate Approximation}

\clearpage
% !TEX root = apxthy.tex


\section{Trigonometric Polynomials}
%
\label{sec:trig}
%
In this chapter we consider approximation of periodic functions by trigonometric
polynomials (aka Fourier spectral methods). Throughout this chapter, let $\TT :=
(-\pi, \pi]$ and we identify $C^j(\TT) = C^j_{\rm per}(\TT)$, $A_{\rm per}(\TT) =
A(\TT)$, $L^p(\TT)$ to be the spaces of $2\pi$-periodic functions on $\R$ that
are, respectively, $j$ times  continuously differentiable, analytic, belong
to $L^p(-\pi, \pi)$. Similarly, $H^j_{\rm per}(\TT) = H^j(\TT)$ denotes the
space of $2\pi$ periodic functions on $\R$ such that their restriction to {\em
any} interval $(a, a+2\pi)$ belongs to $H^j(a,a+2\pi)$.

Examples of periodic functions:
\begin{itemize}
  \item $\sin(nx) \in A(\TT)$
  \item $|\sin(nx)| \in C^{0,1}(\TT)$
  \item $|\sin(nx)|^3 \in C^{2,1}(\TT)$
  \item $e^{-\cos x} \in A(\TT)$
  \item $(c^2+\sin^2 x)^{-1} \in A(\TT)$
  \item \dots
\end{itemize}

Applications:
\begin{itemize}
  \item BVPs with periodic boundary conditions and periodic data, e.g.,
  \begin{align*}
      - (p(x) u_{x})_x + q(x) u &= f(x), \qquad x \in (-\pi, \pi), \\
      u(-\pi) &= u(\pi), \\
      u'(-\pi) &= u'(\pi),
  \end{align*}
  where $p, q, f$ are $2\pi$-periodic, then under suitable conditions on
  $p, q, f$ there exists a unique solution which is also $2\pi$-periodic.
  %
  \item Functions represented in polar coordinates: $u(x, y) = v(r, \theta)$
  then, for $r$ fixed, $\theta \mapsto v(r, \theta)$ is periodic.

  There are many other examples of naturally ``periodic'' coordinate systems,
  including e.g. spherical coordinates, or the dihedral angle.
  %
  % \item Bond-angles and dihedral
\end{itemize}

Approximation by trigonometric polynomials is based on the idea of Fourier
series representation of periodic functions. Talking about Fourier series
becomes much more convenient if we extend the admissible range of all functions
to $\C$. The following definition then becomes natural.

\begin{definition}
  A trigonometric polynomial of degree $N$ is any function of the form
  \[
    \sum_{k = -N}^N a_k e^{i k x}
  \]
  The space of all such polynomials is denoted by  $\Trig_N$.
  The canonical basis is
  \[
     \b\{ e^{ikx} \bsep  k = -N, -N+1, \dots, N \b\}
  \]
\end{definition}

\begin{definition}
  Let $f \in L^1(\TT)$, then its {\em Fourier coefficients} are given by,
  \begin{equation} \label{eq:trig:fourier coeffs}
    \hat{f}_k := \mint_{-\pi}^\pi f(x) e^{- i kx} \,dx
  \end{equation}
  The $N$-th partial sum, is a trigonometric polynomial, which we
  denote by
  \[
    \Pi_N f(x) := \sum_{n = -N}^N \hat{f}_k e^{i kx}.
  \]
\end{definition}



\subsection{Approximation by $L^2$-projection}
%
\label{sec:trig:L2}
%
We will initially study approximation of functions in the $L^2$-norm.
It can then be convenient to normalise the inner product, via
\[
  \< f, g \>_{L^2(\TT)} := \mint_{-\pi}^\pi f^* g \, dx.
\]
Equipped with this inner product, $L^2(\TT)$ is a Hilbert space.


\begin{theorem} \label{th:trig:plancherel}
  \begin{enumerate} \ilist
  \item Convergence of Fourier Series: $\{ e^{ikx} \sep k \in \Z \}$ is an orthonormal basis for $L^2(\TT)$.
  \item Plancherel Theorem: $\mathcal{F} : L^2(\TT; \C) \to \ell^2(\Z; \C)$ is an isomorphism;
    i.e., $f \in L^2(\TT)$ then $\hat{f} \in \ell^2(\Z)$ and
    \[
       \sum_{k \in \Z} \hat{f}_k \hat{g}_k^* = \mint_\TT f g^* \,dx.
    \]
    In particular, $\|f\|_{L^2} = \|\hat{f} \|_{\ell^2}$.
  \end{enumerate}
\end{theorem}
\begin{proof}
  This is left as an exercise, to be completed after we
  study kernel methods. The key point is that
  \begin{equation}
    \mint_{-\pi}^\pi  e^{-ikx} e^{i\ell x} \,dx
    = \mint_{-\pi}^\pi e^{(\ell-i)x}
    = \cases{
      1, & \ell = k, \\
      0, & \text{otherwise.}
    }
  \end{equation}
\end{proof}

\begin{remark}
  There is a general theorem that all (separable) Hilbert spaces are
  isometrically isomorphic to $\ell^2(\N)$ or equivalently to $\ell^2(\Z)$.
  Explain why the Plancherel theorem simply shows that the Fourier series map
  $f \mapsto \hat{f}$ is an the explicit construction of this isometry.
\end{remark}


\begin{proposition} \label{th:trig:PiNf-orthproj}
  Let $f \in L^2(\TT)$, then
  \begin{equation} \label{eq:trip:PiNf-orthproj}
    \| \Pi_N f - f \|_{L^2}^2 = \sum_{|k| > N} |\hat{f}_k|^2.
  \end{equation}
  In particular, $\Pi_N f$ is the $L^2$-orthogonal
  projection of $f$ onto $\Trig_N$, or equivalently, the
  best approximation of $f$ from $\Trig_N$ w.r.t. $\|\cdot\|_{L^2}$.
\end{proposition}
\begin{proof}
  By definition,
  \begin{align*}
    f(x) - \Pi_N f(x) = \sum_{|k|>N} \hat{f}_k e^{ikx},
  \end{align*}
  and Plancherel's theorem then implies \eqref{eq:trip:PiNf-orthproj}.

  The fact that $\Pi_N f$ is the best approximation is a straightforward
  consequence: if $g \in \Trig_N$, then
  \begin{align*}
    \b\|f(x) - \Pi_N f(x) - g\b\|_{L^2}^2
    &= \sum_{|k| \leq N} |\hat{g}_k|^2 + \sum_{|k| > N} |\hat{f}_k|^2 \\
    &\geq \sum_{|k| > N} |\hat{f}_k|^2 \\
    &= \b\|f(x) - \Pi_N f(x) \b\|_{L^2}^2. \qedhere
  \end{align*}
\end{proof}

The main point of Lemma~\ref{th:trig:PiNf-orthproj} is that, as in the
introductory example, we can characterise the error in terms of the
decay of the Fourier coefficients.



\subsection{Decay of Fourier Coefficients}
%
\label{sec:trig:decay}
%
As we already saw in the introductory example, the ``smoother'' $f$ is, the
faster $\hat{f}_k$ decay. The following results are not difficult to generalise
in several ways; see remarks below, but in the spirit of valuing simplicity over
optimality, we will formulate them only for $C^p$ regularity.

\begin{theorem} \label{th:trig:decay}
  \begin{enumerate} \ilist
    \item Let $f \in C^{p}(\TT)$, then there exists $C > 0$ such that
    \[
        |\hat{f}_k| \lesssim C |k|^{-p}.
    \]
    %
    \item {\it Paley--Wiener Theorem:} If $f \in A(\TT)$, then there exists
    $a > 0$ such that
    \[
        |\hat{f}_k| \lesssim e^{-a N}.
    \]
  \end{enumerate}
\end{theorem}
\begin{proof}[Proof of Theorem~\ref{th:trig:decay}(1)]
  Consider the case $p = 1$. Assume for the moment that we can exchange
  summation and differentiation, then we simply have
  \[
    f'(x) = \sum_{k \in \Z} ik \hat{f}_k e^{ikx}.
  \]
  Since $f' \in C(\TT) \subset L^1(\TT)$, $\hat{(f')}_k = ik \hat{f}_k$ are
  bounded and in particular, $|\hat{f}_k| \lesssim |k|^{-1}$. However, this
  calculation requires that we justify the interchange of differentiation
  and summation.

  Instead, let $h > 0$ and consider the function $d_hf(x) :=
  (f(x+h)-f(x)) / h$, then $d_h f(x) = f'(\xi)$ for some $\xi \in (x, x+h)$,
  hence $\|d_h f(x)\|_{\infty}$ is bounded independently of $h$. In particular
  the Fourier coefficients $\widehat{(d_h f)}_k$ are well-defined and bounded.
  On the other hand, we can compute $\widehat{(d_h f)}_k$ explicitly,
  \begin{align*}
    d_hf(x)
      &= \frac{f(x+h)-f(x)}{h} \\
      &= \sum_{k \in \Z} \hat{f}_k \bigg( \frac{e^{i k (x+h)} - e^{ikx}}{h} \bigg) \\
      &= \sum_{k \in \Z} \Big[\smfrac{e^{ikh} - 1}{h} \hat{f}_k\Big] e^{ikx},
  \end{align*}
  that is,
  \[
    \widehat{(d_h f)}_k = \Big[\smfrac{e^{ikh} - 1}{h} \hat{f}_k\Big].
  \]
  We know that $\widehat{(d_h f)}_k$ are uniformly bounded, hence we obtain
  \[
    C \geq \b|\widehat{(d_h f)}_k\b|
      = \B| \smfrac{e^{ikh} - 1}{h} \hat{f}_k \B| \qquad \forall h > 0
  \]
  Let $h \to 0$ to obtain $|k \hat{f}_k| \leq C$. This completes the proof.
\end{proof}

We postpone the proof of the Paley--Wiener Theorem to
Theorem~\ref{th:trig:pw-trefversion}, but instead first discuss the consequences
of these results.

A direct naive calculation shows that, if $f \in C^p(\TT)$, then
\[
    \|f - \Pi_N f \|_{L^2} \lesssim N^{1/2-p}.
\]
But we want to improve this a bit, by removing the $1/2$ factor. We can
do this with the following slightly sharper result.

\begin{lemma} \label{th:trig:fCp-coeffL2}
  Let $f \in C^p(\TT)$, then $(\hat{f}_k |k|^p)_{k \in \Z} \in \ell^2(\Z)$
\end{lemma}
\begin{proof}
  This is a relatively straightforward extension of the Proof of
  Theorem~\ref{th:trig:decay}(1) and is left as an exercise.
\end{proof}



\begin{theorem} \label{th:trig:convergence_L2}
  \begin{enumerate} \ilist
  \item Let $f \in C^{p}(\TT)$ then
  \[
      \|f - \Pi_N f \|_{L^2} \lesssim N^{-p}
  \]
  \item Let $f \in C^\infty(\TT)$, then for each $p > 0$ there exists a
  constant $C_p$ such that
  \[
      \|f - \Pi_N f \|_{L^2} \leq C_p N^{-p}.
  \]
  \item If $f \in A(\TT)$, then there exists $a > 0$ such that
  \[
       \| f - \Pi_N f \|_{L^2} \lesssim e^{-a N}.
  \]
  \end{enumerate}
\end{theorem}
\begin{proof}
  We only prove (1); the results (2, 3) are left as an exercise.

  From Lemma~\ref{th:trig:PiNf-orthproj} we have
  \begin{align*}
    \|f-\Pi_N f \|_{L^2}^2
    &= \sum_{|k| > N} |\hat{f}_k|^2 \\
    &= \sum_{|k| > N} |\hat{f}_k|^2 |k|^{2p} |k|^{-2p} \\
    &\lesssim N^{-2p},
  \end{align*}
  where we used Lemma~\ref{th:trig:fCp-coeffL2} in the last step.
\end{proof}

See explore convergence rates through numerical tests in \nbtrig, where we
see that our theory is not quite sharp.

\subsubsection{Remarks}
\begin{enumerate}
  \item The algebraic convergence rates are not really sharp. In particular, the
    precise structure of $f^{(p)}$ is extremely relevant. For example, one can
    show that, if $f^{(p-1)}$ is absolutely continuous (or even just of bounded variatin)
    then the decay rate $|\hat{f}_k| \leq \|f^{(p)}\|_{L^1} N^{-p}$ still holds.
    In particular, if $f^{(p)} \in C(\TT)$ as we have assumed here, this gives
    additional structure that we have not exploited.

  \item The main message is still relevant: (1) $f \in C^p(\TT)$ regularity
  gives algebraic decay of $\hat{f}_k$; (2) $f \in C^\infty(\TT)$ gives
  super-algebraic decay; (3) $f \in A(\TT)$ gives exponential decay.

  \item We can also derive uniform approximation error estimates which further
  highlight that our results are not sharp, e.g.,  if $f \in C^p(\TT)$, then
  \[
    |f(x) - \Pi_N f(x)| \leq \sum_{|k| > N} |\hat{f}_k|
        \lesssim \sum_{|k| > N} |k|^{-p}
        \lesssim N^{1-p}.
  \]
  In the next section we will show how to construct much better uniform
  approximations with sharp rates. Using a similar trick as in the proof of
  Theorem~\ref{th:trig:convergence_L2} we can improve this to $\|f-\Pi_N f
  \|_\infty \lesssim N^{1/2-p}$. Getting a little deeper into harmonic analysis
  we may even prove that $|\hat f_k| |k|^p \in \ell^p$ for all $p > 1$, which
  indeed implies that $\|f - \Pi_N f \|_\infty \lesssim N^{\epsilon - p}$ for
  all $\epsilon > 0$. This give us a hint that the best approximation error in
  the max-norm is in fact $O(N^{-p})$ when $f \in C^p$. We will choose a very
  different route in \S~\ref{sec:trig:jackson} to prove this result.

  \item The uniform convergence estimate for analytic functions arising from
  the Paley--Wiener theorem is however qualitatively sharp,
  \[
       \|f - \Pi_N f \|_\infty \lesssim e^{- a' N} \qquad \forall a' < a.
  \]
\end{enumerate}




\subsection{Approximation by convolution: Jackson's Theorem}
%
\label{sec:trig:jackson}
%
The overarching idea of kernel methods is, instead of using the
$L^2$-projection $\Pi_N f$ to approximate $f$, we use a convolution operator,
\[
    (K_N \ast f)(x) := \int_{-\pi}^\pi K_N(t-x) f(t) \, dt.
\]
If $K_N(t)$ is a trigonometric polynomial, then $K_N \ast f$ will also be a trigonometric polynomial:

\begin{lemma}
  If $K_N \in C(\TT)$ and $K_N \in \Trig_N$, then $K_N \ast f \in \Trig_N$ for
  all $f \in L^1(\TT)$.
\end{lemma}
\begin{proof}
  \begin{align*}
    \mint_{-\pi}^\pi K_N(x-t) f(t) \,dt
    &=
    \sum_{k =  -N}^N \sum_{k' \in \Z}
        \hat{K}_{N,k} \hat{f}_{k'} \mint_{-\pi}^\pi e^{ik(x-t)} e^{ik't}\,dt
    \\ &=
    \sum_{k = -N}^N \hat{K}_{N,k} \hat{f}_{k} e^{ikx}. \qedhere
  \end{align*}
\end{proof}


The ``original'' kernel is the Dirichlet kernel,
\[
    D_N(x) = \frac{\sin\b( (N+1/2) x \b)}{ \sin(x/2) },
\]
which is interesting in that it represents the $L^2$-projection, i.e.,
$\Pi_N f = D_N \ast f$; cf. Exercise~\ref{exr:trig:dirichlet}.

But there is considerable freedom in the choice of kernel. A particularly
``felicitious'' choice is the Jackson kernel,
\[
    J_M(x) := \gamma_M \left( \frac{\sin( Mx/2)}{\sin(x/2)} \right)^4,
    \qquad
    \int_\TT J_M(x) = 1,
\]
where the second condition determines the normalisation constant $\gamma_M$.
Constructing approximates via the Jackson kernel leads to elegant and
sharp approximation error estimates in the max-norm.

\begin{lemma}
  $J_M \in \Trig_{2M-2}$.
\end{lemma}
\begin{proof}
  Let $z = e^{ix/2}$, then
  \[
      J_M(x)
      =
      \left((z^M - z^{-M}) / (z - z^{-1})\right)^4.
  \]
  Further, we have
  \begin{align*}
    % J_M(x)
    % &=
    \frac{z^M - z^{-M}}{z - z^{-1}}
    &= z^{M-1} + z^{M-3} z^{-1} + z^{M-3} z^{-2} + \dots
      + z z^{-M+2} + z^{-M+1} \\
    &= z^{M-1} + z^{M-3} + z^{M-5} + \dots + z^{-M+1}
    = \sum_{\alpha \in \mathcal{A}} z^\alpha,
  \end{align*}
  where $\mathcal{A} := \{-M+1, -M+3, -M+5, \dots, M-1\}$. Squaring yields
  \begin{align*}
     \left(\frac{z^M - z^{-M}}{z - z^{-1}} \right)^2
     &=
     \sum_{\alpha, \beta \in \mathcal{A}}
     z^\alpha z^{\beta} \\
     &= \sum_{\alpha, \beta \in \mathcal{A}}
     \frac{z^{\alpha +\beta} + z^{-\alpha-\beta}}{2} \\
     &= \sum_{\alpha, \beta \in \mathcal{A}} \cos\b( \smfrac{\alpha+\beta}{2} x \b).
   \end{align*}
   Since $\alpha+\beta$ is always even, it follows that
   $(\frac{z^M - z^{-M}}{z - z^{-1}} )^2 \in \Trig_{M-1}$ and in particular
   $J_M \in \Trig_{2M-2}$.
\end{proof}



\begin{lemma}
  $\gamma_M \geq C M^3$ (Remark: $C = 32/\pi^3$)
\end{lemma}
\begin{proof}
  Using the geometrically evident fact that
  \[
    x/\pi \leq \sin(x/2) \leq x/2
  \]
  we can estimate
  \begin{align*}
    \gamma_M
    &= 2 \int_0^\pi  \left( \frac{\sin( Mx/2)}{\sin(x/2)} \right)^4 \, dx  \\
    &\geq 2 \int_0^{\pi/M} \left( \frac{\sin( Mx/2)}{\sin(x/2)} \right)^4 \, dx \\
    &\geq 2c \int_0^{\pi/M} \left( \frac{Mx/2}{x/2} \right)^4 \, dx \\
    &= C M^4 \int_0^{\pi/M} 1 \,dx = C M^3. \qedhere
  \end{align*}
\end{proof}

\begin{lemma} \label{th:trig:jackson_moments}
  \[
    \int_0^\pi x^m J_M(x) \,dx \leq
      \cases{C, & m = 0, \\
            C M^{-1}, & m = 1.
          }
  \]
\end{lemma}
\begin{proof}
  \begin{align*}
    \int_0^\pi x^m J_M(x) \, dx
    &= \sum_{j = 0}^{M-1} \int_{j\pi/M}^{(j+1)\pi/M} x^m J_M(x) \, dx \\
    &\lesssim
        \frac{1}{\gamma_M} \bg[ \int_0^{\pi/M} x^m M^4 \, dx
        + \int_{\pi/M}^{\pi}
             x^m \bg( \frac{1}{x} \bg)^4 \, dx \bg] \\
    &\lesssim
      \frac{1}{M^3} \b[ M^{m+1} + M^{3-m} \b]
    \lesssim
      \cases{
        1, & m = 0, \\
        M^{-1}, & m = 1.
      }
      \qedhere
  \end{align*}
\end{proof}


\begin{theorem}[Jackson's Theorem] \label{th:trig:jackson}
  \begin{enumerate}
  \item Let $f \in C(\TT)$ with modulus of continuity $\omega$, then
  \[
      \| f - J_N \ast f \|_\infty \lesssim \omega(N^{-1}).
  \]
  In particular, if $f \in C^{0,\sigma}(\TT)$, then
  \[
      \|f - J_N \ast f \|_\infty \lesssim N^{-\sigma},
  \]
  and if $f \in C^1(\TT)$, then
  \begin{equation} \label{eq:trig:jackson:C1-version}
    \|f - J_N \ast f \|_\infty \lesssim N^{-1} \|f'\|_\infty.
  \end{equation}
  \item Let $f \in C^p(\TT)$ and $f^{(p)}$ have modulus of continuity $\omega$,
  then
  \[
      \| f - J_N\ast f \|_\infty \lesssim N^{-p} \omega(N^{-1}).
  \]
  \end{enumerate}
\end{theorem}
\begin{proof}
  (1) 
  \begin{align*}
    \b| J_N \ast f(x) - f(x) \b|
    &=
    \bg| \int_{-\pi}^\pi \B( f(x-t) - f(x) \B) J_N(t) \, dt \bg| \\
    &\leq
    \int_{-\pi}^\pi \b|f(x-t) - f(x) \b| J_N(t) \, dt.
  \end{align*}
  Next, we can use the modulus of continuity to estimate
  \[
    \b|f(x-t) - f(x) \b| \leq \sum_{m = 1}^M
      \b| f(x - mt/M) - f(x - (m-1)t/M) \b|
    \lesssim M \omega(t/M)
  \]
  Choosing $M = \lceil tN \rceil$ we obtain
  \[
    \b|f(x-t) - f(x) \b| \lesssim
      \cases{
        \omega(N^{-1}), & 0 \leq |t| \leq N^{-1}, \\
        t N \omega(N^{-1}), & |t| > N^{-1}.
      }
  \]
  Using Lemma~\ref{th:trig:jackson_moments} we conclude
  \[
    \b| J_N \ast f(x) - f(x) \b|
    \lesssim
      \omega(N^{-1}) \int_0^{1/N} J_N(t) \, dt
      + N \omega(N^{-1}) \int_{1/N}^{\pi} t J_N(t)\, dt
    \lesssim \omega(N^{-1}). 
  \]

  (1.5) Before we prove the second Jackson theorem, we need 
  to make another observation: 
  \[
    \| J_N \ast f \|_\infty 
    \leq \|J_N \|_{L^1} \|f\|_\infty,
  \]
  and hence, for any $t \in \Trig_{2N-2}$, 
  \begin{align*}
      \|f - J_N \ast f\|_\infty 
      &\leq  
      \|f - t\|_\infty + \| t - J_N \ast f \|_\infty 
  \end{align*}

  (2) To prove the second Jackson theorem, we first employ 
  \eqref{eq:trig:jackson:C1-version} to deduce that 
  \[
      E_N(f) = E_N(f - J_N \ast f) 
      \leq 
      % C N^{-1} \| (f-J_N \ast f)' \|_\infty 
      % = 
      C N^{-1} \| f' - (J_N \ast f)' \|_\infty.
  \]
  Next, integration by parts yields 
  \[
     (J_N \ast f)'(x) = \int_{-t}^t J_N'(x-t) f(t) \,dt 
     = \int_{-\pi}^\pi J_N(x-t) f'(t) \,dt 
     = J_N \ast f'.
  \]
  Thus, 
  \[
    E_N(f) \leq
  \]

\end{proof}

To prove Theorem~\ref{th:trig:jackson} (2), we need another auxiliary
results that is also of independent interest.

\begin{lemma} \label{th:trig:jackson-auxEN}
  Let $E_N(f) := \inf_{t_N \in \Trig_N} \|f - t_N \|_\infty$, then for
  $f \in C^1(\TT)$ we have
  \[
    E_N(f) \lesssim N^{-1} E_N(f').
  \]
\end{lemma}
\begin{proof}
  Let $s_N \in \Trig_N$ such that
  \[
    \|f' - s_N\|_\infty \leq 2 E_N(f').
  \]
  (In fact, we can replace $2$ with $1$, since the infimum is attained, but
    we don't need this here.) Then we can write
  \[
    s_N(x) = \sum_{k = -N}^N \hat{s}_{k} e^{ikx}.
  \]
  Since $\hat{s}_{0} = \mint_\TT f' \,dx = 0$ we have in fact
  \begin{align*}
    s_N(x)
    =
    \sum_{\substack{k = -N \\ k \neq 0}}^N \hat{s}_{k} e^{ikx}
    =
    \sum_{\substack{k = -N \\ k \neq 0}}^N \frac{\hat{s}_{k}}{ik} \frac{d}{dx} e^{ikx}
    =
    r_N'(x),
  \end{align*}
  where
  \[
    r_N(x) =
    \sum_{\substack{k = -N \\ k \neq 0}}^N \frac{\hat{r}_{k}}{ik} e^{ikx}.
  \]
  Finally, since $r_N \in \Trig_N$, we have $E_N(f) = E_N(f - r_N)$ and
  Jackson's first theorem, specifically the \eqref{eq:trig:jackson:C1-version}
  variant, yields
  \[
    E_N(f)
    = E_N(f - r_N) \lesssim N^{-1} \| f' - r_N' \|_\infty
    = N^{-1} \| f' - s_N \|_\infty
    \lesssim N^{-1} E_n(f'). \qedhere
  \]
\end{proof}


\begin{proof}[Proof of Theorem~\ref{th:trig:jackson} (2)]
  According to Lemma~\ref{th:trig:jackson-auxEN},
  \[
    E_N(f) \lesssim N^{-1} E_N(f') \lesssim \dots
    N^{-p} E_N(f^{(p)}),
  \]
  and according to Jackson's first theorem, for some $M \sim N$,
  \[
    E_N(f^{(p)}) \leq \| f^{(p)} - J_M \ast f^{(p)} \|_\infty
      \leq \omega(M^{-1}) \lesssim \omega(N^{-1}),
  \]
  that is, $E_N(f) \lesssim N^{-p} \omega(N^{-1})$.
  %
  % Note that this does not yet prove our statement. To prove
  % Theorem~\ref{th:trig:jackson} (2), let $t_N \in \Trig_N$ such that
  % $\|f - t_N \|_\infty \leq C N^{-p} \omega(N^{-1})$, then we have
  % \[
  %   \| f - J_M \ast f \|_\infty
  %     \leq \| f - t_N \|_\infty + \| J_M \ast (f - t_N) \|_\infty
  %         + \|t_N - J_M \ast t_N \|_\infty
  % \]
\end{proof}





\subsection{The Paley--Wiener Theorem}
%
\label{sec:trip:pw}
%
If $f$ is analytic on an interval $[a, b]$, then standard theorems of complex
analysis imply the it can be extended to a analytic function in a
neighbourhood $U$ of $[a, b]$. In the case of periodic functions, such a
neighbourhood can be chosen to be a strip,
\[
  \Omega_\alpha := \{ z \in \C \sep |\Im z| < \alpha \},
\]
for some $\alpha > 0$. This is the starting point for a more refined
version of Theorem~\ref{th:trig:decay}(2).

\begin{theorem} \label{th:trig:pw-trefversion}
  Suppose that $f$ is analytic in $\Omega_\alpha$ with
  $\sup_{z \in \Omega_\alpha} |f(z)| = M_\alpha$, then
  \[
    |\hat{f}_k| \leq 2\pi M_\alpha e^{-\alpha|k|}.
  \]
\end{theorem}
\begin{proof}
  Assume $k > 0$; the case $k < 0$ is analogous.
  Recall that
  \[
    \hat{f}_k = \frac{1}{2\pi} \int_{-\pi}^\pi f(x) e^{-ikx} \,dx.
  \]
  We fix some $\beta < \alpha$ and define a complex contour
  \[
    \mathcal{C} := (-\pi, \pi] \cup (\pi, \pi+ \beta i]
        \cup (-\pi + \beta i, \pi + \beta i] \cup (-\pi, -\pi + \beta i]
      = \mathcal{C}_{1} \cup \mathcal{C}_{2}
        \cup \mathcal{C}_{3} \cup \mathcal{C}_{4},
  \]
  to be traversed counterclockwise. In particular $\frac{1}{2\pi i} \int_{\mathcal{C}_1} f(z) e^{ikz} \, dz = \hat{f}_k$, and periodicity of $f$ yields
  \[
      \sum_{j \in \{2, 4\}}  \int_{\mathcal{C}_j} f(z) e^{ikz} \, dz = 0.
  \]
  Combining these observations with Cauchy's theorem yields
  \begin{align*}
      0
      &= \frac{1}{2\pi} \oint_{\mathcal{C}} f(z) e^{ikz} \, dz \\
      &= \sum_{j = 1}^4\frac{1}{2\pi} \int_{\mathcal{C}_j} f(z) e^{ikz} \, dz  \\
      &= \hat{f}_k + \frac{1}{2\pi} \int_{-\pi}^\pi f(x + \beta i) e^{i (x+\beta i) k} \, dx.
  \end{align*}
  Since we assumed that $k > 0$ we have $|e^{i (x+\beta i)k}| = e^{-\beta k}$, hence
  rearranging the previous identity yields the estimate
  \begin{align*}
    |\hat{f}_k| &\leq \mint_{-\pi}^\pi |f(x+\beta i)| e^{-\beta k} \,dx
      \leq M_\beta e^{-\beta k}.
  \end{align*}
  Since the upper bound valid for all $\beta < \alpha$ it also holds for $\beta
  = \alpha$.
\end{proof}

The previous theorem clarifies that, to precisely understand the
best-approximation of an analytic function $f$ by  trigonometric polynomials
we {\em must} study $f$ not on $\TT$ but in the complex plane. While some
further generalisations are possible, we will restrict ourselves mostly
to the context of Theorem~\ref{th:trig:pw-trefversion} and thus look for the
largest $\alpha$ such that $f$ can be extended to a analytic function
on $\Omega_\alpha$.

Suppose we have found an $\alpha$ such that $f \in A(\Omega_\alpha)$. If $f$
blows up at some $x \pm i \alpha$ then we have found the maximal region of
analyticity. If $f$ is bounded in $\Omega_\alpha$ then it is analytic at every
point $z \in \partial \Omega_\alpha$ and hence we can extend $f$ to a
analytic function in a larger domain $\Omega_{\alpha'}$, $\alpha' > \alpha$.
Thus, to determine the maximal region of analyticity we must find the {\em
poles} of $f$. We obtain the following simple corollary of Theorem~\ref{th:trig:pw-trefversion}.

\begin{corollary} \label{th:trig:pw-sharp}
  Let $f \in A(\TT) \cap A(\Omega_\alpha)$ with $\alpha$ maximal, then
  for all $\epsilon > 0$ there exists $C_\epsilon > 0$ such that
  \[
    |\hat{f}_k| \leq C_\epsilon e^{- (\alpha-\epsilon) |k|}.
  \]
  Moreover, we have the approximation error estimate
  \[
    \| f - \Pi_N f \|_{L^\infty} \lesssim C_\epsilon' e^{-(\alpha-\epsilon) N}
    \qquad \forall \epsilon > 0.
  \]
\end{corollary}

\begin{example}[Smeared Zig-Zag]
  Consider a family of periodic functions inspired by our introductory example,
  \[
    f(x) = (1 + c^2 \sin^2 x)^{-1},
  \]
  where $c > 0$. Then the analytic extension is still given by $f(z) = (1 + c^2
  \sin^2 z)^{-1}$. To find the maximal strip of analyticity we need to compute
  the poles, i.e., the points $z \in C$ such that $\eps^2 + \sin^2 z = 0$, or
  equivalently $\sin z = \pm i \eps$, where $\eps = 1/c$.

  To that end, we first note that
  \[
    \sin z = \sin (x + i y) = \sin x \cosh y + i \b\{ \cos x \sinh y \b\}.
  \]
  Thus the poles are given by the solutions to
  \[
       \sin x \cosh y = 0, \qquad \qquad
       \cos x \sinh y = \pm \eps.
  \]
  Since $\cosh y \neq 0$, The first condition requires $\sin x = 0$, or,
  $x \in \pi \Z$, hence $\cos x = \pm 1$. The second condition therefore
  yields $\sinh y = \pm \eps$, or, equivalently,
  \[
      x \in \pi \Z, \qquad y = \pm \sinh^{-1} \eps.
  \]
  This characterises all the poles of $f(z)$, and in particular shows that
  the maximal strip of analyticity is
  \[
      \Omega_{\sinh^{-1} \eps}
  \]
  Our theory therefore predicts (ignoring the $\epsilon$-factors) that
  \[
      |\hat{f}_k| \lesssim e^{- \sinh^{-1} \eps |k|}
            \sim e^{- \eps |k|} = e^{-|k|/c} \qquad \text{for $\eps \sim 0$}
  \]
  as well as the approximation error estimate
  \[
      \| f - f_N \|_\infty \lesssim e^{- \sinh^{-1} \eps N} \sim e^{- \eps |k|}
      = e^{-|k|/c}
      \qquad \text{for $\eps \sim 0$.}
  \]
  After discussing trigonometric interpolation we will show numerical
  tests demonstrating that this is sharp.
\end{example}


Finally, it is also natural to ask about the case when $f$ is entire, i.e.,
$f \in A(\Omega_\alpha)$ for all $\alpha > 0$. In this case, we simply
obtain \Cref{th:trig:pw-sharp} with $\alpha = \infty$:

\begin{corollary} \label{th:trig:pw-entire}
  Suppose that $f \in A(\TT) \cap A(\C)$  (i.e., $f$ is entire), then for all
  $\alpha > 0$ there exists $C_\alpha > 0$ ($C_\alpha =
  \|f\|_{L^\infty(\Omega_\alpha)}$ such that
  \[
      |\hat{f}_k| \lesssim C_\alpha e^{-\alpha |k|}.
  \]
\end{corollary}



\subsection{Interpolation}
%
\label{sec:trig:interp}
%
We have discussed two strategies to construct approximations of functions by
trigonometric polynomials: $L^2$-projection and convolution (e.g., with the
Jackson kernel). While both are constructive, they both require additional 
computational effort to evalute the relevant integrals. Since this is normally 
done via numerical quadrature, additional errors will be introduced that 
need to be analysed separately. All this can be done, but it turns out that 
a much more practical and performant approach that gives ``near-optimal''
approximants is nodal interpolation.

To specify a trigonometric polynomial $t \in \Trig_N$ we need to determine
$2N+1$ coefficients, which should be possible using $2N+1$ function values,
i.e., we may choose $2N+1$ nodes $x_0, \dots, x_{2N} \in (-\pi, \pi]$
and specify 
\[
    t(x_j) = F_j,  
\]
with $F_j$ some prescribed function values. If the $x_j$ are distinct, then it
is easy to prove (see below and Exercise~\ref{exr:poly:interpunique}) If $F_j =
f(x_j)$ for some $f \in C(\TT)$ the we call the resulting $t$ a {\em nodal
interpolant}. 

An important question is how we can transform the nodal values into coefficients
for the trigonometric polynomial. Naively, this can be achieved by simply
solving a linear system for the coefficients at $O(N^3)$ cost: 
Let $t(x) = \sum_{k = -N+1}^N \hat{F}_k e^{ik x}$, then
\begin{equation} \label{eq:trig:pre-dft}
  \sum_{k = -N+1}^N \hat{F}_k e^{i\pi j/N} = F_j.
\end{equation}
It is straightforward to see (we will return to this in \S~\ref{sec:trig:fft}
that the inversion formula is 
\begin{equation} \label{eq:trig:pre-idft}
    \hat{F}_k = \frac{1}{2N} \sum_{j = -N+1}^N F_j e^{-i\pi k/N},
\end{equation}
that is, the linear system \eqref{eq:trig:pre-dft}  has an orthogonal (up to
scaling) which reduces the solution of the linear system matrix reduces the
solution of \eqref{eq:trig:pre-dft} to a matrix-vector multiplication
\eqref{eq:trig:pre-idft} and hence $O(N^2)$ cost. But it turns out that there is
even an $O(N \log N)$ algorithm - the Fast Fourier Transform. To present this
important algorithm it is more convenient if we work with $2N$ interpolation
nodes, instead of $2N+1$ nodes. This makes the theory of interpolation subtly
different, since with $2N$ conditions we can no longer hope to determing $2N+1$
coefficients. 

In the following, we will restrict ourselves to equi-spaced nodes, 
\[
  x_j =  \frac{j \pi}{N},  \qquad j \in \Z.
\]
The $x_j$ are called {\em interpolation nodes}. They depend on $N$, but we
supress this dependence for the sake of simplicity of notation.

To determine a trigonometric polynomial we may, for example, 
drop the $e^{-iNx}$ basis function from $\Trig_N$, which leads to 
interpolants of the form 
\[
    t(x) = \sum_{k = -N+1}^N c_k e^{ik x}.
\]
But unless $c_N = 0$, this will mean that $t(x) \not\in \R$ even if all $f_j \in
\R$. From \eqref{eq:trig:pre-idft} we see that $c_N \in \R$, hence taking 
the real part of the last group yields 
\[
    \Re \big[ c_N e^{i N x} \big]   = c_N \cos(Nx),
\]
which is the convention normally taken when an even number of interpolation 
points is used. 

Thus, we can define the modified trigonometric polynomial space 
\[
  \Trig_N' := {\rm span}\Big(\Trig_{N-1} \cup \{ \cos N x \} \Big)
    =  \bg\{ t(x) = \sum_{k = -N+1}^{N-1} c_k e^{ikx} + c_N \cos(Nx) \bg\}.  
\]

There is a second good reason for making this modification: the two 
basis functions $e^{iNx}, e^{-iNx}$ agree on the interpolation nodes 
$x_j = j \pi / N$: 

\begin{lemma} \label{th:trig:baby-aliasing}
  Let $x_j = j \pi / N$, then $e^{iN x_j} = e^{-iNx_j}$ for all 
  $j \in \Z$.
\end{lemma}
\begin{proof}
  \[
    e^{iNx_j} = e^{i \pi j} = (-1)^j = (-1)^{-j} = e^{-i\pi j} = e^{-iNx_j}.
    \qedhere
  \]
\end{proof}


Finally, to prepare us for discussing the FFT in the next section, we will
change the interpolation condition to the nodes $x_0, \dots, x_{2N-1}$, which is
of course equivalent due to $2\pi$-periodicity.

\begin{lemma}
  Let $F  = (F_j)_{j = 0}^{2N-1} \in \C^{2N}$, then there exists a unique 
  $t \in \Trig_N'$ such that 
  \[
    t(x_j) = F_j, \qquad j = 0, \dots, 2N-1.
  \]
\end{lemma}
\begin{proof}
  According to Lemma~\ref{th:trig:baby-aliasing} we need to solve 
  \begin{align*}
      && \sum_{k = -N+1}^N c_k e^{i\pi k j/N} &= F_j \\ 
      \Leftrightarrow &&
      \sum_{k = -N+1}^N c_k \big(e^{i\pi j/N}\big)^k &= F_j \\ 
      \Leftrightarrow &&
      \sum_{k = -N+1}^N c_k z_j^k &= F_j, \\ 
      \Leftrightarrow &&
      \sum_{k = -N+1}^N c_k z_j^{k+N-1} &= F_j z_j^{N-1},
  \end{align*}
  where $z_j = e^{i\pi x_j}$ are distinct complex interpolation nodes. Existence
  and uniqueness of algebraic polynomial interpolation gives the stated result.
  (cf. Exercise~\ref{exr:poly:interpunique}).

  {\it REMARK: } the last line in the above chain was unnecessary, but we will 
  revisit this later.
\end{proof}

\medskip 

\begin{definition}
  Let $f \in C(\TT)$ then we define $I_N f \in \Trig_N'$ to be the unique nodal
  interpolant of $f$ at the nodes $x_j = \pi j / N, j \in \Z$, i.e., $I_N f(x_j)
  = f(x_j)$ for $j \in \Z$.
\end{definition}

\medskip 

\begin{remark} 
  The choice of the equi-spaced grid $\{x_j\}$ may seem completely arbitrary,
  and there is {\it a priori} no guarantee that it is optimal or even close to
  optimal. Nevertheless, our introductory example already hints that it is not
  such a bad choice. We will prove in the remainder of this section that it is
  optimal up to a logarithmic factor. We will also return to a more careful
  discussion of different choices of interpolation nodes in \S~\ref{sec:poly}.
\end{remark}

To understand the approximation error of the $I_N f$, let $f \in C(\TT)$, $t_N
\in \Trig_N'$ arbitrary, then
\begin{align*}
  \|f - I_N f \|_\infty &\leq \|f - t_N \|_\infty + \| t_N - I_N f\|_\infty \\
    & = \|f - t_N \|_\infty + \| I_N (t_N - f) \|_\infty \\
    & \leq (1 + \|I_N\|) \| t_N - f \|_\infty,
\end{align*}
where $\|I_N\|$ is the operator norm of $I_N$ associated with $\|\cdot\|$, 
defined by 
\[
    \| I_N \| = \sup_{\substack{f \in C(\TT) \\ \|f\|_\infty = 1}} \| I_N f \|_\infty.
\]
Taking the infimum over all $t_N \in \Trig_N$ we obtain that the
interpolation error deviates from the best approximation error by
factor determined by the operator norm of $I_N$, i.e.,
\[
    \|f - I_N f \| \leq (1 + \|I_N\|) \inf_{t_N \in \Trig_N'} \|f - t_N \|
    \leq (1 + \|I_N\|) \inf_{t_{N-1} \in \Trig_{N-1}} \|f - t_{N-1} \|,
\]
where the final inequality of course shows that the convergence rate does not
change asymptotivally from that in $\Trig_N$ except possibly for a constant
factor.

\begin{definition}
  The interpolation operator norm is also called {\em Lebesgue constant}, and
  typically denoted by $\Lambda_N = \| I_N \|$.
\end{definition}

\begin{remark}
  The above argument works in principle with {\em any} norm. But to obtain a
  finite bound that norm must be such that $C(\TT)$ is complete under it. If
  not, then $\|I_N\|$ becomes infinite. As an exercise, you may check that the
  $L^2$-operator norm, $\|I_N\|_{L(L^2)}$ is indeed infinite. This is not
  surprising since functions $f \in L^2$ do not have well-defined point values.
\end{remark}

To estimate $\Lambda_N$ we wish to write $I_N f$ in terms of a {\em nodal
basis}, i.e.,
\[
  I_N f(x) = \sum_{j = -N+1}^N f(x_j) L_j(x),
\]
where $x_j = \pi j / N$, then we can simply estimate
\begin{align} \label{eq:trig:LamNbound}
  \Lambda_N \leq \sup_{x \in \TT}  \sum_{j = -N+1}^N |L_j(x)|.
\end{align}

An immediate observation is that, since the grid is translation invariant, the
nodal basis will be translation invariant as well, i.e., $L_j(x) = L_0(x -
x_j)$. This already gives us a hint what to look for.

\begin{lemma}
  The nodal basis for trigonometric interpolation (for an even number of
  grid points $2N$) is given by a modified Dirichlet kernel,
  \[
    % L_j(x) = \frac{\sin\b( N (x-x_j) \b)}{N \tan\b( \smfrac12 (x-x_j)\b)}
    %     = \frac{{\rm sinc}\b( N (x - x_j)\b)}{N {\rm sinc}\b( \smfrac12 (x-x_j)\b)}
    %         \cos\b( \smfrac12 (x-x_j) \b).
    L_j(x) = D_N'(x-x_j)
  \]
  where
  \[
    D_N'(x) = \frac{\sin(Nx)}{2N \tan(x/2)}.
  \]
\end{lemma}
\begin{proof}
  To see the identity for $D_N'$ simply use $\sin(\alpha+\beta) =
      \sin\alpha\cos\beta + \cos\alpha\sin\beta$. From the definition of $D_N'$
      it is straightforward to check that $L_j(x_i) = \delta_{ij}$. For the case
      $i = j$ this is a limit argument: (fill in the details!)
  \[
    \lim_{x \to x_j} L_j(x) = 1.
  \]
  Thus we ``only'' need to show that $L_j$ is indeed a trigonometric polynomial,
  or equivalently, $D_N \in \Trig_N$. This is achieved by an analogous argument
  as for the Jackson kernel.

  See also Exercise~\ref{exr:trig:dirichlet} for the Dirichlet kernel related to
  $\Trig_N$ and how it relates to $L^2$-projection.
\end{proof}


See \nbtrig for a numerical exploration of $\Lambda_N := \|I_N\|_{\rm op}$. The
numerical experiments shown there suggest that the following theorem holds.

\begin{theorem} \label{th:trig:lebesgue}
  The Lebesgue constant for trigonometric interpolation with respect to the
  $L^\infty$-norm is bounded by
  \[
    \|I_N\| \leq \smfrac{2}{\pi} \log (N+1) + 1. 
  \]

  In particular, if $f \in C(\TT)$, then
  \[
      \| f - I_N f \|_{L^\infty}
        \lesssim \log N \inf_{t_N \in \Trig_N'} \| f - t_N \|_{L^\infty}.
  \]
\end{theorem}
\begin{proof}
  We will only prove a slightly weaker upper bound $\|I_N\| \leq \frac{2}{\pi} \log N + 2$. Recall
\eqref{eq:trig:LamNbound}, then we need to bound
  \begin{align*}
    \Lambda_N
    &\leq
    \sum_{j =  -N+1}^N |D_N'(x - x_j)|.
  \end{align*}
  By translation invariance and reflection symmetry we only need to consider $x
  = -t$, $t \in (0, \frac{\pi}{2N})$ (the case $t = 0$ is trivial); in this
  case,
  \begin{align*}
    \Lambda_N
    &\leq
    \sum_{j =  -N+1}^N |D_N'(x - x_j)| \\
    &\leq
    \sum_{j =  0}^N |D_N'(x_j+t)|  + \sum_{j = -N+1}^{-1} \dots \\
    &\leq
    \frac{1}{2N} \Bg\{
        \frac{\sin(Nt)}{\tan(t/2)}
        + \sum_{j = 1}^N
        \bg|\frac{\sin\b(N(x_j+t)\b)}{\tan\b((x_j+t)/2\b)}\bg|
      \Bg\} + \dots \\
    &\leq
    \frac{1}{2N} \bg\{
        \frac{Nt}{t} + \sum_{j = 1}^N \frac{2}{ x_j+t }
    \bg\} + \dots \\
    &\leq 
    1 + \frac{1}{\pi} \sum_{j = 1}^N \frac{1}{j} + \dots \\ 
    &\leq 
    2\Big( 1 + \smfrac{1}{\pi} \log(N+1) \Big) 
    \leq 2 + \smfrac{2}{\pi} \log(N+1).
    \qedhere
  \end{align*}
\end{proof}


\subsection{The Fast Fourier Transform}
%
\label{sec:trig:fft}
%
As a final topic on the theme of trigonometric polynomial approximation we will
study how to work efficiently with trigonometric interpolants. This is acieved
via the discrete Fourier transform and its fast implementation, the {\it Fast
Fourier Transform}, likely one of the most important and most widely used
numerical algorithms.

For the following discussion it is best to assume that the number of grid points
is even, in particular we will discuss the FFT only for this case.

Given a function $f \in C(\TT)$ we can evaluate it at grid points $x_j$ which
leads to a grid function $F_j = (f(x_j))_{j=0}^{M-1}$. Given $M \in 2\N$ it is
common to define the DFT and FFT for the grid
%
\[
    x_j = \frac{2\pi j}{M} \qquad j = 0, \dots, M-1.
\]
%
The assumption that $M = 2N$ is even is consistent with
\S~\ref{sec:trig:interp}. In our notation up to now it would have been more
natural to write $x_j = -\pi + \pi j/N$ instead, but since we are considering
periodic functions we just need to shift them into a new domain $[0, 2\pi)$.
Although we could initially avoid some inconveniences we want to eventually be
able to use the FFT algorithms, so we may as well learn now how to convert
between the two representations.

We then ask, what are the coefficients of the trigonometric polynomial $t \in
\Trig_{N}' = \Trig_{M/2}'$ such that
%
\[
  t(x_j) = F_j \qquad \text{for } j = 0, \dots, M-1.
\]
%
We have already seen in \eqref{eq:trig:pre-dft} that these are provided by the
DFT operator: for $F \in \C^{M}$, $k = 0, \dots, M-1$,
%
\begin{align}
  \label{eq:trig:dft}
  {\rm DFT}[F] := \hat{F}, \quad \text{where} \quad
  \hat{F}_k &= \frac{1}{M} \sum_{j = 0}^{M-1} F_j e^{-i x_j k} \\
  \notag
            &= \frac{1}{M} \sum_{j = 0}^{M-1} F_j e^{-i 2\pi j k / M}.
\end{align}
%
Note in particular that this is a trapezoidal rule approximation of
\eqref{eq:trig:fourier coeffs}.

\begin{remark} \label{rem:trig:k-grid}
  Since $x_j = 2 \pi j/ M$ it follows that
  \[
    e^{-i x_j (k \pm M)} = e^{-i x_j k}
  \]
  and hence the $k$-grid $\{0, \dots, M-1\}$ can alternatively be interpreted
  as, with $N = M/2$, 
  \[
    \{ 0, \dots, N, -N+1, -N+2, \dots, -1 \}. \qquad \qedhere
  \]
\end{remark}


\begin{proposition} \label{th:trig:dft}
  Let the ${\rm IDFT}$ be defined by
  \begin{equation} \label{eq:trig:idft}
    U = {\rm IDFT}[\hat{U}], \quad \text{where} \quad
    U_j := \sum_{k = 0}^{M-1} \hat{U}_k e^{i x_j k}
        = \sum_{k = 0}^{M-1} \hat{U}_k e^{i 2\pi j k/ M},
  \end{equation}
  then
  \[
    {\rm IDFT}\big[ {\rm DFT}[F] \big] = F \qquad \forall F \in \C^M.
  \]
  In particular, if $\hat{F} = {\rm DFT}[F]$, then the two trigonometric
  polynomials (cf. Remark~\ref{rem:trig:k-grid}) $t \in \Trig_N, t' \in
  \Trig_N'$
  \begin{align*}
    t(x) &= \sum_{k = 0}^{M-1} \hat{F}_k e^{i k x} \\ 
    t'(x) &= \sum_{k = 0}^{M/2-1} \hat{F}_k e^{i k x}
          + \hat{F}_{M/2} \cos(M/2 x) + \sum_{k = M/2+1}^{M-1} \hat{F}_k e^{i k x} \\  
  \end{align*}
  interpolate $(x_j, F_j)_{j = 0}^{M-1}$, i.e.,
  \[
    t(x_j) = t'(x_j) = F_j \qquad \text{for } j = 0, \dots, M-1.
  \]
\end{proposition}
\begin{proof}
  Left as an exercise.
\end{proof}

Using expression \eqref{eq:trig:dft} the cost of computing ${\rm DFT}[F]$ is
$O(N^2)$. Indeed, this is the cost of a generic matrix-vector multiplication,
i.e., applying a linear operation in $\R^N \to \R^N$ that has no special
structure. Luckily the ${\rm DFT}$ has plenty of structure to exploit, which
finally brings us to the FFT algorithm (specifically the radix-2 variant of
Cooley--Tukey's algorithm, though the idea famously goes back to Gaussz).

We begin by rewriting
\[
  \hat{F}_k = M^{-1} \sum_{j = 0}^{M-1} F_j \omega^{kj},
  \qquad \text{where} \quad \omega := e^{-i 2\pi/M}.
\]
Then,
\begin{align}
  \notag
  \hat{F}_k  &= \sum_{j = 0}^{M/2-1} F_{2j} \omega^{2kj}
      + \sum_{j = 0}^{M/2-1} F_{2j+1} \omega^{k(2j+1)} \\
  \notag
    &= \sum_{j = 0}^{M/2-1} F_{2j} \omega^{2kj}
        + \omega^k \sum_{j = 0}^{M/2-1} F_{2j+1} \omega^{2kj} \\
  \label{eq:trig:fft_split}
    &=: \hat{G}_k + \omega^k \hat{H}_k.
\end{align}
In particular, since $\omega^2 = e^{-i2\pi/(M/2)}$, we note that $\hat{G}_k$ is
the DFT of $(F_{2j})_{j=0}^{M/2-1}$, while $\hat{H}_k$ is the DFT of
$(F_{2j+1})_{j=0}^{M/2-1}$.

A final remark is that, {\it a priori} $\hat{G}_k$ and $\hat{H}_k$ will be given
only for $k = 0, \dots, M/2-1$, but the expressions are $M/2$-periodic and
\eqref{eq:trig:fft_split} allows us to recover $\hat{F}$ for all $k = 0, \dots,
M-1$. Specifically, we obtain the following identity:
%
\begin{equation} \label{eq:trig:fft_trick}
  \begin{split}
    \hat{F}_k &= \hat{G}_k + \omega^k \hat{H}_k, \qquad k = 0, \dots, M/2-1, \\
    \hat{F}_k &= \hat{G}_{k-M/2} - \omega^{k-M/2} \hat{H}_{k-M/2},
      \qquad k = M/2, \dots, M-1.
  \end{split}
\end{equation}
(We could also write $\omega^k$ instead of $\omega^{k-M/2}$; this is
equivalent.)
% The case $k = 0, \dots, N/2-1$ is already clear from
% \eqref{eq:trig:fft_split}. For $k > N/2-1$ it follows from the
% $N/2$-periodicity of $\hat{G}_k, \hat{H}_k$, i.e.,
% \[
%   \hat{G}_{k+N/2} = \hat{G}_k, \qquad \text{and} \qquad
%   \hat{H}_{k+N/2} = \hat{H}_k,
% \]
% as well as
% \[
%   \omega^{N/2} = e^{-i2\pi(N/2)/N} = e^{-i\pi} = -1.
% \]

Suppose now that $M/2$ is still divisible by 2, then the can split the
computation of $\hat{F}, \hat{G}$ again into four smaller DFTs. This process can
of course be iterated. If $M = 2^m$, then after $m \approx \log M$ iterations
iterations we compute $\approx M$ DFTs of length $O(1)$. Combining the small
DFTs into the larger DFTs requires $O(M)$ operations at each level. Since there
are $O(\log M)$ levels, this means that the cost of computing the original DFT
is $O(M \log M)$. Algorithms that use some variant of this strategy are called
{\em Fast Fourier Transform}s.



\subsection{Examples}
%
We are now fully equipped to applying trigonometric polynomial approximation for
numerical simulation. We will consider
\begin{itemize}
  \item a linear, homogeneous boundary value problem
  \item a transport equation with variable coefficients
  \item a filtering problem
\end{itemize}
These examples may be found in \nbtrig.




\subsection{Exercises}

\begin{exercise} \label{exr:trig:hilbert-onb}
  \begin{enumerate} \ilist
    \item Recall the definition of a complex Hilbert space and
        check that $( L^2(\TT), \< \cdot, \cdot \>_{L^2(\TT)} )$ is indeed
        a pre-Hilbert space, i.e. check all conditions except for completeness.
        (Completeness is a bit more involved, but it is not particularly
        difficult; feel free to look this up  in a suitable textbook.)

    \item Complete the proof of the Plancherel Theorem; i.e.
      Theorem~\ref{th:trig:plancherel}(ii).

    \item Using Jackson's theorem, prove also
    Theorem~\ref{th:trig:plancherel}(i).

    {\it Hint: use the fact that $\Pi_N$ is an orthogonal projector and
    in particular has operator norm 1.}
  \end{enumerate}
\end{exercise}

\begin{exercise} \label{exr:trig:convergence_L2}
  Complete the proof of Theorem~\ref{th:trig:convergence_L2}.
\end{exercise}

\begin{exercise} \label{exr:trig:functions}
  For the following functions $f$, categorize their regularity as closely as
  possible and estimate the rate of convergence of $\|f-\Pi_N f\|_{L^2}$.
  \begin{enumerate} \ilist
    \item  $f(x) = \sin(x)$
    \item  $f(x) = \sin(x/2)$
    \item $f(x) = |\sin(x)|$
    \item $f(x) = |\sin(x)|^3$
    \item $f(x) = (1 + c^2 \sin^2 x)^{-1}$
    \item $f(x) = \exp( - \sin(x))$
    \item $f(x) = \exp( - 1 / (1-x^2) ) \chi_{(-1,1)}(x)$, extended $2\pi$-periodically to $\R$.
    % x^2 < 2.49999^2 ? exp(3 - 3 / (1-(x/2.5)^2)) : 0.0
  \end{enumerate}
  Can you sharpen your estimates after working through
  Exercise~\ref{exr:trig:gibbs}?
\end{exercise}


\begin{exercise}[Gibbs Phenomenon] \label{exr:trig:gibbs}
  Consider the periodic, piecewise constant function
  \[
      f(x) = \cases{
        1, & x \in (0, \pi], \\
        -1, & x \in (-\pi, 0].
      }
  \]
  \begin{enumerate} \ilist
  \item Prove that, there exists no sequence of trigonometric polynomials
  $t_N \in \Trig_N$ such that $t_N \to f$ uniformly, but that
  \[
    \|\Pi_N f - f\|_{L^2} \to 0 \qquad \text{as } N \to \infty.
  \]

  \item Show that the Fourier series for $f$ is given by
  \[
    \Pi_N f(x) = \frac{4}{\pi} \sum_{\substack{j = 1 \\ j \text{ odd}}}^{N}
        \frac{\sin(jx)}{j}.
  \]

  \item Deduce that
  \[
      \| \Pi_N f - f \|_{L^2} \lesssim N^{-1/2}.
  \]

  \item {\bf Gibbs Phenomenon: } Prove that
  \[
    \lim_{N \to \infty} \Pi_N f\b(\smfrac{\pi}{N} \b) > 1
  \]
  You may use without proof that
  \[
      \int_0^\pi \frac{\sin(t)}{t} \,dt \approx
      \frac{\pi}{2} + \pi \cdot (0.089489\dots).
  \]

  {\it If you plot $\Pi_N f$ you will observe oscillations around the
  discontinuity. This ``picture'' is what is commonly known as the Gibbs
  phenomenon. It is a special case of {\bf ringing artefacts}, which are
  a common occurance when piecewise smooth data is approximated using
  global basis functions. This can be nicely visualised in image processing;
  see e.g. {\tt https://en.wikipedia.org/wiki/Ringing\_artifacts}.}

  \item {Piecewise smooth functions: } Make an educated guess what the
  rate of convergence is for $\|\Pi_N f - f \|_{L^2}$ when $f$ is piecewise
  $C^\infty(\TT)$, all derivatives up to $f^{(p-1)}$ are continuous and
  $f^{(p)}$ has jump discontinuities at finite many points. This includes
  functions such as $|\sin(nx)|, |\sin(nx)|^q$ for $q$ odd.

  {\it Hint: A rigorous derivation of this convergence rate is quite
  possible; consider the function $g(x) = \sin(x/2)$, continued periodically.}
  %
  \qedhere
  \end{enumerate}
\end{exercise}


\begin{exercise}[Dirichlet Kernel] \label{exr:trig:dirichlet}
  \begin{enumerate} \ilist
    \item Prove that
    \[
      D_N(x) = \frac{\sin\b((N+1/2) x\b)}{\sin(x/2)}
            = 1 + 2 \sum_{k = 1}^N \cos(k x)
            = \sum_{k = -N}^N e^{ikx}.
    \]
    \item Deduce that,
    \[
      (D_N \ast e^{in \bullet})(x) =
        \cases{
          e^{inx}, & -N \leq n \leq N, \\
          0, & \text{otherwise}
        }
    \]
    \item Deduce that, if $f \in L^1(\TT)$, then
    \[
        D_N \ast f = \Pi_N f.
    \]
    \item Show that $\|D_N\|_{L^1} \lesssim \log N$ and hence
    \[
        \| D_N \ast f \|_{L^\infty} \leq \|D_N \|_{L^1} \|f\|_\infty
          \lesssim \log N \|f\|_\infty.
    \]
    {\it HINT: to estimate $D_N$ use a similar splitting into sub-intervals
    as in the Jackson kernel estimates.}
    \item Deduce that
    \[
        \| f - \Pi_N f \|_{\infty}
        \lesssim  \log N \inf_{t_N \in \Trig_N} \| f - f_N \|_\infty,
    \]
    and in particular, if $f \in C^p(\TT)$ and $f^{(p)}$ has modulus of
    continuity $\omega$, then
    \[
        \| f - \Pi_N f \|_\infty \log N N^{-p} \omega(N^{-1}). \qedhere
    \]
    % Hint: use the fact that $\Pi_N t_N = t_N$ for all $t_N \in \Trig_N$.

  \end{enumerate}
\end{exercise}


\begin{exercise} \label{exr:trig:periodic extension}
  Let $f \in A(\TT)$. Prove that there exists $\alpha > 0$ such that $f$ has an
  analytic extension to $\Omega_\alpha$. Further, show that this extension
  (still called $f$) must be $2\pi$-periodic, i.e.,
  \[
      f(x + i y) = f(x + 2\pi + i y) \qquad \forall x+iy \in \Omega_\alpha. \qedhere
  \]
\end{exercise}


\begin{exercise}[The Exponentially Convergent Trapezoidal Rule]
  \label{exr:trig:trapezoidal rule}
  Let $f \in A(\TT)$, and consider the trapezoidal rule approximation
  of $I[f] := \mint_{-\pi}^\pi f\,dx$;
  \[
    Q_N[f] := \frac{1}{2N} \sum_{j = -N+1}^N f(x_j),
  \]
  where $x_j := j\pi/N$.
  %
  \begin{enumerate} \ilist
    \item Prove that,
    \[
        \frac{1}{2N} \sum_{j = -N+1}^N e^{ikx_j} =
          \cases{
              1, & k \in 2N \Z, \\
              0, & \text{otherwise.}
          }
    \]

    \item Suppose $f$ is analytic in $\Omega_\alpha$, where $\alpha > 0$ is
    maximal. Derive a sharp convergence rate for $|Q_N[f] - I[f]|$.
    {\it (You may of course revisit our sketches from the introductory lecture.)}

    \item {\it Poisson's example: } The perimeter of an ellipse with axis
    lengths $1/\pi, 0.6/\pi$ is given by the integral
    \[
        I = \frac{1}{2\pi} \int_{-\pi}^\pi \sqrt{1 - 0.36 \sin^2\theta}\,d\theta.
    \]
    {\it (You may justify this, but this is not required.)}
    \begin{itemize}
        %
      \item Compute the region of analyticity for $f(\theta) = \sqrt{1 - 0.36
      \sin^2\theta}$, hence prove a rate of convergence for $Q_N[f]$.
      {\it (For this problem, you also need to estimate the prefactor!)}
      %
      \item Only solve one of the following two problems:

      (OPTION 1) How many terms to you need to obtain 3, 5, 7 digits of accuracy?
      Using only a calculator, compute $I[f]$ to within 3 digits of accuracy.
      How many ``non-trivial'' function evaluations did you need?

      (OPTION 2) numerically demonstrate the convergence (use Julia, Matlab, Python or any language you wish.) \qedhere
    \end{itemize}
  \end{enumerate}
\end{exercise}


% \begin{exercise}[Spectral Differentiation]
% \end{exercise}


\begin{exercise}
  Prove Proposition~\ref{th:trig:dft}.
\end{exercise}

\begin{exercise}
  {\bf Radix-3 FFT: } Instead of $M$ even suppose that $M = 3 M'$ (you may
  actually still assume that $M$ is even for consistency with our treatment of
  trigonometric interpolation, but this is not really relevant here). Generalise
  the FFT to this case, i.e., derive the analogues of \eqref{eq:trig:fft_split}
  and \eqref{eq:trig:fft_trick}.

  (Bonus: Can you also generalise to $M = n M'$?)
\end{exercise}


\clearpage
% !TEX root = apxthy.tex


\section{Algebraic Polynomials}
%
\label{sec:poly}
% 
Our second major topic concerns approximation of functions defined on an
interval $f : [-1, 1] \to \R$, without loss of generality. But contrary to
\S~\ref{sec:trig} we no longer assume periodicity. Instead we will approximate
$f$ by algebraic polynomials,
\[
      f(x) \approx p_N \in \Poly_N
\]
where $\Poly_N$ denotes the space of degree $N$ polynomials,
\[
   \Poly_N := \bg\{ \sum_{n = 0}^N c_n x^n \bsep c_n \in \R \bg\}.
\]
Note in particular that in the terms of "simplicity" these are indeed the 
simplest functions to evaluate numerically in that they only require addition 
and multiplication operations. 

In terms of a basic convergence result we have the following initial 
proposition, which we will not prove now, but it will follow from our 
later work.

\begin{proposition}[Weierstrass Approximation Theorem] \label{th:poly:Weierstrass}
   $\bigcup_{N \in \N} \Poly_N$ is dense in $C([-1,1])$ and by extension also 
   in $L^p(-1,1)$ for all $p \in [1, \infty)$.
\end{proposition}

\bigskip 

Indeed, as we have argued before, convergence in itself of {\em some} sequence
of approximations  is rarely useful, but we require (i) rates and (ii) explicit
constructions. Much of this chapter is therefore devoted to interpolation.

It is a standard fact (and easy to prove) that for any $N+1$ distinct points
$x_0, \dots, x_N \in \R$ and values $f_0, \dots, f_N$ there exists exactly one
polynomial $p_N \in \Poly_N$ interpolating those values, i.e., 
\[
   p_N(x_j) = f_j, \qquad j = 0, \dots, N.
\]
(Indeed, the same is even true for $x_j \in \C$.) These equations form 
a linear system for the coefficients $c_n$, which can be solved to obtain 
the interpolation polynomial, which in turn can be easily readily 
numerically. 

A key question is how to choose the interpolation points $x_j$? It may seem
intuitive to take equispaced nodes, $x_j = -1 + 2j/N$.  We start this section by
exploring precisely this approach to approximate some smooth functions on
$[-1,1]$; see \nbpoly for some motivating examples. In this Julia notebook we
clearly observe that this yields a divergent sequence of polynomials, but by
exploring also other kinds of fits we also see that this does not preclude the
possibility of computing a (very) good approximation. We therefore focus
initially by deriving a ``good'' set of interpolation nodes. The same idea will
also naturally lead to the Chebyshev polynomials.


\subsection{Chebyshev Points, Chebyshev Polynomials and Chebyshev Series}
%
We can motivate the idea of the Chebyshev points by mapping the polynomial
approximation problem to the trigonometric approximation problem:

Let $f\in C([-1,1])$, then let $g \in C(\TT)$ be defined by
\[
   g(\theta) = f(\cos\theta).
\]
Note that $g$ ``traverses'' $f$ twice!

We will later see that $g$ inherits the regularity of $f$ even across domain
boundaries; for now let us understand the consequence of this observation. We
know from \S~\ref{sec:trig} that equispaced interpolation of $g$ yields an
excellent trigonometric interpolant, i.e., we choose $\theta_j = -\pi + 2\pi
j/N$ and we choose coefficients $\hat{g}_k$ such that
\[
   t_N(\theta_j) = \sum_{-N}^N \hat{g}_k e^{ik \theta_j} = g(\theta_j)
\]
%
We may ask to interpolate $f$ at the analogous points, $x_j = \cos(\theta_j)$
but since $g$ contains ``two copies'' we only take half of the nodes.
This gives the Chebyshev nodes 
%
\begin{equation} \label{eq:poly:chebnodes}
   x_j := \cos\b( \pi j / N \b) \qquad j = 0, \dots, N.
\end{equation}

We can readily test our hypothesis that these yield much better approximations;
see again \nbpoly. Thus, for future reference we define the Chebyshev
interpolant $I_N f$ to be the unique function $I_ f \in \Poly_N$ such that
\[
   I_N f(x_j) = f(x_j) \qquad \for j = 0, \dots, N,
\]
where $x_j$ are the Chebyshev nodes \eqref{eq:poly:chebnodes}.


Next, we ask what the analogue of the Fourier series is. We write
\[
   g(\theta) = \sum_{k \in \Z} \hat{f}_k e^{ik\theta},
\]
then using that $g$ is real and $g(-\theta)=g(\theta)$,
\[
   g(\theta) = \hat{g}_0 + 2 \sum_{k = 1}^N \hat{g}_k \cos(k\theta)
\]
It is therefore natural to define the {\em Chebyshev polynomials}
%
\begin{equation} \label{eq:poly:defn_Tk}
   T_k(\cos\theta) = \cos(k\theta), \qquad k \in \N := \{0,1,2,\dots\}.
\end{equation}
%
A wide-ranging consequence of this definition is that
\[
      |T_k(x)| \leq 1 \qquad \forall k.
\]

\begin{lemma} \label{th:poly:chebpolys}
   The functions $T_k : [-1,1] \to \R$ are indeed polynomials and
   satisfy the recursion
   \begin{equation} \label{eq:poly:chebrecursion}
      T_{k+1}(x) = 2 x T_k(x) - T_{k-1}(x),
   \end{equation}
   with initial conditions $T_0(x) = 1, T_1(x) = x$.
\end{lemma}
\begin{proof}
   The identities $T_0(x) = 1, T_1(x) = x$ follow immediately from
   \eqref{eq:poly:defn_Tk}. If we can prove the recursion, then
   the fact that $T_k$ are polynomials follows as well.

   To that end, we introduce another representation,
   \[
      T_k\B( \smfrac{z + z^{-1}}{2} \B)
      = T_k(\Re z)
      = \Re z^k = \frac{z^k + z^{-k}}{2},
   \]
   where $|z| = 1$. Then,
   \begin{align*}
      & \hspace{-1cm} T_{k+1}(\Re z) - 2 \Re z T_k(\Re z) + T_{k-1}(\Re z) \\
      &= \smfrac12 \B(
         z^{k+1} + z^{-k-1}  - (z+z^{-1}) (z^k+z^{-k})
         + z^{k-1} + z^{-k+1} \B) \\
      &= \smfrac12 \B( z^{k+1} + z^{-k-1}
               - z^{k+1} - z^{k-1} - z^{1-k} - z^{-1-k}
               + z^{k-1} + z^{-k+1} \B) \\
      &=0. \qedhere
   \end{align*}
\end{proof}

For future reference we define the Joukowsky map
\[
   \phi(z) = \frac{z+z^{-1}}{2}.
\]
and note that it is analytic in $\C \setminus \{0\}$.

We now know that $T_k(x)$ are indeed polynomials of degree $k$ and in light of
the foregoing motivating discussion, we have the following result.

\begin{lemma}
   Let $f \in C([-1,1])$ is uniformly continuous, then there exists {\em
   Chebyshev coefficients} $\tilde{f}_k \in \R$ such that the {\em Chebyshev
   series}
   \begin{equation} \label{eq:poly:chebseries}
      f(x) = \sum_{k = 0}^\infty \tilde{f}_k T_k(x)
   \end{equation}
   is absolutely and uniformly convergent.

   The Chebyshev coefficients are given by the following equivalent formulas,
   \begin{align*}
      \tilde{f}_k
      &=  \frac{2}{\pi} \int_{-1}^1 \frac{f(x) T_k(x)}{\sqrt{1-x^2}} \,dx \\
      &=  \frac{1}{2\pi i} \oint_{\SS} \,\,\b(z^{-1+k} + z^{-1-k}\b) f(\phi(z))
                  \, dz \\
      &= \frac{1}{\pi i} \oint_{\SS} \,\, z^{-1+k} f(\phi(z)) \, dz \\
      &= \frac{1}{\pi i} \oint_{\SS} \,\, z^{-1-k} f(\phi(z)) \, dz.
   \end{align*}
   For $k = 0$ a factor $1/2$ must be applied.
\end{lemma}
\begin{proof}
   If $f \in C([-1,1])$ with modulus of continuous $\omega$, then $g \in C(\TT)$
   also has a modulus of continuity and hence the Fourier series converges
   uniformly and equivalently, the Chebyshev series does as well.

   The expressions for $\tilde{f}_k$ are simply transplanting the fourier
   coefficients $\hat{g}_k$ to Chebyshev coefficients $\tilde{f}_k$.
\end{proof}

In analogy with the truncation of the Fourier series $\Pi_N g$ (which
is the $L^2(\TT)$-projection or best-approximation we define
the Chebyshev projection
\[
   \PCheb_N f(x) := \sum_{k = 0}^N \tilde{f}_k T_k(x).
\]


\subsection{Convergence rates}
%
\label{sec:poly:rates}
%
As we have learned in \S~\ref{sec:trig}, the real power of polynomials is in
the approximation of analytic functions, hence we begin again with this
setting.

Intuitively, the idea is that analyticity of $f$ on $[-1,1]$ translates into
analyticity of the corresponding periodic function $g(\theta) = f(\cos\theta)$.
Exponential decay of the Fourier coefficients $\hat{g}_k$ then translates into
exponential decay of the Chebyshev coefficients  $\tilde{f}_k$. But we can prove
this exponential decay directly with a relatively straightforward variation of
the argument we used in \S~\ref{sec:trip:pw}, which is interesting to see the
analogies.


We begin by defining
\[
   F(z) := f(\Re z) = f\b( \smfrac12(z+z^{-1}) \b)
         = f(\phi(z)) \qquad \for z \in \SS := \{|z|=1\}.
\]
%
where $\phi(z) = \smfrac12 (z+z^{-1})$ is also called Joukowsky map. $\phi$ is
clearly analytic in $\C \setminus \{0\}$. Thus, if $f$ is analytic on $[-1,1]$
then $F$ must be analytic on $\SS$. Next, we note that analyticity of $g(\theta)$
on the strip $\Omega_\alpha$ is equivalent to analyticity of $F$ on the annulus
%
\[
   \SS_\rho := \{ z \in \C \sep \rho^{-1} \leq |z| \leq \rho \},
\]
%
with $\rho = 1+\alpha$. Let the corresponding {\em Bernstein ellipse} be the
pre-image of $\SS_\rho$ under the Joukowsky map,
%
\[
   E_\rho := \phi^{-1}(S_\rho),
\]
%
then analyticity of $f$ in $E_\rho$ implies analyticity of $F$ in $\SS_\rho$.

Finally, we recall from the derivation of the Chebyshev polynomials $T_k(x)$
that they can also be written as
\[
   \smfrac12 (z^k + z^{-k}) = T_k(\phi(z)).
\]

After these preparations, we can prove the following result.


\begin{theorem}[Decay of Chebyshev coefficients]
   Let $\rho > 1$ and $f \in A(E_\rho)$ with  $M := \|f\|_{L^\infty(E_\rho)} <
   \infty$, then the Chebyshev coefficients of $f$ satisfy
   \[
      |\tilde{f}_k| \leq 2 M \rho^{-k}, \qquad k \geq 1.
   \]
\end{theorem}
\begin{proof}
   We start with
   \[
      \tilde{f}_k = \frac{1}{\pi i} \oint_{\SS} z^{-1-k} F(z) \, dz
   \]
   Since $F$ is analytic on $\SS_\rho$ (and hence in the neighbourhood of $\SS_\rho$)
   we can expand the contour to {\it (Exercise: explain why this can be done using
   Cauchy's integral formula and a suitable sketch!)}
   \[
      \tilde{f}_k = \frac{1}{\pi i} \oint_{|z|=\rho} z^{-1-k} F(z) \, dz
   \]
   and hence we immediately obtain
   \[
      |\tilde{f}_k| \leq \frac{2\pi \rho \rho^{-1-k} M}{\pi} = 2 M \rho^{-k}.
      \qedhere
   \]
\end{proof}

Decay of Chebyshev coefficients gives the following approximation error
estimates.

\begin{theorem}[Chebyshev Projection and Interpolation Error]
   %
   \label{th:poly:err_analytic}
   %
   Let $\rho > 1$ and $f \in A(E_\rho)$ with $M := \|f\|_{L^\infty(E_\rho)} <
   \infty$, then
   \begin{align}
      \label{eq:poly:projerror}
      \| f - \PCheb_N f \|_{L^\infty(-1,1)} &\leq \frac{2M \rho^{-N}}{\rho-1}, \\
      \label{eq:poly:interperror}
      \| f - I_N f \|_{L^\infty(-1,1)} &\leq C M \log N \rho^{-N},
   \end{align}
   where $C$ is a generic constant.
\end{theorem}
\begin{proof}
   For the proof of \eqref{eq:poly:projerror} we use the fact that
   $\|T_k\|_\infty \leq 1$ to estimate
   \begin{align*}
      \| f - \PCheb_N f \|_\infty
      &\leq
      \sum_{k = N+1}^\infty |\tilde{f}_k|  \\
      &\leq
      2M \sum_{k = N+1}^\infty \rho^{-k} \\
      &=
      \frac{2M \rho^{-N}}{\rho-1}.
   \end{align*}
   The estimate \eqref{eq:poly:interperror} follows from the bound on the
   Lebesgue constant
   \[
      \| I_N \|_{L(L^\infty)} \leq C \log N,
   \]
   which follows from the analogous bound for trigonometric interpolation
   given in Theorem~\ref{th:trig:lebesgue}.

   (For a sharp bound, it is in fact known that $\Lambda_N \leq \frac{2}{\pi}
   \log(N+1) + 1$.)
\end{proof}

\begin{remark}
   One can in fact prove that
   \[
      \| f - I_N f \|_{L^\infty(-1,1)} \leq \frac{4M \rho^{-N}}{\rho-1},
   \]
   using an aliasing argument; see \cite[Thm. 8.2]{Trefethen2013-rg},
   somewhat similar to the argument we used for our convergence estimate of
   the trapezoidal rule in Exercise~\ref{exr:trig:trapezoidal rule}.
\end{remark}

\begin{example}[Fermi-Dirac Function]
   %
   \label{exa:poly:fermi-dirac}
   %
   Consider the Fermi-Dirac function
   \begin{equation}
     f_\beta(x) = \frac{1}{1 + e^{\beta x}},
   \end{equation}
   where $\beta > 0$.
 
   {\it REMARK: The Fermi--Dirac function describes the distribution of particles
   over energy states in systems consisting of many identical particles that obey
   the Pauli exclusion principle, e.g., electrons. A broad range of important
   algorithms in computational physics are fundamentally about approximating the
   Fermi--Dirac function. The parameters $\beta$ is inverse proportional to
   temperature (that is, Fermi-temperature).}
 
   Extending $f_\beta$ to the complex plane simply involves replacing $x$ with
   $z$, i.e.,
   \[
     f_\beta(z) = \frac{1}{1 + e^{\beta z}},
   \]
   which is well-defined {\em except at the poles}
   \[
       z_j = \pm i  \frac{\pi}{\beta}.
   \]

   In Exercise~\ref{exr:poly:ellipse} we show that the semi-minor axis of the 
   Bernstein ellipse $E_\rho$ is $\frac12 (\rho-\rho^{-1})$, hence the largest 
   $\rho$ for which ${\rm int}E_\rho$ does not intersect any singularity is 
   given by 
   \[ \frac12 (\rho-\rho^{-1}) = \frac{\pi}{\beta}x. \]
   Solving this quadratic equation for $\rho$ yields one positive root 
   \[ \rho = \smfrac{\pi}{\beta}+\sqrt{1 + \smfrac{\pi^2}{\beta^2}}
   \]
   Of particular interest is the low temperature regime $\beta \to \infty$ 
   (recall that $\beta \propto$ inverse temperature), for which we obtain 
   \[
      \rho \sim 1 + \smfrac{\pi}{\beta}.
   \]
   
   In this regime we therefore expect an approximation rate close to 
   \[
      \| f_\beta - I_N f_\beta \|_{\infty} 
      \lesssim \beta \b(1 + \smfrac{\pi}{\beta}\b)^{-N}
      \sim \beta \exp\b( - \pi \beta^{-1} N).
   \] 
   (Why is this not a rigorous and in fact likely false bound? You can get 
   a rigorous reformulation from the foregoing theorems.)
 \end{example}


For convergence rates for $C^{j,\sigma}([-1,1])$ and similar functions, we
want to adapt the Jackson theorems. We could again "transplant" the argument
from the Fourier to the Chebyshev setting, but it will be more convenient
this time to simply apply the Fourier results directly. The details
are carried out in Exercise~\ref{exr:poly:convergence}. We obtain
the following result.

\begin{theorem}[Jackson's Theorem(s)]
   \label{th:poly:jackson}
   %
   Let $f \in C^{(j)}([-1,1])$, $j \geq 0$, where $f^{(j)}$ has modulus of
   continuity $\omega$, then
   \begin{equation}
      \label{eq:poly:jackson1}
      \inf_{p_N \in \Poly_N} \| f - p_N \|_{L^\infty} \leq
      C N^{-j} \omega\b(N^{-1}\b).
   \end{equation}
\end{theorem}
\begin{proof}
   See Exercise~\ref{exr:poly:convergence}.
\end{proof}

We cannot yet test these predictions numerically, since we don't yet have 
a numerically stable way to evaluate the Chebyshev interpolants (or projections). 
We will remedy this in the next two sections. 
 

\subsection{Chebyshev transform}
%
We have seen in \nbpoly that naive evaluation of the Chebyshev interpolant leads
to highly unstable numerical results. The emphasis here is on the term
``naive''. Indeed, there exist at least two natural and numerically stable way
to evaluate the Chebyshev interpolant.

The first approach we consider is the Discrete Chebyshev transform (DCT), an
immediate analogy of the Discrete Fourier transform (DFT). As in the Fourier case, 
once we have transformed the polynomial to the Chebyshev basis, we can 
evaluate it in $O(N)$ operations. But in the Chebyshev case, this is even more 
efficient due to the recursion formula \eqref{eq:poly:chebrecursion}. Moreover, 
the polynomial derivatives are straightforward to compute in this case as well.


Let $F = (F_j) \in \R^{N+1}$ (we immagine that $F_j = f(x_j)$ are nodal values
of some $f \in C([-1,1])$ at the Chebyshev nodes), then there exists a unique
polynomial $p_N \in \Poly_N$ such that $p_N(x_j) = F_j$. We write $p_N(x) =
\sum_{k = 0}^N \tilde{F}_k T_k(x)$, then
\begin{equation}
   \label{eq:poly:chebtransform}
   \tilde{F} := {\rm DCT}[F] := \b( \tilde{F}_k \b)_{k = 0}^N.
\end{equation}
Since polynomial interpolation is linear and unique the operator is
an invertible linear mapping, with inverse (obviously) given by
\begin{equation}
   \b({\rm IDCT}[\tilde{F}]\b)_j = \sum_{k = 0}^N \tilde{F}_k T_k(x_j).
\end{equation}

\begin{lemma} \label{th:poly:dct_explicit}
   Let $\tilde{F} = {\rm DCT}[F]$, then
   \[
      \tilde{F}_k = \frac{p_k}{N}\bg\{
            \smfrac12 \b( (-1)^k F_0 + F_N \b)
            + \sum_{k = 1}^{N-1} F_k T_k(x_j)
         \bg\}.
   \]
\end{lemma}

We won't prove \Cref{th:poly:dct_explicit} since we won't need this expression. 
It is only stated here for the sake of completeness. The interested reader 
will be able to check it by a direct computation; it is also implicitly 
contained in the following discussion. 


{\it A priori} the cost of evaluating the DCT and IDCT is $O(N^2)$, but the 
connection between the Fourier and Chebyshev settings gives us an $O(N\log N)$
algorithm which we now derive. Let $F = {\rm IDCT}[\tilde{F}]$, then writing 
\[
   T_k(x_j) = T_k(\cos(j\pi/N)) = \cos(kj\pi/N) 
\]
we obtain 
\begin{align}
   \label{eq:poly:costtransform}
   F_j 
   &= 
   \sum_{k = 0}^N \tilde{F}_k \cos(kj\pi/N) 
   \\ \notag &= 
   \sum_{k = 0}^N \tilde{F}_k \smfrac12 \b( e^{i2\pi kj/ (2N)} + e^{-i2\pi kj/(2N)}),
\end{align}
which looks {\em almost} like a IDFT on the grid $\{-N, \dots, N\}$. We 
can rewrite this a little more, 
\begin{align*}
   F_j
   &= 
   \tilde{F}_0 + \sum_{k = 1}^{N-1} \b[\smfrac12 \tilde{F}_k\b] e^{i2\pi kj/ (2N)}
   + \tilde{F}_N \smfrac12 \b( e^{i2\pi N j/ (2N)} + e^{-i2\pi N j/ (2N)} \b) 
   \\  & \qquad 
   + \sum_{k = -N+1}^{-1} \b[\smfrac12 \tilde{F}_{-k}\b] e^{i2\pi kj/ (2N)}
   \\ &= 
   \tilde{F}_0 + \sum_{k = 1}^{N-1} \b[\smfrac12 \tilde{F}_k\b] e^{i2\pi kj/ (2N)}
   + \tilde{F}_N e^{i2\pi N j/ (2N)}
   + \sum_{k = N+1}^{2N-1} \b[\smfrac12 \tilde{F}_{2N-k}\b] e^{i2\pi kj/ (2N)}
   \\ &=: 
   \sum_{k = 0}^{2N-1} \hat{G}_k e^{i2\pi kj/ (2N)},
\end{align*}
where we have defined 
\[
   \hat{G}_k := \cases{
      \tilde{F}_k, & k = 0, \\ 
      \smfrac12 \tilde{F}_k, & k = 1, \dots, N-1, \\ 
      \tilde{F}_k, & k = N, \\ 
      \smfrac12 \tilde{F}_{2N-k}, & k = N+1, \dots, 2N-1.
   }
\] 
Let $\hat{G}[\tilde{F}]$ be defined by this expression, then we have shown 
that 
\[
   F_j = \b({\rm IDCT}[\tilde{F}]\b)_j = 
   \b( {\rm IDFT}[\hat{G}[\tilde{F}]] \b)_j, \qquad j = 0, \dots, N.
\]
After determining $F_j$ for $j = N+1, \dots, 2N-1$ we can then evaluate the 
DCT via the DFT.  From the expression \eqref{eq:poly:costtransform} we 
immediately see that 
\begin{align*}
   F_{j}
   &= 
   \sum_{k = 0}^N \tilde{F}_k \cos(kj\pi/N - 2\pi k) 
   \\ &= 
   \sum_{k = 0}^N \tilde{F}_k \cos(k2\pi(j-2N)/2N) 
   \\ &= 
   \sum_{k = 0}^N \tilde{F}_k \cos(k2\pi(2N-j)/2N)
   \\ &= 
   F_{2N-j}
\end{align*}
That is, if we define 
\[
   G_j := \cases{
      F_j, & j = 0, \dots, N, \\ 
      F_{2N-j}, & j = N+1, \dots, 2N-1
   }
\]
then we obtain 
\[
   {\rm DFT}[G] = \hat{G},   
\]
from which we can readily obtain $\tilde{F}$. 

In {\tt Julia} code an $O(N\log N)$ scaling Chebyshev transform might 
look as follows: 

\begin{verbatim}
   "fast Chebyshev transform"
   function fct(F)
      N = length(F)-1
      G = [ F; F[N:-1:2] ]
      Ghat = real.(fft(F))
      return [Ghat[1]; 2 * Ghat[2:N]; Ghat[N+1]]
   end 

   "fast inverse Chebyshev transform"
   function ifct(Ftil)
      N = length(Ftil)-1
      Ghat = [Ftil[1]; 0.5 * Ftil[2:N]; Ftil[N+1]; 0.5*Ftil[N:-1:2]]
      G = real.(ifft(Ghat))
      return G[1:N+1]
   end
\end{verbatim}


\begin{remark}
   The expression \eqref{eq:poly:costtransform} is in fact another kind of 
   well-known transform, the {\em Discrete Cosine Transform} (one of several 
   variants). A practical implementation of the fast Chebyshev transform 
   should therefore use an efficient implementation of the fast cosine transform 
   rather than the FFT.
   For the sake  of simplicity (to avoid studying yet another transformation) 
   we did not study this transform in detail, but there is plenty of literature 
   and software available on this topic. 
\end{remark}



\subsection{Barycentric interpolation formula}
%
\label{sec:poly:bary}
%
The second method we discuss is the {\em barycentric interpolation formula}.
After precomputing some ``weights'' it gives another $O(N)$ method to evaluate
the Chebyshev interpolant (or indeed {\em any} polynomial interpolant) in a
numerically stable manner. This method entirely avoids the transformation to the
Chebyshev basis. (This section is taken almost verbatim from
\cite{Trefethen2013-rg}; see also \cite[Ch. 5]{Trefethen2013-rg} for a more
detailed, incl historical, discussion).

We begin with the usual Lagrange formula for the nodal interpolant. 
Let $p(x_j) = f_j, j = 0, \dots, N$ where $p \in \Poly_N$, then  
\[
   p(x) = \sum_{j = 0}^N f_j \ell_j(x), 
   \qquad \text{where} \quad 
   \ell_j(x) = \frac{ \prod_{n \neq j} (x - x_n)}{\prod_{n \neq j} (x_j-x_n)}.
\]
This formula has the downside that it costs $O(N^2)$ to evaluate $p$ at a 
single point $x$. 

But we observe that $\ell_j(x)$ have a lot of terms in common. This can be 
exploited by defining the {\em node polynomial}
\[
   \ell(x) := \prod_{n = 0}^N (x-x_n),
\]
then we obtain 
\begin{equation} \label{eq:poly:bary_weights}
   \ell_j(x) = \ell(x) \frac{\lambda_j}{x - x_j} 
   \qquad \text{where}  \qquad 
   \lambda_j = \frac{1}{\prod_{n \neq j} (x_j - x_n)}.
\end{equation}
The ``weights'' $\lambda_j$ still cost $O(N^2)$, but they are independent of $x$
and can therefore be precomputed (Moreover, for various important sets of nodes
there exist fast algorithms. For Chebyshev nodes there is an explicit
experession; see below.). Since the common factor $\ell(x)$ does not depend on
$j$ we can now evaluate all $\ell_j(x), j = 0, \dots, N$ at $O(N)$ cost and thus
obtain the {\em first form of the barycentric interpolation formula}, 
\begin{equation} \label{eq:poly:bary1}
   p(x) = \ell(x) \sum_{j = 0}^n \frac{\lambda_j}{x - x_j} f_j.
\end{equation}
Once the weights $\lambda_j$ have been precomputed, the cost of evaluating 
$p(x)$ becomes $O(N)$. However, \eqref{eq:poly:bary1} has a different 
shortcoming: in floating point arithmetic it is prone to overflow or underflow.
Specifically, suppose that $x = -1$ and we compute $\ell(x)$ with $x_j$ 
ordered decreasingly as defined in \eqref{eq:poly:chebnodes}, then after 
approximately the first $M \approx N/4$ terms we have evaluated 
\[
   \bg|\prod_{n = 0}^M (x - x_j) \bg|
   \geq \b( \smfrac34 \b)^{M+1} 
\]
which quickly becomes very large. The issue is also reflected in the
coefficients $\lambda_j$, which for Chebyshev points are $O(2^N)$ (cf.
Exercise~\ref{exr:poly:bary}). In practise, one typically gets overflow 
beyond 100 or so grid points. 

This can be avoided with the second form of the barycentric formula: observing
that $\sum_{j = 0}^N \ell_j \equiv 1$ we obtain 
\[
   1 = \ell(x) \sum_{j = 0}^N \frac{\lambda_j}{x-x_j}, 
\]
and hence arrive at the second form of the barycentric interpolation formula:

\begin{theorem}[Barycentric interpolation formula]
   \label{th:poly:bary}
   Let $p \in \Poly_N$, with $p(x_j) = f_j$ at $N+1$ distincts points 
   $\{x_j\}$ then 
   \[
      p(x) = \frac{ 
         \sum_{j = 0}^N \frac{\lambda_j f_j}{x-x_j} 
      }{
         \sum_{j = 0}^N \frac{\lambda_j}{x-x_j}
      }, 
      \qquad \text{where} \qquad 
      \lambda_j = \frac{1}{ \prod_{n \neq j} (x_j-x_n)},
   \] 
   with the special case $p(x_j) = f_j$.
\end{theorem}

\begin{theorem}[Barycentric interpolation formula in Chebyshev Points]
   \label{th:poly:barycheb}
   Let $\{x_j\}$ be the Chebyshev points \eqref{eq:poly:chebnodes}, then 
   the barycentric weights $\lambda_j$ from \Cref{th:poly:bary} 
   may be chosen as 
   \[
       \lambda_j = \cases{ 
          (-1)^j, & j = 1, \dots, N-1, \\ 
          \frac12 (-1)^j, & j = 0, N. }
   \]
\end{theorem}
\begin{proof}
   See Exercise~\ref{exr:poly:bary}. 
\end{proof}

\subsubsection{Numerical stability of barycentric interpolation}
%
\label{sec:poly:barystab}
%
While the DFT is matrix multiplication with an othogonal matrix, and the FFT 
an algorithm that even reduced the number of operations it is natural to 
expect that these algorithms are numerically stable. By contrast, this is 
not at all obvious {\it a priori} for the barycentric formula. We will therefore 
spend a little time discussing this. 
%
To simplify this discussion we will only analyse the numerical stability 
of the {\em first} barycentric formula \eqref{eq:poly:bary1}. Understanding 
stability of the second barycentric formula is slightly more involved; 
see \cite{Higham2004-fn} for the details. 

We have to begin by explaining the standard model of floating point arithmetic. 
Let $\otimes \in \{ +, -, *,  / \}$ be one of the standard four floating point 
operations, then applying the operation $a \otimes b$  to two floating point
numbers will give an error, which we express as 
\[
   {\rm fl}\b( a \otimes b \b) = (a\otimes b)(1+\delta),
\]
where $|\delta| \leq \eps$ and $\eps$ denotes machine precision (typically
$10^{-6}$). That is, floating point arithmetic controls the {\em relative
error}. For more on this topic, in particular additional subtleties that we are
sweeping under the carpet here, see \cite{Higham2002-nk}.


For example, consider the evaluation of an inner product of two vectors 
${\bf a}, {\bf b} \in \R^2$, 
\begin{align*}
   \fl({\bf a} \cdot {\bf b})
   &= \fl\b( \fl(a_1 b_1) + \fl(a_2b_2)\b) \\ 
   &= \fl\b( a_1 b_1 (1+\delta_1) + a_2b_2 (1+\delta_2)) \\ 
   &= \b( a_1 b_1 (1+\delta_1) + a_2b_2 (1+\delta_2))(1+\delta_3) \\ 
   &= a_1 b_1 (1+\delta_1)(1+\delta_3)
      + a_2b_2 (1+\delta_2)(1+\delta_3).
\end{align*}
Upon setting 
\[
   \tilde{a_1} = a_1 (1+\delta_1), \quad 
   \tilde{b_1} = b_1 (1+\delta_3), \quad 
   \tilde{a_2} = a_2 (1+\delta_2), \quad 
   \tilde{b_2} = b_2 (1+\delta_3),
\]
we obtain 
\[
   \fl({\bf a} \cdot {\bf b}) = \tilde{\bf a} \cdot \tilde{\bf b},
\]
where $\|{\bf a} - \tilde{\bf a}\| = O(\eps)$ and $\|{\bf b} - \tilde{\bf b}\| =
O(\eps)$. This is called {\em backward stability}: the numerically evaluated
quantity is the exact quantity for an exact computation with perturbed data. 

% As a second example we can consider 
% \begin{align*}
%    \fl\bg( \frac{f(x+h) - f(x)}{h} \bg)
%    &= \frac{(f(x+h)(1+\delta_1) - f(x))(1+\delta_2)}{h} (1+\delta_3)  \\ 
%    &= \frac{f(x+h) - f(x)}{h}(1+\delta_3) + 
% \end{align*}

We can now turn to the first barycentric formula. First we consider the 
evalutation of a weight $\ell(x)$, 
\begin{align}
   \fl\B( \prod_{n=0}^N (x-x_n) \B) 
   &= 
   \fl\bg( \fl\bg( \prod_{n=0}^{N-1} (x-x_n)\bg) * \fl(x-x_N) \bg)  \\ 
   &= 
   \fl\bg( \prod_{n=0}^{N-1} (x-x_n)\bg) * (x-x_N) (1+\delta_{1}) (1+\delta_{2}),
\end{align}
and by induction 
\begin{align}
   \fl\bg( \prod_{n=0}^N (x-x_n) \bg) = 
   \ell(x) \, \prod_{m = 1}^{2N+1} (1+\delta_m). 
\end{align}
The argument for $\lambda_j$ is of course analogous, hence we obtain
with a little extra work:

\begin{proposition} \label{th:poly:barystab}
   Let 
   \[
      \tilde{p}_N(x) := \fl\bg( \ell(x) \sum_{j = 0}^N \frac{\lambda_j}{x - x_j} \bg)
   \]
   be the numerically evaluated polynomial in the standard model of 
   floating point arithmetic, then 
   \[
      \tilde{p}_N(x) = 
      \ell(x) \sum_{j = 0}^N \frac{\lambda_j f_j}{x - x_j}
      \prod_{m = 1}^{5N+5} (1+\delta_{jm}).
   \]
\end{proposition}
\begin{proof}
   This is a straightforward continuation of the calculations above. 
\end{proof}

The key point of \Cref{th:poly:barystab} is that this is a {\em backward
stability} result, i.e., let $\tilde{f}_j = f_j\prod_{m = 1}^{5N+5}
(1+\delta_{jm})$, then $\hat{p}_N$ is interpolates the values $\tilde{f}_j$. 
In particular, the error in the floating point polynomial $\hat{p}_N(x)$ 
is no larger than if we had small errors in the nodal values $f_j$, which 
we will normally have anyhow. 

Finally, for the second barycentric formula, the numerical stability result is
weaker, but one can still show that for interpolation nodes with moderate
Lebesgue constant, and reasonable functions $f$ that we are interpolating, the
numerical stability is of no concern; see \cite{Higham2004-fn} for more details.



\subsection{Applications}

\alert{todo}
\begin{itemize}
   \item spectral methods; see Sec 21 in Tref book
   \item computation of a special function?
   \item Approximating a Matrix function
   \item Chebyshev filtering 
   \item conjugate gradients 
\end{itemize}

\subsection{Exercises}
%
% IDEAS FOR COMPUTATIONAL EXERCISES: 
% - test instability of Vandermonde interpolation 
% - super-exponential convergence for entire functions ... 
%   e^x should be easy to analyse too! Also e^{-x^2}!
%   http://www.chebfun.org/examples/approx/EntireBound.html


% \begin{exercise}
%    \label{exr:poly:vandermonde}
%    The seemingly ``canonical'' approach to constructing a polynomial interpolant
%    in the monomial basis is via the linear system 
%    \[
%       c_0 + c_1 x_j + c_2 x_j^2 + \cdots + c_N x_j^N = f_j.
%    \]
%    The system matrix $V = (x_j^n)_{j,n=0}^N$ is called a {\em Vandermonde matrix}. 
   
%    1. Suppose we take $x_j$ to be equispaced points on the complex unit circle, 
%    i.e., $x_j = e^{i2\pi j/N}$, $j = 0, \dots, N-1$ (i.e., $N \to N-1$!, then 
%    show that $V$ is the discrete Fourier transform operator, in particular it 
%    is unitary (up to rescaling) and thus has condition number $\kappa(V) = 1$.

%    2. 

%    Show that 

   
%    \alert{condition number of the Vandermonde matrix}.
% \end{exercise}

\begin{exercise}[Interpolation: Existence and Uniqueness] Prove that for any
   collection of nodes $z_0, \dots, z_N \subset \C$ with $x_i \neq z_j$  for $i
   \neq j$, and nodal values $f_j$, there exists a unique interpolant $p \in
   \Poly_N$ such that $p(z_j) = f_j$. 
\end{exercise}

\begin{exercise}[Runge Phenomenon]
   \label{exr:poly:Runge Phenomenon}
   %
   For a partial explanation of the Runge phenomenon (cf \nbpoly) consider 
   the following steps: 
   \begin{enumerate} \ilist 
      \item Suppose $f \in C^{N+1}([-1,1])$. Prove that there exists 
      $\xi \in (-1,1)$ such that 
      \[
         f(x) - I_N f(x) =  \frac{f^{(N+1)}(\xi)}{(N+1)!} 
            \ell_N(x),
      \]
      where $\ell_N(x)$ is the node polynomial for the interpolation points. 
      \item Prove that for equispaces nodes, 
      $\|\ell_N\|_\infty \approx (2N)^{N} (N-1)!$.
      \item For $f(x) = 1 / (1+25 x^2)$, prove that 
       $\| f^{(N+1)} \|_\infty \| \ell_N \|_\infty  / (N+1)! \to 
       \infty$ as $N \to \infty$.

       {\it (Note this does not prove divergence but proved a strong 
       hint why divergence occurs.)}
   \end{enumerate}
\end{exercise}


\begin{exercise}[Clenshaw's Algorithm]
   %
   \label{exr:poly:clenshaw}
   %
   Let $p \in \Poly_N$ be given in the Chebyshev basis and let $x \in [-1,1]$. 
   Show that $p(x)$ can be evaluated by Clenshaw's algorithm:
   \begin{align*}
      & u_{N+1} = 0, \qquad u_N = \tilde{f}_N; \\ 
      & u_n = 2 x u_{n+1} - u_{n+2} + \tilde{f}_n, n = N-1, N-2, \dots, 0; \\ 
      & p(x) = \smfrac12 \b( \tilde{f}_0 + u_0 - u_2).
   \end{align*}
\end{exercise}

\begin{exercise}[Orthogonality of $T_k$]
   Consider the weighted space 
   \begin{align*}
      L^2_C &:= \b\{ f : (-1,1) \to \R, \text{ measurable, } 
            \|f\|_{L^2_C} < \infty \b\}, \qquad 
            \text{where} \\  
      \|f\|_{L^2_C}^2 &:= \int_{-1}^1 \frac{|f|^2}{1 - x^2} \,dx.
   \end{align*}
   Prove that $L^2_C$ is a Hilbert space and show that the Chebyshev 
   polynomials are (up to scaling) and orthonormal basis of this space. 

   Thus, conclude that the Chebyshev projection $\tilde\Pi_N$ is in fact that 
   best-approximation with respect to the $\|\cdot\|_{L^2_C}$-norm. 
\end{exercise}

\begin{exercise}[Bernstein Ellipse] 
   \label{exr:poly:ellipse}
   %
   Prove that the Bernstein Ellipse $E_\rho$, $\rho > 1$ is indeed an ellipse
   with centre $z = 0$, foci $\pm 1$, semi-major axis $\frac12 (\rho+\rho^{-1})$
   and semi-minor axis $\frac12 (\rho-\rho^{-1})$.
\end{exercise}



\begin{exercise}[Convergence Bounds]
   \label{exr:poly:convergence}
   \begin{enumerate} \ilist
   \item Complete the proof of \eqref{eq:poly:interperror} by proving
      \[
         \| I_N \|_{L(L^\infty)} \leq C \log N,
      \]
      where $I_N$ is the Chebyshev nodal interpolation operator.

   \item In preparation for the proofs of the best approximation error estimates
      for differentiable (non-analytic) functions, prove that, if $f \in
      C([-1,1])$ with modulus of continuity $\omega$, then $g \in C(\TT)$ and it
      has the same modulus of continuity.

   \item Prove Theorem~\ref{th:poly:jackson}, case $j = 0$.

   \item Let $E_N(f) := \inf_{p \in \Poly_N} \|f - p\|_\infty$. Prove that 
      \[
         E_N(f)  \leq C N^{-1} E_{N-1}(f'),
      \]
      where $C$ is independent of $N$ and try to quantify $C$.

   \item Complete the proof of Theorem~\ref{th:poly:jackson} (general $j$).
   Indeed, you should obtain a more precise formula.
   \end{enumerate}
\end{exercise}

\begin{exercise} 
   \label{exr:poly:examplefunctions}
   %
   \begin{enumerate} \ilist 
   \item For the following functions give bounds on the rate of polynomial best
   approximation in the max-norm, as sharp as you can manage: 
   \begin{enumerate} \ilist
      \item $f(x) = |\sin(5 x)|$ 
      \item $f(x) = \sqrt{|x|}$
      \item $f(x) = x (1 + 1000 (x - 1/2))^{-1/2}$
      \item $f(x) = e^{- \cos(3x)}$
      \item $f(x) = x^{100}$
      \item $f(x) = e^{-x^2}$ 
      \item $f(x) = {\rm sign}(x)$
   \end{enumerate}
   \item and for the following two functions also in the $L^2$-norm:
   \begin{itemize}
      \item $f(x) = {\rm sign}(x)$
      \item $f(x) = \sqrt{|x|}$
   \end{itemize}
   \end{enumerate}
\end{exercise}

\begin{exercise}[Barycentric Chebyshev Interpolation]
   %
   \label{exr:poly:bary}
   %
   Let $x_j$ be the Chebyshev points on $[-1,1]$.
   \begin{enumerate} \ilist
   \item In general (not only for Chebyshev points), demonstrate that $\lambda_j
      = 1 / \ell'(x_j)$.
   \item Prove that the node polynomial satisfies 
   \[
      \ell(x) = 2^{-N} \b(T_{N+1}(x) - T_{N-1}(x)\b)
   \]
   \item Show that 
   \[
      T_{N+1}'(x_j) - T_{N-1}'(x_j) = 
      \cases{ 
         2N(-1)^j, & 1 \leq j \leq N-1, \\
         4N(-1)^j, & j = 0, N.
      }
   \]
   \item Deduce that, if $\lambda_j$ is given by \eqref{eq:poly:bary_weights}, then 
   \[
      \lambda_j = \frac{2^{N-1}}{N} (-1)^j, \qquad j = 1, \dots, N-1,
   \]
   and suitably adjusted for $j = 0, N$. Explain why we can rescale the weights
   $\lambda_j$ without changing the validity of the barycentric formula, and
   hence complete the proof of Theorem~\ref{th:poly:barycheb}.
   \qedhere
   \end{enumerate}
\end{exercise}


\begin{exercise}[Coordinate Tranformations]
   \label{exr:poly:coordinates}
   %
   The purpose of this exercise is to investigate how the choice of 
   coordinate systems can expand the range of approximable functions, as 
   well as have an affect on the rate of convergence.

   The basic idea is to considier functions $F : [a, b] \to \R$ and via a
   coordinate transformation $f(x) = F(\xi(x))$ transform them to functions $f :
   [-1,1] \to \R$. This can can multiple consequences, including: (1) we can
   represent functions on an arbitrary interval (inclusinf $\R$); (2) we can
   transform functions in such a way to increase the region of analyticity and
   thus accelerate convergence.
   %
   \begin{enumerate} \ilist
      \item Consider $F(y) = e^{-2y} - 2 e^{-y}$, then $F(y) = f(e^{-y})$ where
      $f(x) = x^2 - 2x$ is a quadratic polynomial. Suppose though that this
      ``optimal'' coordinate transform $x = e^{-y}$ is not known. 

      Instead, consider the coordinate transformation $x = 2/(1+y) - 1 =
      \xi^{-1}(y)$, that is, $\xi^{-1}(0) = 1, \xi^{-1}(\infty) = -1$, and 
      let $f(x) = F(\xi(x))$. 
      \begin{enumerate} \alist 
         \item Establish an upper bound (as sharp as you can manage) for
          approximation by Chebyshev projection and interpolation of $f$ on
          $[-1,1]$ in the max-norm. 
         \item Convert this bound to an approximation result for $F(y)$ on $[0,
          \infty)$. 
         \item Can you give a simpler / more direct characterisation of the
         effective approximation space for functions on $[0, \infty)$ that you
         used here?
      \end{enumerate}
      
      \item Now consider the function $F(y) = (\eps^2 + y^2)^{-1/2}$ on 
      $[-1, 1]$. Recall the rate of convergence of Chebyshev projection and 
      Chebuyshev interpolation. 

      Now consider a coordinate transformation 
      \[
         \xi^{-1}(y) = \frac{\arctan(x/\eta)}{\arctan(1/\eta)},
      \]
      and explicitly compute its inverse. Show that $\xi, \xi^{-1} : [-1,1] \to
      [-1,1]$ are bijective. 

      \begin{enumerate} \alist 
         \item For any $\eta > 0$ establish an upper bound (as sharp as you can
         manage) for approximation of $f(x) = F(\xi(x))$ by Chebyshev projection
         and Chebuyshev interpolation.
         
         \item Discuss which choices of $\eta$ appear to be particularly good.
         Visualise the effect of $\xi$ on the function $f$ as well as on 
         the singularities in the complex plane.
      \end{enumerate}
   \end{enumerate}
\end{exercise}



\clearpage
% !TeX root = ./apxthy.tex


\section{Splines}
%
\label{sec:splines}
%
In this very short chapter we will briefly introduce and explore some
consequences of piecewise polynomial approximation (as opposed to global
polynomial approximation as in \S~\ref{sec:poly}). The basic results will be
very easy to obtain. For lack of time we will skip the more interesting
algorithmic aspects, in particular B-Splines (we will briefly define them and
show some examples, but we won't go into the implementation details, at least
not this year).

\subsection{Motivation} 
%
\label{sec:splines:motivation}
%
Let us motivate the idea of splines as follows: consider
the function $f(x) = \sqrt{x}$ on $[0, 1]$. After rescaling to $[-1,1]$ we can
approximate it with polynomials to obtain the convergence rate (cf. Jackson's
Theorem \ref{eq:poly:jackson1}) 
\[
    \inf_{p \in \Poly_N} \|f - p\|_{L^\infty(0,1)} \lesssim N^{-1/2}.
\]
This is a very slow rate of convergence, purely caused by the singularity at 
$x = 0$. But in $[1/2, 1]$ $f$ is analytic and on that interval we would 
expect 
\[
    \inf_{p \in \Poly_N} \|f - p\|_{L^\infty(1/2,1)} \lesssim \rho^{-N},
\]
for some $\rho > 1$. We can then prescribe a second polynomial on $[1/4, 1/2]$,
and so forth, thus obtaining a piecewise polynomial approximation. The
subintervals $[1/2,1], [1/4, 1/2], \dots$ are called a mesh and the flexibility
in choosing these sub-intervals can lead to very strong results. We will later
see that in this particular case we obtain almost exponential convergence.


\subsection{Splines for $C^j$ functions}
%
\label{sec:splines:Cj}
%
To work with splines we will need to construct polynomial approximations on
arbitrary sub-intervals $[a,b] \subset \R$. The Chebyshev nodes on $[a,b]$ are
simply the rescaled nodes 
\[
    x_j^{[a,b]}  = a + \frac{(x_j+1)(b-a)}{2},
\]
where $x_j$ are the Chebyshev nodes on $[-1,1]$. The resulting 
interpolation operator is denoted by $I_N^{[a,b]}$. 

We can now quantify the effect of domain size with the following lemma. 

\begin{lemma}
    Let $f \in C^{p-1,1}([a,b])$ where $a < b$, and $N \leq p$, then 
    \[
        \|f - I_N^{[a,b]} f \|_{L^\infty(a,b)} 
        \leq \frac{c^N \log N}{N!} \b(b-a\big)^N \| f^{(N)} \|_{L^\infty(a,b)},
    \]
    where $c$ is a generic constant.
\end{lemma}
\begin{proof}
    Let $g(y) = f(\xi(y))$ where $\xi(y) = a + (b-a)(1+y)/2$, i.e., 
    \[ 
        \xi : [-1,1] \to [a, b]
    \]
    is affine and bijective. Then according to Jackson's theorem (the sharp
    version; cf. Exercise~\ref{exr:poly:convergence}), 
    \[
        \|f - I_N^{[a,b]} f\|_{L^\infty(a,b)} = 
        \| g - I_N g \|_{L^\infty(-1,1)}  
        \leq  \frac{c_1^N\log N}{N!} \| g^{(N)} \|_{L^\infty(-1,1)}.
    \]
    Next, since $\xi$ is affine it is easy to show that 
    \[
        g'(y) = f'(\xi(y)) \xi'(y) = f'(\xi(y)) \smfrac{b-a}{2},
    \]
    and hence 
    \[
        g^{(j)}(y) = f^{(j)}(\xi(y)) \B(\smfrac{b-a}{2}\B)^j.
    \]
    Combining this with the interpolation error estimate for $g$ 
    yields the stated result.
\end{proof}

Thus we see that we now have two parameters to control the approximation error:
the polynomial degree $N$ and the interval lengths $(b-a)$. This extra freedom
is what can make splines a powerful alternative to polynomials. 


\begin{definition}
    Let $y_0 < y_1 < \dots < y_M$ be a partition of an interval $[y_0, y_M]$,
    then we define the space of splines (piecewise polynomials) of degree $N$ on
    that partition to be 
    \[
        \Spl_N(\{y_i\}) := \b\{ s : [y_0, y_M] \to \R, \quad 
            s|_{[y_{m-1}, y_m]} \in \Poly_N \text{ for all }
            m = 1, \dots, M \b\}
    \]
    Splines are of course $C^\infty$ in each interval $[y_{j-1}, y_j]$, but 
    sometimes it is also interesting to require that splines have a certain 
    regularity on the entire interval $[y_0, y_M]$. We therefore define 
    \[
        \Spl_N^p(\{y_i\}) := \Spl_N(\{y_i\}) \cap C^p([y_0, y_M]).
    \]
    It is worth nothing that $s \in \Spl_N^p$ implies in fact that $s \in
    C^{p,1}$.
\end{definition}

\begin{remark}
    It is of course also possible to define splines with varying polynomial
    degree, i.e. in each subinterval $[y_{j-1}, y_j]$ we might impose a degree
    $N_j$. This has advantages for some applications but we will not consider it
    here. 
\end{remark}

It takes a bit more work to construct splines of regularity $p = 1$ or higher,
but $\Spl_N^0$ splines are obtained by simply taking Chebyshev interpolants on
each sub-interval. We call the resulting interpolant $I_{N,M}$, 
\[
    I_{N,M} f(x) := I_N^{[y_{m-1},y_m]} f(x)    \qquad \text{for }
    x \in [y_{m-1}, y_m].
\]
We then obtain the following basic approximation 
error estimates. 

\begin{theorem} \label{th:splines:convergence_Cj}
    Let $f \in C^p([a,b])$ and $a = y_0 < \dots < y_M = b$ a partition of $[a,
    b]$, and let $h_m := y_m - y_{m-1})$ be the mesh size, and $N \leq p$, then 
    \[
        \| f - I_{N,M} f \|_{L^\infty(a,b)}
        \leq  C_N \max_{m = 1, \dots, M} h_m^N
        \| f^{(N)} \|_{L^\infty(y_{m-1}, y_m)},
    \]
    where $C_N = \frac{c^N \log N}{N!}$.
    In particular, if the partition is uniform, 
    $y_m = a + h m$ where $h = (b-a)/M$ then 
    \[
        \| f - I_{N,M} f \|_{L^\infty(a,b)}
        \leq C_N h^N \|f^{(N)}\|_{L^\infty(a,b)}.
    \]
\end{theorem}
\begin{proof}
    Left as an exercise. 
\end{proof}


\subsection{Splines for functions with singularities}
%
\label{sec:splines:sing}
%
We will demonstrate how splines can be used to effectively resolve singular
behaviour using the example from the beginning of this chapter, 
\[ 
    f(x) = \sqrt{x} \qquad \text{on } x \in [0, 1]
\]
A possible analytic continuation is given by 
\[
    f(r e^{i \varphi}) = \sqrt{r} e^{i \varphi / 2},
\]
which is analytic in $\C \setminus (-\infty, 0]$. Moreover, we have $|f(z)| =
\sqrt{|z|}$ which will make it easy to estimate $\|f\|_{L^\infty(E_\rho)}$ where
$E_\rho$ will be some suitable Bernstein ellipsi.

Our strategy will be to use a partition  
\[
    0, 2^{-M}, 2^{-M+1}, \dots, 2^{-1}, 1.
\]
Since $f$ is analytic in each subinterval $[2^{-m}, 2^{-m+1}]$ we will be able
to use the exponential convergence rates from
Theorem~\ref{th:poly:err_analytic}.

Let us therefore consider $f$ on $[2^{-m}, 2^{-m+1}]$. We rescale 
\[
    g(y) = f\b(2^{-m} + 2^{-m-1}(1+y)\b),
\]
then the singularity $x = 0$ maps to $y = -3$, hence $g$ in analytic in $\Re z >
-3$. In particular taking $\rho = 4$ we have $a = \smfrac12(\rho+\rho^{-1}) < 3$ 
and 
\begin{align*}
    \|g\|_{L^\infty(E_\rho)} &\leq g(a) \leq f(2^{-m} + 2^{-m-1}(1+a)) \\
    &\leq  f(2^{-m} + 2^{-m+1}) \\ 
    &\leq \sqrt{2^{-m+2}} \\ 
    &= 2^{-m/2+1}.
\end{align*}
Thus, we obtain 
\[
    \| f - I_N^{[2^{-m},2^{-m+1}]} f\|_{L^\infty(2^{-m},2^{-m+1})}
    =
    \| f - I_N g \|_{L^\infty(-1,1)} 
    \leq C 4^{-N} 2^{-m/2}
\]
To make our life a little easier we can just estimate 
\[
    \| f - I_N^{[2^{-m},2^{-m+1}]} f\|_{L^\infty(2^{-m},2^{-m+1})}
    \leq 
    C 4^{-N} \qquad \text{for } m = M, M-1, \dots, 1;
\]
that is, 
\[
    \| f - I_{N,M} f \|_{L^\infty(2^{-M}, 1)} \leq 
    C N^{-4}.    
\]

Finally, we address the first interval $[0, 2^{-M}]$. We rescale again 
as before, but now the singularity becomes part of the domain $[-1,1]$, i.e.,
$g \in C^{0,1/2}([-1,1])$ and no better. Jackson's theorem therefore tells 
us the 
\[
    \| g - I_N g \|_{L^\infty(0, 2^{-M})}
    \leq 
    C \omega_g(N^{-1}) = C N^{-1/2}.
\]
But the constant matters here! Specifically, we can show that 
\[
    \omega_g(r) = c 2^{-M/2} \sqrt{r},
\]
that is, we even have 
\[
     \|f - I_N^{[0, 2^{-M}]} f \|_{L^\infty(0, 2^{-M})} 
     \leq C 2^{-M/2} N^{-1/2}.
\]
Let us again make our life a little easier and ignore the $N^{-1/2}$ term, then 
we want to balance $2^{-M/2} = 4^{-N}$; that is, 
\[
    M = 4 N.    
\]
With this choice, we finally obtain 
\[
    \| f - I_{N,M} f \|_{L^\infty(0, 1)} \leq C 4^{-N}.
\]

To conclude we convert this into a cost estimate. The cost of evaluating 
$I_{N,M} f$ at a single point in space is the same as evaluating a 
polynomial of degree $N$, that is 
\[
    {\rm COST-EVAL}(I_{N,M} f) = O(N)
\]
and in particular, we obtain the very nice exponential convergence 
result 
\[
    \| f - I_{N,M} f \|_{L^\infty(0, 1)} \leq C \rho^{-{\rm COST-EVAL}},
\]
for some $\rho > 0$. The cost to ``build and store'' $I_{N,M} f$ is the cost of
evaluating $f$ at $M \cdot N$ points, i.e., $O(N^2)$ so this cost is a little
higher, but still very attractive.

This example is intended to demonstrate the power of adapting the spline grid to
the features of the function to be approximated. Automating this process is of
great interest but goes beyond the scope of this module.

\begin{remark}
    We can do slightly better by balancing the two terms in 
    \[
        \| f - I_N^{[2^{-m},2^{-m+1}]} f\|_{L^\infty(2^{-m},2^{-m+1})}
        \leq 
        C 4^{-N_m} 2^{-m/2}
        = C 4^{- N_m - m/4},
    \]
    i.e., choosing $N_m +  m/4 = N = {\rm const}$. But one can easily 
    check that this only gives an improvement in some constants, but 
    not qualitatively.
\end{remark}

\subsection{Exercises}


\begin{exercise} \label{exr:splines:}
    Prove Theorem~\ref{th:splines:convergence_Cj}
\end{exercise}

\begin{exercise}
    \begin{enumerate} \ilist 
    \item Suppose you are given a function $f \in C^{p-1,1}([-1,1])$. For
    simplicity, assume even that in each subinterval $[a,b] \subset
    [-1,1]$ the regularity of $f$ is no better than $C^{p-1,1}$. Assume
    you discretise $[-1,1]$ with a uniform grid. How would you optimally
    balance the grid spacing $h$ against the polynomial degree $N$?
    (i.e. minimise the error against the number of function evaluations
    you need to specify the approximant)

    \item Now suppose that $f \in A([-1,1])$; how would you balance $h$
    against $N$ now?

    \item For the following functions compare the performance of 
    global polynomial versus $\Spl_N^0$ approximation on a uniform grid:
    \begin{itemize}
        \item $f(x) = |x|$ 
        \item $f(x) = |x+\pi|$ 
        \item $f(x) = |\sin(x/2)|$ 
        \item $f(x) = (1+25 x^2)^{-1}$
        \item $f(x) = x \sin(1/x)$ 
    \end{itemize}
    \end{enumerate}
\end{exercise}

\begin{exercise}
    For the following functions $f : [-1,1] \to \R$, design a spline
    approximation with quasi-optimal rate of convergence in
    $\|\cdot\|_{L^\infty(-1,1)}$ in terms of evaluation cost. 
    \begin{itemize}
        \item $f(x) = |x|$ 
        \item $f(x) = |\sin(x/2)|$ 
        \item $f(x) = (1+25 x^2)^{-1}$
        \item $f(x) = x \sin(1/x)$ 
    \end{itemize}
\end{exercise}

\begin{exercise}[Linear Splines] Show that for we can write continuous linear
    spline interpolations, i.e. $s \in \Spl_1^0(\{y_m\})$ in terms of a nodal
    basis, 
    \[
        s(y) = \sum_{m = 0}^M f(y_m) \phi_m(y),
    \]
    where $\phi_m$ are ``hat-functions'' that you should specify 
    explicitly. 
\end{exercise}


\begin{exercise}[Hermite Interpolation with Cubic Splines]
    Let $y_0 < \dots < y_M$ be a grid and let $f_m, f_m'$ be 
    function and derivative values at those grid points. Show that there 
    exists a unique cubic spline $s \in \Spl_3^1(\{y_m\})$ such that 
    \[
        s(y_m) = f_m, \quad \text{and} \quad 
        s'(y_m) = f_m' \quad \text{for } m = 0, \dots, M.    
    \]
    {\it HINT: in each interval $[y_{m}, y_{m+1}]$ write $s(x) = f_{m} + f_{m}'
        (x-x_{m}) + a_m (x-x_m)^2 + b_m (x-x_m)^3$ and show that there exist
        unique $a_m, b_m$ such that $s(x_{m+1}) = f_{m+1}, s'(x_{m+1}) =
        f_{m+1}'$. You may wish to derive explicit expressions for 
        $a_m, b_m$ in preparation for the next exercise.}
\end{exercise}


\begin{exercise}[B-Splines]
    Depending on regularity requirements of an application it is
    sometimes advantageous to require higher regularity of the approximant,
    i.e., we should consider $\Spl_N^p$, $p > 0$. The case $\Spl_N^{N-1}$
    turns out to be particularly natural; these are alled the B-splines. And
    amongst those, the cubic splines enjoy particular polularity.

    \begin{enumerate} \ilist 
        \item Suppose for the moment that $s \in \Spl_3^2(\{y_m\})$ with 
        $s(y_m) = f_m$ where $f_m$ are some nodal values. Prove that, 
        for {\em any} $g \in C^2[a,b]$ with $g(y_m) = f_m$, 
        \[
            \int_{a}^b |s''(x)|^2 \,dx \leq \int_a^b |g''(x)|^2 \,dx,
        \]
        {\em provided} that $s$ satisfies a condition at the end-points 
        $a = y_0, b = y_M$, which you should derive. 

        Thus, $s''$ with this end-point condition minimises curvature amongst
        all $C^2$ functions satisfying the nodal interpolation conditions. 
        These splines are therefore called natural splines. 

        {\it HINT: } Consider $\int_a^b |s''|^2 + 2 s'' (g''-s'') + |s'' - g''|^2 \, dx$
        and show that the middle term vanishes if the correct end-point 
        condition is applied.

        \item Given $(f_m)_{m = 0}^M \in \R^{M+1}$, prove that there exists a
        unique $s \in \Spl_3^2(\{y_m\})$ satisfying the nodal interpolation
        conditions $s(y_m) = f_m$ and the end-point conditions found in part
        (ii). For the sake of simplicity you may wish to assume that the nodes
        are equispaced, i.e. $y_m = y_0 + h m$.

        {\it HINT: Prescribe artificial derivative values $f_m'$, then derive a
        tridiagonal linear system for $(f_m')_{m=0}^M$ and show that it has a
        unique solution. Note that this system can be solved in $O(M)$ time.}
        \qedhere
    \end{enumerate}
\end{exercise}



\clearpage
% !TeX root = ./apxthy.tex

\section{Nonlinear Approximation}
%
\label{sec:nonlin}
%

\subsection{Best polynomial approximation}
%
\label{sec:poly:bestapprox}
%
Best approximation in Hilbert spaces is a linear operation, indeed an orthogonal
projection. By contrast best approximation max-norms is far less trivial. This
section concerns the best approximation of continuous functions with polynomials
in the $L^\infty$-norm. Although this norm is not strictly convex, it turns out
that the best-approximant is still unique. Moreover, its characterisation leads
to an algorithm (the Remez algorithm). The high cost of the Remez algorithm an
together with the fact that Chebyshev interpolation (or projection) typically
gives accuracy very close to best-approximation means this is rarely used in
practise, however the mathematics is still interesting and worth studying.
Moreover, this is our first non-trivial example of a {\em non-linear
approximation algorithm}.

\alert{maybe move this to a section on "nonlinear approximation"?}

\begin{theorem} \label{th:poly:bestapprox}
   Let $f \in C([-1,1])$, then there exists a unique best approximation 
   $p \in \Poly_N$ such that $\|f - p \|_\infty \leq \|f-q\|_\infty$ for 
   all $q \in \Poly_N$. 

   A polynomial $p \in \Poly_N$ is the best approximation if and only if 
   it equioscillates at (at least) $N+2$ points $y_0 < \dots < y_{N+1}$; 
   that is, 
   \[
      (f-p)(y_j) = \pm (-1)^j \|f-p\|_\infty.
   \] 
\end{theorem}
\begin{proof}
   test
   {\it 1. Existence: } This is covered in
   Exercise~\ref{exr:prelims:bestapprox}. Let $E := \inf_{p \in \Poly_N}
   \|f-p\|_\infty$.
   
   {\it 2. Equi-oscillation implies optimality: } Suppose $p$ satisfies 
   the equi-oscillation property and $q \in \Poly_N$ such that 
   $\|f-q\|_\infty < \|f-p\|_\infty$. Without loss of generality, we then have 
   \begin{align*}
      (f-q)(y_j) &< (f-p)(x_j), \qquad j \text{ even},  \\ 
      (f-q)(y_j) &> (f-p)(x_j), \qquad j \text{ odd}, \\ 
   \end{align*}
   and hence 
   \begin{align*}
      (p-q)(y_j) &> 0, \qquad j \text{ odd}, \\ 
      (p-q)(y_j) &< 0, \qquad j \text{ even}.  \\ 
   \end{align*}
   Consequently $p-q$ has at least $N+1$ roots, which means that $p - q = 0$.
      
   {\it 3. Optimality implies equi-oscillation: } Let $p \in \Poly_N$ and
   suppose there exist {\em at most} $M < N+2$ points $y_1 < \dots y_M$ at which
   $f-p$ equi-oscillates. Without loss of generality, assume that $(f-p)(y_1) =
   -E$, then we can find points  
   \[ 
      z_1 \in (-1, y_1), z_2 \in (y_1, y_2), \dots, 
      z_M \in (y_{M-1}, y_M), z_{M+1} \in (y_M, 1)
   \]
   such that 
   \begin{align*}
      (f-p) &< E, \qquad \text{in } [-1, z_1], [z_2, z_3], \dots 
      (f-p) &> E, \qquad \text{in } [z_1, z_2], [z_3, z_4], \dots 
   \end{align*}
   Now, let 
   \[
      \delta p(x) := (z_1 - x)(z_2-x)\cdots(z_{M+1}-1), 
   \]
   then we readily see that 
   \[
      \|f - (p + \eps \delta p)\|_\infty < \|f - p\|_\infty
      \qquad \text{for $\eps$ sufficiently small.}
   \]
   Thus, $p$ was not optimal. 
   
   {\it 4. Uniqueness: } Suppose that $p, q$ are both best approximations, then
   $r := (p+q)/2$ is a best approximation as well. Let $y_j$ be the
   equi-oscillation points. $|r(y_j)| = E$ is only possible if $p(y_j) = q(y_j)
   = \pm E$. Thus $q, p$ agree at $N+2$ points and are therefore equal.
\end{proof}

Interestingly, then proof is semi-constructive and with a bit of immagination 
gives rise to the following (not quite an) algorithm:

{\bf Remez Algorithm:} Input: $f, N$
\begin{enumerate}
\item Choose initial interpolation nodes $x_0 < \dots < x_N$. E.g., Chebyshev 
nodes are a canonical choice. 
\item Solve the system 
\[
      b_0 + b_1 x_j + \dots + b_{N} x_{j}^{N} + (-1)^j E = f(x_j), 
      \qquad j = 0, \dots, N+1
\]
for the $n+2$ unknowns $b_i, E$.
\item 
\end{enumerate}

% A complete analysis of the Remez algorithm is a little involved


%%%%%%%%%%%%%%%%%%%%%%%%%%%%%%%%%%%%%%%%%%%%%%%%%%%%%%%%%%%%%%%%%%%%%%%%%%
%%%%%%%%%%%%%%%%%%%%%%%%%%%%%%%%%%%%%%%%%%%%%%%%%%%%%%%%%%%%%%%%%%%%%%%%%%
%%%%%%%%%%%%%%%%%%%%%%%%%%%%%%%%%%%%%%%%%%%%%%%%%%%%%%%%%%%%%%%%%%%%%%%%%%
\bibliographystyle{alpha}
\bibliography{apxthy}


\end{document}









\subsection{Rational Approximation by Example}



%%%%%%%%%%%%%%%%%%%%%%%%%%%%%%%%%%%%%%%%%%%%%%%%%%%%%%%%%%%%%%%%%%%%

\clearpage



\clearpage

\section{Splines}

\clearpage





\section{Approximation with Tensor Products}



\section{Fitting}

\subsection{Least Squares}

\subsection{Greedy Algorithms}



%%%%%%%%%%%%%%%%%%%%%%%%%%%%%%%%%%%%%%%%%%%%%%%%%%%%%%%%%%%%%%%%%%%%
%%%%%%%%%%%%%%%%% PART II %%%%%%%%%%%%%%%%%%%%%%%%%%%%%%%%%%%%%%%%%%
%%%%%%%%%%%%%%%%%%%%%%%%%%%%%%%%%%%%%%%%%%%%%%%%%%%%%%%%%%%%%%%%%%%%
\clearpage

\vspace{5cm}

{\huge Part II: Topics in Multivariate Approximation}

\clearpage

%%%%%%%%%%%%%%%%%%%%%%%%%%%%%%%%%%%%%%%%%%%%%%%%%%%%%%%%%%%%%%%%%%%%


\section{Sparse Grids and Tensor Networks}


\section{Radial Basis Functions and GPs}


\section{Ridge Functions and Neural Networks}




% Exam Ideas:
% - Q1: basic facts about Part I
% - Qx: Something about the Fejer Kernel:
%   https://en.wikipedia.org/wiki/Fejér_kernel
% - spectral differentiation
% - barycentric formula for trig interp


% \begin{exercise}[Piecewise Smooth Functions] \label{exr:trig:pwsmooth}
%   In this exercise we will develop an $L^2$-error estimate for
%   piecewise smooth functions.
%   \begin{enumerate} \ilist
%     \item Let $g(x) := \sin(x/2)$, continued periodically. With a similar
%     argument as in the previous exercise prove that
%       $\|g - \Pi_N g \|_{L^2} \lesssim N^{-1/2}$.
%     %
%     \item Let $f \in C^{p-1}(\TT)$ with $f^(p-1)$ piecewise $C^{\infty}$
%     (technically, $C^{p,1}$ suffices) with finitely many points $\{x_j\}_{j=1}^J
%     \subset (-\pi, \pi]$ where $f^{(p)}$ jumps. Prove that there exist
%     $a_j \in \C$ such that
%     \[
%       f(x) - \sum_{j = 1}^J g(x - x_j) \in C^{p,1}(\TT).
%     \]
%     %
%     \item Deduce that
%     \[
%         \|f - \Pi_N f \|_{L^2} \lesssim N^{-p-1/2}. \qedhere
%     \]
%   \end{enumerate}
% \end{exercise}
