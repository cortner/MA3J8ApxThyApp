%% This is a sample for an exam with 1 compulsory question.
%% Needs v1.9 of exams.cls to support examcomp1 environment
%%
%
\documentclass{exams}
%
% If you want \Bbb or \frak
\usepackage{amsfonts, amsmath}
%
% Use other packages as needed.
% Packages compatible with article class ought to work OK.


%%
%% Process your file twice so that the format can
%% get the CONTINUED and END on the correct pages.
%%


%%
%% This data generates the first page header and
%% the running head
%%
\examcode{MA 3J8}
\examyear{3} % 1-4 for UG, 5=MSc
\examdate{June 2019}
\examtitle{Approximation Theory and Applications} % In CAPS
               %
\examlength{4} % This is the number of questions for full marks.
               % For UG papers the actual number is normally
               % one more than this.
               %
\examtime{3}   % This is the length of exam in hours. It is optional.
               % If not supplied a length of \examlength x .5 hours
               % will be assumed.
               %
%
\marks         % Uncomment this if you put marks in margin.
               % A warning will be issued if you use \mark and you don't.
               %
%\resit         % Uncomment this for a resit exam.
%\calculators   % Uncomment this if calculators are allowed.


%%
%% You can number parts of a question using
%% the enumerate environment. Adjust its appearance
%% as follows if you prefer roman numerals for the
%% parts. The default is lower case letters a), b),...
%%
\renewcommand{\theenumi}{\alph{enumi}}
\renewcommand{\labelenumi}{\theenumi)}
%% If you need more than one level of enumerates
%% then define the format for those, too.
\renewcommand{\theenumii}{\roman{enumii}}
%%



%%
%% Define your own macros, e.g.
%%
\newcommand{\R}{\mathbb{R}}
\newcommand{\C}{\mathbb{C}}
\newcommand{\N}{\mathbb{N}}
\newcommand{\Q}{\mathbb{Q}}
\newcommand{\Z}{\mathbb{Z}}
\newcommand{\g}{\mathfrak{g}}

\begin{document}

%% Marks can be displayed for parts of a question using the \mark{}
%% macro at the end of a line where the marks should appear. Add
%% \marks to the preamble to put a comment on this in the rubric.
%% A warning will be issued if you use \mark{} without adding
%% \marks to the preamble.

\def\TT{\mathbb{T}}

\begin{examcomp1}

\clearpage

\begin{question}
  \begin{enumerate}

  \item  (i) Define the space of trigonometric polynomials of degree $N$. State
  the orthogonality property that the canonical basis satisfies.  \mark{3} 

  (ii) For $f \in C(\TT)$, (continuous, $2\pi$-periodic), define the
  Fourier coefficients $\hat{f}_k$, the Fourier series, as well as the
  $N$-partial sum $\Pi_N f$.  \mark{3}
  
  (iii) Use the orthogonality property from (i) to prove that  \mark{3}
  \[
      \| f \|_{L^2(\TT)}^2 = \sum_{k \in \Z} |\hat{f}_k|^2  
  \]
  
  (iv) Prove that $\Pi_N f$ is the best approximation to 
  $f$ in the $L^2(\TT)$-norm.  \mark{3}

  \item  Derive an
  error estimate for $\| \Pi_N f - f \|_{L^2(\TT)}$, where  \\ 
  (i) $|\hat{f}_k| \leq C |k|^{-p}, p \geq 1$; \mark{2} \\  
  (ii) $|\hat{f}_k| \leq C e^{-\alpha |k|}$. \mark{2}  

  (iii) In each of the two cases (i, ii) state a regularity assumption on 
  $f$ that gives rise to this decay in the Fourier coefficients. \mark{2}

  (iv) For the following functions state without proof estimates on the rate of
  convergence for $\| \Pi_N f - f \|_{L^2(\TT)}$ based on (i, ii, iii):  \mark{8}
  \begin{align*}
    f_1(x) &= |\sin(5x)|^5
    & 
    f_2(x) &= (1 + \cos^2(x))^{-1}
    \\ 
    f_3(x) &= \sin(\cos(x))
    &
    f_4(x) &= x \sin^2(x)
  \end{align*}
  {\it All functions are defined for $x \in (-\pi, \pi]$ and 
  extended $2\pi$-periodically.}

  \item Let $\mathcal{P}_N$ denote the space of algebraic polynomials of degree
  up to $N$ and let $E(\rho), \rho > 1$ denote a Bernstein ellipse. If $f$ is
  analytic in $E(\rho)$ and $M = \|f\|_{L^\infty(E(\rho))}$ then  
  \begin{equation} \label{eq:approxcheb}
    \inf_{p \in \mathcal{P}_N} \|f - p\|_\infty \leq \frac{2 M \rho^{-N}}{\rho-1}.
  \end{equation}
  
  (i) Define the set $E(\rho)$. \mark{2}

  (ii) Using \eqref{eq:approxcheb}, establish an explicit super-exponential 
  convergence rate for $f(x) = e^x$, specifying all constants. {\it A 
  sharp estimate is not required.} \mark{6}

  \item (i) Given interpolation nodes $x_0 < \dots < x_N \in [-1,1]$, define the
  associated nodal interpolation operator $I_N$ for polynomials. {\it You need 
  not show that it is well-defined.}  \mark{2}

  (ii) Prove an interpolation error estimate of the form  \mark{4}
  \[
    \|f - I_N f \|_{L^\infty(-1,1)} \leq C_N 
     \inf_{p \in \mathcal{P}_N} \|f - p\|_{L^\infty(-1,1)},
  \]
  where $C_N$ should be precisely defined but need not 
  be calculated or estimated. Properties of $I_N$ that you 
  use should be stated, but need not be proven.
  \end{enumerate}
\end{question}

\clearpage

\begin{question}
  \begin{enumerate}

  \item (i) Let $f \in C(\TT)$ (continuous and $2\pi$-periodic). 
  Define what it means for $f$ to have a modulus of continuity $\omega$. 
  \mark{1} 

  (ii) State Jackson's first theorem for the approximation of $f$ in the
  max-norm by trigonometric polynomials. \mark{2} 

  \item Let $f \in C([-1,1])$ with modulus of continuity $\omega$.
  
  (i) Use a suitable coordinate transformation to derive the Chebyshev nodes 
  and Chebyshev polynomial basis from equispaced nodes and the canonical trigonometric 
  polynomial basis. {\it (You need not prove that the transformed 
  functions are in fact algebraic polynomials.)}  \mark{3}
  
  (ii) Derive an estimate for the approximation of $f$ in the max-norm by
  algebraic polynomials from the result in (a)(ii).  \mark{5}

  (iii) If $f \in C^1([-1,1])$, give a sharp bound for its modulus of continuity
  $\omega$. \mark{1}

  \item (i) Let $E_N(f)$ denote the best approximation error of $f \in C([-1,1])$ by
  algebraic polynomials of degree $N$. Prove that, if $f \in C^1([-1,1])$, then
  there exists a generic constant $C$ such that  \mark{3}
  \[
      E_N(f) \leq C N^{-1} E_{N-1}(f').
  \]
  {\it HINT: use (b)(ii) and (b)(iii) and the fact that 
    $E_N(f) = E_N(f+q)$ for all $q \in \mathcal{P}_N$. }

  (ii) Let $f \in C^j([-1,1]), j \geq 1$ where $f^{(j)}$ has modulus of
  continuity $\omega$. Prove a sharp max-norm approximation error estimate of
  the form
  \mark{5}
  \[
      \inf_{p \in \mathcal{P}_N} 
      \big( \|f - p \|_\infty + \| f' - p' \|_\infty\big) \leq 
      C \epsilon(N),
  \]
  where $\epsilon(N)$ is a rate that you should specify.

  % (iii) Now suppose that $f$ is analytic in the Bernstein ellipse $E(\rho)$. 
  % Establish an analogous max-norm approximation error estimate.  \mark{4}
  \end{enumerate}
\end{question}

\clearpage


\begin{question}
  \begin{enumerate}

  \item For $f \in C^{N+1}([a, b])$, $N > 0$ prove that there exists  \mark{4}
  a constant $C_N$ and a polynomial $p_N \in \mathcal{P}_N$ such that 
  \begin{align*}
    \| f - p_N \|_{L^\infty(a,b)} 
    &\leq C_N (b-a)^{N+1} \| f^{(N+1)} \|_{L^\infty(a,b)}, 
    \qquad \text{and} \\ 
    \| f' - p_N' \|_{L^\infty(a,b)}
    &\leq C_N (b-a)^{N} \| f^{(N+1)} \|_{L^\infty(a,b)}. \\ 
  \end{align*}
  \vspace{-12mm}

  {\it You may use, without proof that, for $g \in C^{(N+1)}([0,1])$,  
  there exists $q_N \in \mathcal{P}_N$ such that }
  \[
      \| g - q_N \|_{L^\infty(0,1)} 
      + \| g' - q_N' \|_{L^\infty(0,1)}
      \leq C_N \| f^{(N+1)} \|_{L^\infty(0,1)}.
  \]  

  \item (i) Define the spaces $\mathcal{S}_N((y_m)_{m=0}^M)$ of splines of
    degree $N$ and  $\mathcal{S}_N^p((y_m)_{m=0}^M)$ of splines of degree $N$,
    regularity $C^p$ and nodes $y_m$. \mark{2}

    (ii) For $y_m = h m$ where $h = 1/M$ prove that  \mark{3}
    \begin{equation} \label{eq:splineapproxerr}
        \inf_{s \in \mathcal{S}_N} \|f - s \|_{L^\infty(0,1)}
        \leq 
        C_N h^{N+1} \| f^{(N+1)} \|_{L^\infty(0,1)}
    \end{equation}

    (iii) Let $f_1(x) = \cos(x)$ and $f_2(x) = |\cos(2x)|^3$. 
    For each of $f_1, f_2$ and for each $M \in \N$ and $N \in \N$ give a
    sharp upper bound for the approximation error \eqref{eq:splineapproxerr}.
    {\it You need not give a rigorous proof.}
    \mark{3}

  \item (i) Still consider $y_m = h m$, $h = 1/M$. Given $f \in C^1([0,1])$
  prove that there exists a unique spline $s := I_h f \in \mathcal{S}_3^1$
  satisfying  \mark{4}
  \[
      s(y_m) = f(y_m) \quad \text{and} 
      \quad 
      s'(y_m) = f'(y_m) \qquad \text{for } m = 0, \dots, M.
  \]
  {\it HINT: For $x \in [y_{m-1}, y_m]$, write 
  $s(x) = f_{m-1} + f_{m-1}' (x-y_{m-1}) + a_m (x-y_{m-1})^2 
    + b_m (x - y_{m-1})^3$, then show that you can uniquely 
    determine $a_m, b_m$.}

  (ii) Prove that there exists a constant $C_H$ such that, 
  for $f \in C^3([0,1])$,   \mark{4}
  \[
      \| f - I_h f \|_{L^\infty(0,1)} \leq C_H h^4 \|f^{(4)}\|_{L^\infty(0,1)}
  \]  
  {\it You may use the following identity without proof:
    \[
      \begin{pmatrix} 
        h^2 & h^3 \\ 
        2h & 3h^2 
      \end{pmatrix}^{-1}
      = 
      h^{-4} 
      \begin{pmatrix}
        3 h^2 & - h^3 \\ 
        -2h & h^2 
      \end{pmatrix}
    \] 
  }

  \end{enumerate}
\end{question}

\clearpage

\begin{question}
  \begin{enumerate}

  \item (i) Consider the uniform grid $x_n := n \pi / N$. Prove the identity  \mark{2}
  \begin{equation}\label{eq:aliasing}
      \frac{1}{2N} \sum_{n = 0}^{2N-1} e^{i k x_n} = 
      \begin{cases}
          1, & k \in 2N \Z, \\ 
          0, & \text{otherwise}.
      \end{cases}
  \end{equation}
  
  (ii) For $f \in C(\TT)$ define the trigonometric interpolant $I_N f \in
  \mathcal{T}_N'$ on the nodes $x_n$. Define also the modified space of
  trigonometric polynomials $\mathcal{T}_N'$ with only $2N$ degrees of freedom.
  Motivate this modification by referring to \eqref{eq:aliasing}.
  \mark{4}

  {\it (You may assume throughout the remainder of the question that $I_N f$ is
  well-defined.)}

  (iii) Let $F = (F_n)_{n = 0}^{2N-1}$, where $F_n = f(x_n)$, and define the
  Discrete Fourier transform  $\hat{F} \in \R^{2N}$. Briefly explain how this can be
  used to evaluate the trigonometric interpolant $I_N f$.  \mark{4}
  
  \item Now consider a finer equi-spaced grid $y_m = m \pi / M$, $m = 0, \dots,
  2M-1$ where $M > N$ is an integer multiple of $N$.

  (i) Formulate the least squares problem to fit a trigonometric polynomial $t
  \in \mathcal{T}_N'$ to data $(y_m, f(y_m))$ with weights $w_m = 1$, in the
  form $\min \|A c - Y \|_2^2$ where $c$ are the coefficients of $t$.
  \mark{3}

  (ii) Derive the normal equations and explicitly
  compute the $2N \times 2N$ system matrix that needs to be inverted to solve
  them. 
  \mark{3}

  (iii) Deduce an $O(M \log M)$-scaling algorithm to solve the least squares
  problem from (i). \mark{4}
  \end{enumerate}
\end{question}

\clearpage

\begin{question}
  \begin{enumerate}

  \item Consider a multi-variate polynomial in a tensor 
  product Chebyshev basis, 
  \[
      p({\bf x}) = \sum_{{\bf k} \in \{0,\dots,N\}^d} c_{\bf k} T_{\bf k}({\bf x}), 
      \qquad \text{where } {\bf x} \in \R^d.
  \]
  (i) Define what we mean by $T_{\bf k}({\bf x})$, where 
  ${\bf k} \in \mathbb{N}^d, {\bf x} \in \R^d$.  \mark{1}
  
  (ii) Estimate the computational cost of evaluating $p({\bf x})$ at a single
  point ${\bf x}  \in \R^d$.
  \mark{3}

  (iii) Suppose that for all $i = 1, \dots, d$, $x_i \mapsto f({\bf x})$ is analytic in $E(\rho)$ (with $x_j, j \neq i$
  remaining fixed) where $E(\rho)$ is the usual Bernstein ellipse. Prove the following tensor product polynomial
  approximation in the max-norm, defining what we mean by $\tilde{\Pi}_N^{(i)}$,
  \mark{5}
   \[ 
      \|f - \tilde\Pi_N^{(d)} \cdots \tilde\Pi_N^{(1)} f \|_\infty 
      \leq C (\log N)^d \rho^{-N}.
   \]
  {\it You may use \eqref{eq:approxcheb} without proof, as well as 
  the bound $\| \tilde\Pi_N g \|_\infty 
  \leq C \log N \|g\|_\infty$ for $g \in C([-1,1])$.}
  
  (iv)  Convert the result from (iii) into an estimate in terms of the cost
  of storing and evaluating the multivariate polynomial, and thus explain the
  term ``curse of dimensionality''. \mark{3}
  
  \item Consider a function $f \in C([-1,1]^d) \to \R$ with its multivariate 
  Chebyshev expansion coefficients satisfying 
  \[
      \sum_{{\bf k} \in \N^d} \prod_{i=1}^d (1+k_i)^2 |f_{\bf k}| =: M_f < \infty.
  \]
  (i) Show that there is a non-tensor truncation of this expansion
  that leads to the error estimate  \mark{3}
  \[
      \| f - p_N \|_\infty \leq M_f (1+N)^{-2}.
  \]
   (ii) By estimating the number of retained coefficients, convert the error 
   estimate in (i) to an estimate in terms of the cost of evaluating the 
   approximant $p_N$. 
   
  {\it You may use, without proof that, if $c_0 n \log n \leq m \leq c_1 n \log n$ then 
  for $m, n$ sufficiently large, 
  $n \geq c_2 m /
  \log m$ where $c_j > 0$. }
   \mark{5}

  %  {\it HINT: Consider first the case $d = 2$.}
  \end{enumerate}
\end{question}


%
%% \end{exam1comp} will automatically print END after the last question.
%% Run LaTeX twice to get this right.
\end{examcomp1}

%%
%% Some optional extras...
%%
\clearpage 

{\bf Comments for external examiner: } (see also solutions) \\ The exam has a
mixture of testing knowledge, proofs, minor variations of proofs (to test
understanding), a few concrete examples. This is a new module, so there is no
risk of overlap with previous exams. Except on Q1, at most 5 points on each Q
require deeper modification of the ideas learned in the lectures.

\begin{itemize}
\item Q. 1 is meant to be easy and check the most basic knowledge and
understanding of the material. It is almost exclusively bookwork (BW), only
(biv)  and (cii) are (straightforward) variations of examples seen in the
lectures. Every student who has studied seriously should be able to 
obtain about 80--90\% of this question.
\item Q.2: (a) is BW and tests one of the fundamental results covered in the lectures;
(b) is BW and tests a key idea emphasized several times; (ci) is 
an exercise; (cii) is variation of an exercise.
\item Q.3: (a) is 1/2 BW and 1/2 a trivial extension of the same idea; 
(bi) and (bii) are BW testing the basic results on splines; (biii) is 
an immediate application with similar examples seen in the lectures and 
exercises; (ci)  is an exercise; (cii) is new and requires a clear head - 
the main point is to connect this with the concept of the Lebesgue 
constant, then it is easy (but there are also less direct ways to prove this).
\item Q.4: This exercise mixes knowledge about trigonometric interpolation 
and least squares methods. (a) is standard BW; (bi) is still bookwork; 
(bii) is a direct application of (ai) and a minimal variation of 
what was seen in the lectures; (biii) requires them to realise that a 
certain operation can be performed via the FFT. It requires them to make 
a connection but is technically not difficicult.
\item Q.5: (a) is almost verbatim bookwork; (b) is new: (bi) is an application of 
the greedy truncation idea discussed in the lectures; (bii) 
requires a bit of legwork, but similar calculations have been performed in 
the lectures many times.
\end{itemize}

% \begin{itemize}

% \item Q. 2: (a) (i) and (ii) is straightforward BW, (iii) tests a bit of
% understanding and creativity, (iv) is a straightforward application;
% (b) (i) is a straightforward application of standard techniques,
% (ii) tests a bit of understanding and creativity.

% \item Q. 3: (a) is BW; (b) is relatively straightforward application of
% bookwork, with similar examples seen. The main step is to convert the linear
% system into the companian matrix form, which the students have seen in the
% lectures. The rest should be straightforward applications of learned techniques.

% \item Q. 4: (a) (i) is BW, (ii) is an application of (i) which the students
% haven't seen, but if they don't spot it they can still use the result. (iii) is
% BW. (b) is BW, but slightly different from how it was covered in the lectures,
% it tests whether the students have understood how to complete a square in the
% vectorial setting. (c) is new, it tests that the students can apply the
% technique from (b) in a setting similar to what they have seen in the lectures
% and revisions.

% \item Q. 5: (a) is a simple variation of material covered in the lectures.

% (b) the modification $u = 0$ makes the problem closer to the material of the
% lectures, but it will be non-obvious to most students since the concept
% of controllability in the lectures suddenly becomes the concept of
% observability, which they have seen in a different context. If the students
% can spot this then the rest of this question should be straighforward. Even
% without this, they will be able to get 6-8 points here.

% \end{itemize}


%% Typeset solutions can be provided and these will follow the standard layout

% \begin{solution}

% Test

% \end{solution}

% \begin{solution}

% Test

% \end{solution}

\end{document}

%%% Local Variables:
%%% mode: latex
%%% TeX-master: t
%%% End:
